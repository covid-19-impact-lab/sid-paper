\noindent Governments worldwide are adopting nuanced policy measures to reduce the
number of Covid-19 cases and to balance social and economic costs. Epidemiological
models have a hard time predicting the effects of fine-grained policies. We propose a
novel simulation-based model to address this shortcoming. We build on state-of-the-art
agent-based simulation models but replace the way contacts between susceptible and
infected people take place. Firstly, we allow for heterogeneity in the types of contacts
(e.g. recurrent or random) and in the infectiousness of each contact type. Secondly, we
strictly separate the number of contacts from the probabilities that a contact leads to
an infection. The number of contacts changes with social distancing policies, the
infection probabilities remain invariant. This allows us to model many types of targeted
policies that cannot be incorporated into other models. To validate our model, we show
that it can predict the effect of the German November lockdown even if no similar policy
has been observed during the time period that was used to estimate the model parameters.

\vspace{1cm}
\noindent JEL Classification: C63, I18

\noindent Keywords: Covid-19, agent based simulation model, public health measures
