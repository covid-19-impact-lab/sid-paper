\subsection{Reducing Numbers of Contacts Through Policies}
\label{sec:policies}

The main motivation of our model is to predict the effect of policies that affect the
number of contacts people have. Examples range from school closures and lockdowns to
more nuanced policies such as mandatory quarantines for symptomatic individuals or a
class splitting policy where only half of the students come to school in person and the
other half joins digitally with weekly rotation.

Instead of thinking of policies as completely replacing how many contacts people have,
it is often more helpful to think of them as adjusting the pre-pandemic number of
contacts.

Therefore, we implement policies as a step that happens after the number of contacts is
calculated but before individuals are matched.

On an abstract level, a policy is a functions that modifies the number of contacts of
one contact type. For example, school closures simply set all school contacts to zero. A
lockdown where only essential workers are allowed to work means that approximately two
thirds of the working population have zero work contacts and the rest has the same
number of contacts as before.

This, in conjunction with our fine-grained contact types, allows us to easily implement
a wide variety of policies. Allowing policies to depend on the health states of the
entire population means that adaptive lockdowns where, for example, schools close when a
certain threshold of infections is surpassed at the county level would be as simple as
determining which counties are above the threshold and then setting all school contacts
in these counties to zero.

The dependency of policies on health states also makes it possible to model contact
tracing. For example, a policy could check whether each child has a classmate who's
received a positive test result and then bar all children of that class from attending
school.

Some policies can be easily implemented if the background characteristics are suitably
extended. For example, a schooling policy with A/B schooling, where each half attends
school every other week can be implemented by storing whether the child would attend in
even or odd weeks in the background characteristics and then using that information in
the policy function.

For some policies the exact effect on each contact type is not easy to determine. If
this refers to a policy during the estimation period, it is possible to estimate such
parameters by fitting the model to time series data of infection rates. This is only
possible if the policy was not active during the whole estimation period and thus the
infection probabilities can be identified separately. If instead it refers to a policy
that we want to simulate, we make a scenario analysis in which the model is simulated
with several assumptions about how the policy affects the number of contacts.
