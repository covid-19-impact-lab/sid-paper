\section{Additional Results}
\label{sec:additional_results}

\subsection{Simulated vs. Empirical Data}
\label{subsec:new_cases_fit}

This compares simulated data from our model with empirical data from Germany. We look at
observed infections, fatality rates, the spread of the B117 mutation, vaccinations and
rapid test demands. Where available we do not only look at aggregated statistics but also
analyze the model fit for age groups.\comment[id=J]{summarize the
fit}\comment[id=HM]{Also need to show simulated total infections somewhere. So far only
ever shown for 2021.}


\begin{figure}[ht]
  \centering
  \includegraphics[width=\textwidth]{../figures/results/figures/scenario_comparisons/combined_fit/full_new_known_case_with_single_runs}
  \caption{Simulated and Empirical Infections}
  \figurenotes{The figure shows the weekly incidence rates per 100,000 people for the
  reported versus the simulated infections rates.}
  \label{fig:aggregated_fit2}
\end{figure}


\begin{figure}[ht]
  \centering
  \includegraphics[width=\textwidth]{../figures/results/figures/incidences_by_group/age_group_rki/full_combined_baseline_new_known_case}
  \caption{Simulated and Empirical Infections by Age Group}
  \figurenotes{The figure shows the weekly incidence rates per 100,000 people for the
  reported versus the simulated infections rates for different age groups.}
  \label{fig:age_group_fit}
\end{figure}


\begin{figure}[ht]
  \centering
  \includegraphics[width=\textwidth]{../figures/results/figures/incidences_by_group/state/full_combined_baseline_new_known_case}
  \caption{Simulated and Empirical Infections by Federal State}
  \figurenotes{The figure shows the weekly incidence rates per 100,000 people for the
  reported versus the simulated infections rates for different federal states.}
  \label{fig:state_fit}
\end{figure}


\FloatBarrier


\subsection{Share of Cases that are Detected  \comment[id=K]{Figure notes missing}}
\label{subsec:appendix_share_known_cases}

\begin{figure}[ht]
  \centering
  \includegraphics[width=\textwidth]{../figures/results/figures/share_known_cases/full_combined_baseline_by_age_group_rki}
  \caption{Share of Detected Cases by Age Group in the Main Prediction}
  \label{fig:share_known_cases_by_age_group}
  \floatfoot{\noindent}
\end{figure}

It's noteworthy that the share of detected cases increases rapidly in May for the five to
fourteen year olds. This is a direct result of the mandatory tests in
school.

\subsection{Rapid Tests \comment[id=K]{The figures will likely change layout later}}
\label{subsec:appendix_rapid_tests}

\begin{figure}[ht]
  \centering
  \includegraphics[width=\textwidth]{../figures/results/figures/rapid_test_statistics/demand_shares}
  \caption{Share of the Population Demanding a Rapid Test for a Particular Reason}
  \label{fig:rapid_tests_by_reason}
  \floatfoot{\noindent Note that the lower three (household demand, symptomatic without
  PCR demand and other contact demand together form the private demand category. Also
  note that these do not add up to the total share of demanded rapid tests as individuals
  may have more than one reason to demand a rapid test on any given day. }
\end{figure}

\begin{figure}[ht]
  \centering
  \includegraphics[width=\textwidth]{../figures/results/figures/rapid_test_statistics/share_infected_among_demand}
  \caption{Share of Individuals that Demand a Rapid Test for a Particular Reason that are Actually Infected}
  \label{fig:share_of_rapid_tests_by_reason_by_infected}
  \floatfoot{\noindent Note that the lower three (household demand, symptomatic without
  PCR demand and other contact demand together form the private demand category. Also
  note that this is not the same as individuals getting a positive rapid test. The
  sensitivity is quite low before individuals become infectious. Therefore, if
  individuals are still in the latent period of their infection they are likely to get a
  false positive rapid test.}
\end{figure}

\FloatBarrier


\subsection{Scenarios}
\label{subsec:appendix_scenarios}

This is the results section.\comment[id=K]{The results can be found in `figures/results`.
This includes both figures and tables for lookup of numbers and summary tables.}


\begin{figure}[ht]
  \centering
  \begin{subfigure}{.6\textwidth}
    \includegraphics[width=0.9 \textwidth]{../figures/results/figures/scenario_comparisons/one_off_and_combined/full_new_known_case_cropped}
  \end{subfigure}%
  \begin{subfigure}{.6\textwidth}
    \includegraphics[width=0.9 \textwidth]{../figures/results/figures/scenario_comparisons/one_off_and_combined/full_newly_infected_cropped}
  \end{subfigure}
  \caption{The Effect of Policies on Observed and Unobserved Cases}
  \label{fig:explain_decline}
  \figurenotes{\ldots}
\end{figure}



\begin{figure}[ht]
  \centering
  \begin{subfigure}{.6\textwidth}
    \includegraphics[width=0.9 \textwidth]{../figures/results/figures/scenario_comparisons/school_scenarios/full_new_known_case}
  \end{subfigure}%
  \begin{subfigure}{.6\textwidth}
    \includegraphics[width=0.9 \textwidth]{../figures/results/figures/scenario_comparisons/school_scenarios/full_newly_infected}
  \end{subfigure}
  \caption{The Effect of Different School Scenarios on Observed and Unobserved Cases}
  \label{fig:school_scenarios_detailed}
\end{figure}


\FloatBarrier

\begin{tabular}{lr}
\toprule
{} &  predicted total infections among 5-14 year olds from Easter until 2021-05-31 \\
scenario                               &                                                                               \\
\midrule
 educ open after easter  without tests &                                             667117 \\
 educ open after easter  with tests    &                                             514931 \\
 close educ after easter               &                                             384124 \\
\bottomrule
\end{tabular}


\FloatBarrier
