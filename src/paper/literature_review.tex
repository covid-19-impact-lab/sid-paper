\subsection{Literature Review}
\label{sec:literature_review}

A commonly used model class in epidemiology are agent-based simulation
models. In a prototypical agent-based simulation model, individuals are
simulated as moving particles. Infections take place when two particles come closer than
a certain contact radius (e.g. \citet{Silva2020} and \citet{Cuevas2020}). While the
simulation approach makes it easy to incorporate heterogeneity in disease progression, it
is hard to incorporate heterogeneity in meeting patterns. Moreover, policies are modeled
as changes in the contact radius or momentum equation of the particles. The translation
from real policies to corresponding model parameters is a hard task.

These shortcomings have motivated variations of agent-based simulation models where
moving particles have been replaced by contact networks for households, work and random
contacts.

The OpenABM-Covid-19 model by \citet{Hinch2020} is the closest in spirit to
ours\comment[id=HM]{I'd rather be close to Aleta et al. ;-)}. They model detailed
contact networks for workplaces, schools and households and can evaluate the effect of
several NPIs. The main focus of their application are contact tracing policies.

\citet{Aleta2020} develop an agent-based simulation model with a very high geographical
resolution by estimating contact networks from fine grained mobility data for the
Boston metropolitan area. They use this model to show how NPIs, contact tracing and PCR
testing can influence the infection dynamics. However, they do not calibrate their model
to match actual infection numbers which makes it more suitable to explore the general
mechanics of different disease mitigation measures than for their quantitative
evaluation.

\citet{Bicher2021} simulate the entire Austrian population. They use data from the first
wave (February 21 to April 9, 2020) to calibrate their model and predict the effect of
different NPIs and contact tracing policies until November 2020. They use the same data
provided by \citet{Mossong2008} as we to calibrate contact networks for households,
workplaces and schools. The model focuses on analyzing the effect of different contact
tracing strategies and not on modelling enacted Austrian policies over a long period
of time.

Moreover, there are several forthcoming applications\comment[id=HM]{What is that
supposed to mean?}\comment[id=J]{Does that mean we should not cite the working
papers here?} of agent-based simulation models with contact networks. Examples are
\citet{Basurto2020}, \citet{DelliGatti2020} and \citet{Mellacher2020}.

Our model combines elements from the above models and adds several others. To the best of
our knowledge, our model is the only one with the following features:

1. The free model parameters have been estimated with the method of simulated moments
   \citep{McFadden1989}. Despite having few free parameters our model does an excellent
   job in explaining observed case numbers and the spread of the B.1.1.7 mutation over
   more than nine months of data.

2. We have an extremely fine grained \replaced[id=K]{representation}{} of schools and
   preschools. We can thus easily model all \replaced[id=K]{schooling policies}{NPIs}
   that have been implemented in Germany in the past nine months. This includes complete
   school closures, phases where only those students whose parents could not find any
   private childcare arrangement could attend, split class approaches for some or all age
   groups and combinations thereof. Moreover, we can account for additional hygiene
   measures whose effect is estimated inside the model.

3. We model the evolution of the pandemic and all enacted policies since the start of the
   second wave. Since the vast majority of cases has occurred in that time period and we
   also model unobserved infections our simulations take into account that many people
   are already immune because they have recovered from an infection and that this
   immunity is not spread randomly across the population.

4. We have an extremely detailed model of PCR and rapid tests with a share of detected
   cases that varies over time and across age groups. We fit estimates of the share of
   known cases from other studies for time periods where those are
   available\comment[id=K]{Citation here?} but also estimate a plausible share of known
   cases after vaccinations become widespread and methodologies that rely on the case
   fatality ratio to identify the share of known cases are not applicable anymore.

 \comment[id=K]{Four is of course plenty but what about weekly recurring contacts or using the google mobility data? -> maybe add these to point 2?}
