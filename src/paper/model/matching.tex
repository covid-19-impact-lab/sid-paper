\subsection{Matching Individuals}
\label{sec:matching}

The empirical data described above only allows to estimate the number of contacts each
person has. In order to simulate transmissions of Covid-19, the numbers of contacts have
to be translated into actual meetings between people. This is achieved by a matching
algorithm:

As described in section \ref{sec:number_of_contacts}, some contact types are recurrent
(i.e. the same people meet regularly), others are non-recurrent (i.e. it would only be
by accident that two people meet twice). The matching process is different for recurrent
and non recurrent contact models.

%% Recurrent contacts ============================================================
Recurrent contacts are described by two components: 1) A set of time invariant groups,
such as school classes or groups of co-workers. Those groups are generated once at the
beginning of the simulation. The groups can be sampled from empirical data or created by
randomly matching simulated individuals into groups. 2) A deterministic or random
function that takes the value 0 (non-participating) and 1 (participating) and can depend
on the weekday, date and health state.\comment[id=K]{Actually the full states and params
enter into the contact functions} This can be used to model things like vacations,
weekends or symptomatic people who stay home (see section \ref{sec:policies} for
details).

Given those two components, the disease transmission for recurrent contacts is extremely
simple: On each simulated day, every person who does not stay home meets all other group
members who do not stay home. If there is a contact between individual $i$ who is
infected with virus variant $v$ and infectious and individual $j$ who is in age group
$a$ and susceptible, then $j$ becomes infected with the following probability

\begin{equation}
    \label{eq:infection_probability_formula}
    P(infection) = \beta_c \cdot s_{c, t} \cdot \sigma_v \cdot \zeta_a
\end{equation}

where $\beta_c$ denotes the base infection probability of contact type $c$,
$s_{c, t}$ is a seasonality factor between zero and one that depends on the contact
type $c$ and time $t$ (see Equation~\ref{eq:seasonality}), $\sigma_v$ is the
infectiousness factor of virus variant $v$ and $\zeta_a$ is an age dependent
susceptibility factor.

The assumption that all group members have contacts with all other group members
is not fully realistic, but a good approximation to reality, especially in
light of the suspected role of aerosol transmission for Covid-19 \citep{Morawska2020,
Anderson2020}. Alternatively, the infection probability of recurrent contact types can
be interpreted as being the product of a true infection probability and the probability
that an actual contact takes place.

%% Non recurrent contacts

The matching of non-recurrent contact types is more difficult because the contact network
is resampled randomly every day. Moreover, it needs to allow for assortative matching. In
our application, all random contacts are assortative with respect to age group $a$ (it is
usually more likely to meet people from the same age group) and county (it is more likely
to meet people from the same county) but in principle any set of discrete variables can
be used. This set of variables that influence matching probabilities
\replaced[id=K]{induce}{introduce} a discrete partition of the population into groups.

Below we first describe one iteration of a simplified matching algorithm that
illustrates what we want to achieve. In practice, we approximate the result of this
matching algorithm by a two stage sampling procedure that is computationally more
efficient. The matching is done for each non-recurrent contact type $c$. The following
step is repeated until no individual has unmatched contacts left. Let $z$ be an iteration
counter for the matching algorithm and $i$ denote the individual whose unmatched
contacts we are trying to match.

Let $K_{z,i,c}$ denote the number of unmatched contacts of individual $i$ of contact type
$c$ before iteration $z$ is completed. Note that $K_{z,i,c} \leq n_{ic}$ which is the
total number of contacts individual $i$ has of type $c$.

Let $a_i$ denote $i$'s age group and $county_i$ her county of residence.

We first draw individual $j$ from the distribution defined by probability mass function
$F_{z}$ over individuals $j \neq i$ in the synthetic population where the probability
$f_{zj}$ is calculated as follows:

\begin{equation}
    \label{eq:combined_matching_probabilities}
    f_{zj} = \underbrace{\alpha_{c,a_i,a_j, county_i, county_j}\vphantom{\frac{bla}{\sum_i^k}}}_{\text{Group Probability}} \cdot \underbrace{\frac{K_{z,j,c}}{\sum_{l=1}^N K_{z,l,c} \cdot \mathbb{I}_{county_l=county_j \land a_l = a_j}}}_{\text{Individual Probability}}
\end{equation}

We then draw an individual $j$. If one of the two participants is susceptible and the
other one is infectious, we sample whether an infection takes place. The success
probability for this event is calculated as in
Equation~\ref{eq:infection_probability_formula}. Finally we update the remaining
numbers of unmatched contacts by setting:

\begin{align}
    K_{z+1,i,c} &= K_{z, i, c} - 1 \\
    K_{z+1, j, c} &= K_{z, j, c} - 1
\end{align}


The runtime of this algorithm scales roughly cubic in the number $N$ of simulated
individuals. This is because the number of iterations is proportional to $N$, in each
iteration we have evaluate Equation~\ref{eq:combined_matching_probabilities} $N$ times
and each evaluation of that equation entails a sum over $N$ individuals.

This makes it prohibitively expensive. We
therefore replace the above algorithm by a two stage sampling procedure, where we
first sample the group from which individual $j$ will be drawn according to the
group probabilities defined in Equation~\ref{eq:combined_matching_probabilities}.
Next we sample an individual from this group with the Individual probabilities defined
in Equation~\ref{eq:combined_matching_probabilities}.

Thus, while the calculation of any given second stage probability entails exactly the
same number of calculations as before we do not have to calculate a second stage
probability for all simulated individuals but only for those who are members of
the group that was sampled in the first stage.

It is easy to see that ex-ante the probability of being sampled are identical between
the two stage sampling and the one stage sampling. The only drawback is
that towards the end of the matching process it becomes possible to sample a group in
which no unmatched contacts are left. In our empirical application this happens
extremely rarely. This is so for two reasons: Firstly, the first stage sampling
probabilities have been estimated from the same dataset as the numbers of contacts so
there cannot be any mismatches such as for example a group that has a low probability
of being sampled in the first stage but where all members have many contacts.
Secondly, the group sizes are relatively large and we go over individuals in random
order. Therefore, groups where no unmatched contacts remain only occur very late in the
matching process.\footnote{If unmatched contacts were a concern one could simply
use the fast two stage sampling process for a first pass over contacts and then
match all remaining contacts with the slow algorithm.}




