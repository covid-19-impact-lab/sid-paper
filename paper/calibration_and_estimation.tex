\section{Calibration and Estimation}
\label{sec:calibration_and_estimation}

The model is described by a large number of parameters that govern the number of contacts
a person has, the likelihood of becoming infected on each contact, the likelihood of
developing light or strong symptoms or even dying from the disease as well as the
duration each stage of the disease takes.

Most of these parameters can be calibrated from existing datasets or the medical
literature. Only the infection probabilities have to be estimated inside the model by
fitting it to time series data of case numbers and fatality rates.


\subsection{Medical Parameters}

This section discusses the medical parameters used in the model, their sources and how
we arrived at the distributions used in the model.\footnotemark

\footnotetext{Additional information can be found in the
\href{https://sid-dev.readthedocs.io/en/latest/reference_guides/epi_params.html}{online
documentation}.}


\subsubsection{Length of Presymptomatic Stage / Incubation Period}


Estimates of the incubation period usually give a range from 2 to 12 days. A meta
analysis by \citet{McAloon2020} comes to the conclusion that ``The incubation period
distribution may be modeled with a lognormal distribution with pooled $\mu$ and $\sigma$
parameters (95\% CIs) of 1.63 (95\% CI 1.51 to 1.75) and 0.50 (95\% CI 0.46 to 0.55),
respectively.'' For simplicity we discretize this distribution into four bins.


\subsubsection{Begin of Infectiousness}

The period between infection and onset of infectiousness is called latent or latency
period. However, the latency period is rarely given in epidemiological reports on
Covid-19. Instead, scientists and agencies usually report the incubation period, the
period from infection to the onset of symptoms. A few studies used measurements of virus
shedding to estimate infectiousness during the course of the disease. When measurements
started before the onset of symptoms the development of the viral load before symptoms
gives us an indication of number of days between the onset of infectiousness and
symptoms.

The European Centre for Disease Prevention and Control estimates that people become
infectious between one and two days before the symptoms set in. This is similar to
\citet{He2020} who estimate this to take 2.3 days and is in line with \citet{Peak2020}.

Given these numbers and the length of the incubation period we can calculate the latency
period for symptomatic people. To our knowledge no estimates for the latency period of
asymptomatic cases of COVID-19 exist. We assume it to be the same for symptomatic and
asymptomatic cases.

Thus, we arrive at the following distribution for latency periods: 40\% have one day.
35\% have two days. 20\% have three days and 5\% have 5 days.


\subsubsection{Duration of Infectiousness}

We assume that the duration of infectiousness is the same for both symptomatic and
asymptomatic individuals as evidence suggests little differences in the transmission
rates of SARS-CoV-2 virus between symptomatic and asymptomatic patients
(\citet{Yin2020}) and that the viral load between symptomatic and asymptomatic
individuals are similar (\citet{Zou2020}, \citet{Byrne2020}, \citet{Singanayagam2020}).

Our distribution of the duration of infectiousness is based on \citet{Byrne2020}.

For symptomatic cases they arrive at 0-5 days before symptom onset (figure 2) and 3-8
days of infectiousness afterwards.\footnotemark Thus, we arrive at 0 to 13 days as the
range for infectiousness among individuals who become symptomatic (see also figure 5).
This duration range is very much in line with the meta-analysis’ reported evidence for
asymptomatic individuals (see their figure 1). Thus, we arrive at 0 to 13 days as the
range for infectiousness among individuals who become symptomatic. This duration range
is very much in line with the meta-analysis' reported evidence for asymptomatic
individuals.

\footnotetext{
    Viral loads may be detected much later but 8 days seems to be the time after which
    most people are culture negative, as also reported by \citet{Singanayagam2020}.
}


Following this evidence we assume the following discretized distribution of the
infectiousness period: 10\% of individuals are infectious for three days, 25\% for five
days, another 25\% for seven days, 20\% for nine days and 20\% for eleven days.


\subsubsection{Duration of Symptoms}

We use the duration to recovery of mild and moderate cases reported by \cite[Figure~S3,
Panel~2]{Bi2020} for the duration of symptoms for asymptomatic and non-ICU requiring
symptomatic cases.

We collapse the data to the following distribution: 10\% recover after 15 days and 30\%
require 18, 22 or 27 days respectively.

These numbers are only used for mild cases. We do not disaggregate by age. Note that the
length of symptoms is not very important in our model given that individuals stop being
infectious before their symptoms cease.


\subsubsection{Time from Symptom Onset to Admission to ICU}

The data on how many percent of symptomatic patients will require ICU is pretty thin. We
rely on data by the US CDC (\citet{Stokes2020}) and
\href{https://github.com/BDI-pathogens/OpenABM-Covid19/blob/572e24ca2dbf7153789a92ad3a27e4c515d0e576/documentation/parameters/parameter_dictionary.md}{the
OpenABM-Project}. Table~\ref{tab:symptomatic-to-ICU} shows our derivations for the
probabilities of requiring intensive care per age group.

\begin{table}[tb]
    \caption{Shares of symptomatic patients who will require ICU care by age groups.}
    \label{tab:symptomatic-to-ICU}
    \centering

    \begin{tabular}{ll}
        \toprule
        Age Group & Share \\
        \midrule
        0-9 & 0.00005 \\
        10-19 & 0.00030 \\
        20-29 & 0.00075 \\
        30-39 & 0.00345 \\
        40-49 & 0.01380 \\
        50-59 & 0.03404 \\
        60-69 & 0.10138 \\
        70-79 & 0.16891 \\
        80-100 & 0.26871 \\
        \bottomrule
    \end{tabular}

    \tablenotes{
        The data is taken from \citet{Stokes2020} and
        \href{https://github.com/BDI-pathogens/OpenABM-Covid19/blob/572e24ca2dbf7153789a92ad3a27e4c515d0e576/documentation/parameters/parameter_dictionary.md}{the
        OpenABM-Project}. }

\end{table}

For those who will require intensive care we follow \citet{Chen2020} who estimate the
time from symptom onset to ICU admission as 8.5 $\pm$ 4 days.

This aligns well with numbers reported for the time from first symptoms to
hospitalization: \citet{Gaythorpe2020} report a mean of 5.76 with a standard deviation
of 4. This is also in line with the durations collected by
\href{https://www.rki.de/DE/Content/InfAZ/N/Neuartiges_Coronavirus/Steckbrief.html#doc13776792bodyText16}{the
Robert-Koch-Institut}.

We assume that the time between symptom onset and ICU takes 4, 6, 8 or 10 days with
equal probabilities. These times mostly matter for the ICU capacities.


\subsubsection{Death and Recovery from ICU}

We take the survival probabilities and time to death and time until recovery from
intensive care from the \href{https://tinyurl.com/y5owhyts}{OpenABM Project}.

They report time until death to have a mean of 11.74 days and a standard deviation of
8.79 days. Approximating this with the normal distribution, we have nearly 10\%
probability mass below 0. We use it nevertheless as several other distributions (such as
chi squared and uniform) were unable to match the variance. Discretizing the
distribution leads to 41\% of individuals who will die from Covid-19 after one day in
intensive care, 22\% day after 12 days, 29\% after 20 days and 7\% after 32 days. Again,
we rescale this for every age group among those that will not survive.

They report a mean duration of 18.8 days until recovery and a standard deviation of
12.21 days. Approximating this with the normal distribution, we have over 5\%
probability mass below 0. Of those who recover in intensive care, 22\% do so
after one day, 30\% after 15 days, 28\% after 25 days and 18\% after 45 days.



\subsection{Number of Contacts}
\label{sub:number_of_contacts}

We calibrate the parameters for the predicted numbers of contacts from contact diaries of
over 2000 individuals from Germany, Belgium, the Netherlands and Luxembourg
\citep{Mossong2008}. Each contact diary contains all contacts an individual had
throughout one day, including information on the other person (such as age and gender)
and information on the contact. Importantly, for each contact individuals entered of
which type the contact (school, leisure, work etc.) was and how frequent the contact with
the other person is.

Thus, we can use the empirical distributions from this data as pre-pandemic number of
contacts.


\FloatBarrier


\subsection{Assortative Matching}

As mentioned in section \ref{sec:matching}, the probability that two individuals are
matched can depend on background characteristics. In particular, we allow this
probability to depend on age and county of residence. While we do not have good data on
geographical assortativity and just roughly calibrate it such that 60 \% of contacts are
within the same county, we can calibrate the assortative mixing by age from the same data
we use to calibrate the number of contacts.

\begin{figure}[ht]
    \centering
    \includegraphics[width=0.9 \textwidth]{../figures/assortative_matching_probability_example.png}
    \caption{Distribution of random non-work contacts by age of participants.}
    \label{fig:assortativity}
    \figurenotes{
        The figure shows the distribution of random non-work contacts by age groups. A
        row shows the share of contacts a certain age group has with all other age
        groups. Higher values are colored in darker red tones. The diagonal represents
        the share of contacts with individuals from the same age group. }
\end{figure}

Figure~\ref{fig:assortativity} shows that assortativity by age is especially strong for
children and younger adults. For older people, the pattern becomes more dispersed around
their own age group, but within-age-group contacts are still the most common contacts.

\FloatBarrier


\subsection{Infection Probabilities}
\label{sec:estimation}

To calibrate infection probabilities outside of the model, it would be important to know
the exact duration and distance of each contact type as well as viral loads. Since this
is not available in any dataset, we estimate those parameters inside the model with the
method of simulated moments \citep{McFadden1989} by minimizing the distance between
simulated and observed infection rates. Since our model includes a lot of randomness, we
average simulated infection rates over several model runs.

Currently, we use data for Germany from August until November. We do not use earlier
periods to save computational time. Moreover, we would be worried that the there are
seasonal effects that we currently do not model.

To avoid overfitting and simplify the numerical optimization problem, we only allow for
four different probabilities: 1) for contacts in schools, preschools and nurseries. 2)
for work contacts. 3) for households. 4) for leisure activities.

\subsection{Policies}

\FloatBarrier

In our empirical application we distinguish four groups of contact types: households,
education, work and other contacts. For households we assume that the individuals'
contacts in their households do not change over our estimation period. For nurseries,
preschools and schools we implement vacations as announced by the German federal states
as well as school closures. For the moment we ignore both emergency childcare and that
lack of childcare leads working parents to stay home.
%
% Schließung von Kindertagesstätten und Schulen: 37,4 Millionen ausgefallene Arbeitstage
% http://www.iab-forum.de/schul-und-kitaschliessungen-krankheit-quarantaene-die-coronabedingten-arbeitsausfaelle-der-erwerbstaetigen-steigen-auf-592-millionen-arbeitstage/
%
%
% https://www.sueddeutsche.de/politik/schulschliessung-lockdown-bildung-1.5190377:
% In allen Ländern geht trotz des Lockdowns ein erheblicher Anteil der Schülerinnen und
% Schüler in die Schule.
% https://gfx.sueddeutsche.de/apps/e525337/www/_image_desktopw1840q70-1e2e2bf78b7d4430.png
% 18% der Grundschüler in Notbetreuung in BW
%
For our work models\footnotemark we use the reductions in work mobility reported in the
Google Mobility Data \citep{Google2021} to calibrate our work policies.
Reductions in work contacts are not random but governed through a work contact priority
where the policy changes the threshold below which workers stay home.
Figure \ref{fig:work_multiplier} shows the share of workers that go to work in our model
over time.

\begin{figure}[ht]
    \centering
    \includegraphics[width=1.1 \textwidth]{../figures/results/figures/data/work_multiplier_since_sep}
    \caption{Share of Workers with Work Contacts}
    \label{fig:work_multiplier}
    \figurenotes{The figure shows the work mobility as reported by \cite{Google2021}. We take this as a proxy of the share of workers who are not in home office, i.e. who still have physical work contacts.}
\end{figure}

\footnotetext{We distinguish non-recurrent work contacts, daily work contacts and weekly work contacts.}

For the last group of contacts which cover things like leisure activities, grocery
shopping etc. we have no reliable data by how much policies reduce them.
In addition, they are likely to be affected by social and psychological factors such as
pandemic fatigue and vacations. Because of this we estimate them like the infection
probabilities to fit the time series data. We use very few change points and tie them to
particular events such as policy announcements or particular holidays.

\FloatBarrier

\subsection{Rapid Test Demand}

In our model, there are five reasons why rapid tests are done:
\begin{enumerate}
    \item someone plans to have work contacts
    \item someone is an employees of an educational facility or school pupil
    \item someone lives in a household where someone has tested positive or developed symptoms
    \item someone has developed symptoms but has not received a PCR test
    \item someone plans to participate in a weekly non-work meeting
\end{enumerate}

% work rapid tests

For work contacts, we know from the COSMO study (\cite{Betsch2021}, 20th/21st of April) that 60\% of workers who receive a test offer by their employer regularly use it. We assume this to be time constant.

In addition, there are some surveys that allow us to trace the expansion of employers who
offer tests to their employees. Mid march, 20\% of employers offered tests to their
employees \citep{DIHK2021}. In the second half of March, 23\% of employees reported being
offered weekly rapid tests by their employer \citep{Ahlers2021}. This share increased to
60\% until the first days of April \cite{ZDF2021}. Until mid April 70\% of workers were
expected to receive a weekly test offer \citep{AerzteZeitung2021}. However, according to
surveys conducted in mid April \citep{Betsch2021}, less than two thirds of individuals
with work contacts receive a test offer. Starting on April 19th employers were required by law to provide two weekly tests to their employees \citep{Bundesanzeiger2021}.
We assume that compliance is incomplete and only 80\% of employers actually offer tests.

% educ rapid tests

\textcolor{red}{Sources still missing below this}

We assume that employees in educational facilities start getting tested in 2021 and that
by March 1st 30\% of them are tested weekly. The share increases to 90\% for the week before Easter. At that time both Bavaria and Baden-Württemberg were offering tests to teachers and North-Rhine Westphalia and Lower Saxony were already testing students and tests for students and teachers were already mandatory in Saxony. After Easter we assume that 95\% of teachers get tested twice per week.

Tests for students started later so we assume that they only start in February and only
10\% of students get tested by March 1st. Relying on the same sources as above we
approximate that by the week before Easter this share had increased to 40\%.

After Easter we assume the share of students receiving twice weekly tests to
based on tests already being mandatory in North-Rhine Westphalia and Bavaria while still
being voluntary in Baden-Württemberg. There tests become mandatory on April 19th.


% household member rapid tests



% own symptom rapid test demand



% private meeting test demand



\textcolor{red}{
    Add a section on how we calibrate rapid test demand; Mainly describe the
    datapoints we have and say that we usually interpolate linearly in between data
    points. (Only exception to that is private rapid test demand, which we fit to data)
}


\begin{figure}[ht]
    \centering
    \caption{Share of Individuals With Rapid Tests}
    \label{fig:share_ever_rapid_test}
    \begin{subfigure}{.6\textwidth}
      \includegraphics[width=0.9 \textwidth]{../figures/results/figures/scenario_comparisons/spring_fit/full_share_ever_rapid_test}
    \end{subfigure}%
    \begin{subfigure}{.6\textwidth}
      \includegraphics[width=0.9 \textwidth]{../figures/results/figures/scenario_comparisons/spring_fit/full_share_rapid_test_in_last_week}
    \end{subfigure}
    \caption{Share of Individuals Having Done a Rapid Test in the Last Week}
    \label{fig:share_rapid_test_last_week}
    \figurenotes{The figure compares the share of individuals who have ever done a rapid test or done a rapid test within the last week in our simulations to the shares reported in the \href{https://projekte.uni-erfurt.de/cosmo2020/web/topic/wissen-verhalten/80-schnelltests/}{COVID-19 Snapshot Monitoring survey}. The
    left panel compares the share of individuals who have ever done a rapid test. The right panel compares the share of individuals who have done a rapid test within the last seven days in our simulation compared to the share reporting to have done at least weekly rapid tests in the last four weeks in the COSMO survey. Overall our calibration of rapid tests are slightly conservative. The overall share is below that in the study. We fit the share of weekly tests quite exactly. However, the study only covers adults while our share also includes children who are tested very regularly when attending school.
    }
\end{figure}
