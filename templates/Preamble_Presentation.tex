% !TeX program = pdflatex
% !TeX TXS-program:compile = txs:///pdflatex/
% !TeX TS-program = pdflatex
% !BIB program = biber
% !TeX TXS-program:bibliography = txs:///biber




%%%%%%%%%%%%%%%%%%%%%%%%%%%%
%%  FUNDAMENTAL SETTINGS  %%
%%%%%%%%%%%%%%%%%%%%%%%%%%%%


\newcommand{\serifbodyfont}{0}
% 0: Body is set in sans-serif font
% 1: Body is set in serif font
\newcommand{\serifheadingfont}{0}
% 0: Headings (``structure'') are set in sans-serif font
% 1: Headings (``structure'') are set in serif font

\newcommand{\showsectionnavigation}{0}
% 0: ``Headline'' is disabled.
% 1: ``Headline'' (at the very top of slides) is included and displays all sections of the presentation,
%      with the current section being highlighted.




%%%%%%%%%%%%%%%%%%%%%%%%%%%%
%%  FUNDAMENTAL PACKAGES  %%
%%%%%%%%%%%%%%%%%%%%%%%%%%%%


\usepackage{ifthen}

\usepackage{calc}

\usepackage{tcolorbox}

\usepackage[LY1, LGR, TS1, T1]{fontenc}
% LGR (Greek) is needed to enable the ``sansserif math'' features.
\usepackage[utf8]{inputenc}

\usepackage{tabularx}

\ifxetex
	\usepackage[protrusion=true, expansion=false]{microtype}
\else
	\usepackage[protrusion=true, expansion=false, kerning=true]{microtype}
\fi
% The microtype package enables so-called hanging punctuation. That is, when a punctuation sign like ":", ".", "-", etc. is found at the beginning or end of a line, it is protruded a little into the page margin. This results in "optical margin alignment," because the protrusion makes the margin alignment look straighter.


\definecolor{darkgray}{rgb}{0.4,0.4,0.4}

\usepackage[absolute, overlay]{textpos}
\usepackage{graphicx}
\usepackage{amsmath}
%\usepackage{amsfonts}
%\usepackage{amssymb}
\usepackage{epstopdf}
\DeclareGraphicsRule{.tif}{png}{.png}{`convert #1 `dirname #1`/`basename #1 .tif`.png}

\usepackage[ngerman, american, USenglish]{babel}
% German and US English hyphenation and quotation marks
\selectlanguage{USenglish}
\usepackage[autostyle=true]{csquotes}

\usepackage{multirow}

\usepackage{xcolor}
\usepackage{pgf}%,pgfarrows,pgfnodes,pgfshade}
\usepackage{tikz}
\usepackage{pgfplots}
\usetikzlibrary{mindmap, trees, patterns}

\usepackage{xspace}

% \usepackage{eurosym}

%\usepackage{etoolbox}
%	% Enables manipulating LaTeX commmands via \preto, \appto, \patchcmd, etc.
%	% loaded automatically by beamer
\usepackage{xpatch}
% Enables manipulating LaTeX commmands via \xpatchcmd etc.

\hypersetup{%
	pdfpagemode=FullScreen,
	breaklinks=true,
	colorlinks, linkcolor=, urlcolor=SpotColor, citecolor=%
}
\urlstyle{same}
\newcommand{\email}[1]{\href{mailto:#1}{\nolinkurl{#1}}}

\usepackage{import}	 % To allow for relative paths in nested \input's (\import's)




%%%%%%%%%%%%%%%%%%%%%
%%  FONT SETTINGS  %%
%%%%%%%%%%%%%%%%%%%%%


\subimport{0_0_Preamble/}{Preamble_Fonts_Charter_FiraSans}

\ifnum \serifbodyfont=0
	\setbeamerfont{normal text}{family=\sffamily}
	\renewcommand{\familydefault}{\savesffamily}
	\renewcommand{\mddefault}{\savesfmdseries}
	\renewcommand{\bfdefault}{\savesfbfseries}
\else
	\usefonttheme{serif}
	\setbeamerfont{normal text}{family=\rmfamily}
\fi
\ifnum \serifheadingfont=0
	\setbeamerfont{structure}{family=\sffamily, shape=\upshape, series=\bfseries}
\else
	\setbeamerfont{structure}{family=\rmfamily, shape=\upshape, series=\bfseries}
\fi

% Make frame titles and headings bold
\usefonttheme{structurebold}
\usefonttheme{professionalfonts}

\setbeamerfont{footline}
	{parent=structure, size=\tiny, series=\mdseries}
\setbeamercolor{footline}
	{fg=SpotColor}

%% Use the bm (= boldmath) package for better support of setting math in bold ==>
%% Prevent the "Too many math fonts used" error:
\newcommand{\bmmax}{0}
\newcommand{\hmmax}{0}
\usepackage{bm}
%% <==

\usepackage{fontawesome}

\usepackage{mathtools}
%\mathtoolsset{centercolon}
% This makes the compilation fail in combination with tikz. See
% https://tex.stackexchange.com/questions/89467/why-does-pdftex-hang-on-this-file.
%% Inspired by https://tex.stackexchange.com/questions/251460/how-to-put-symbols-of-equal-size-on-top-of-each-other
\newcommand{\succeqq}{%
	\mathrel{%
		\vcenter{\offinterlineskip
			\ialign{##\cr$\succ$\cr\noalign{\kern 1pt}$=$\cr}%
		}%
	}%
}
\newcommand{\nsucceqq}{\mathrel{\not\succeqq}}
\newcommand*{\coloneqq}{\mathrel{%
		\mathrel{%
			\raisebox{0.15ex}{\scalebox{0.85}{\ensuremath{:}}}\hspace{-0.2pt}%
		}%
		=%
}}

% !TeX program = pdflatex
% !TeX TXS-program:compile = txs:///pdflatex/
% !TeX TS-program = pdflatex
% !BIB program = biber
% !TeX TXS-program:bibliography = txs:///biber




%%%%%%%%%%%%%%%%%%%%%%%%%%%%%%%%%%%%%%%%%%%%%%%%%
%%  SANS-SERIF MATH IN SANS-SERIF ENVIRONMENT  %%
%%%%%%%%%%%%%%%%%%%%%%%%%%%%%%%%%%%%%%%%%%%%%%%%%


% See https://tex.stackexchange.com/questions/41497/how-to-typeset-some-text-including-math-content-in-sans-serif
% See https://tex.stackexchange.com/questions/33165/make-mathfont-respect-the-surrounding-family

% Necessary for use of kpfonts
% ==>
\makeatletter
\newif\ifkp@upRm
\newif\ifkp@osm
\newif\ifkp@vosm
\makeatother
% <==

\DeclareMathVersion{normalup}
\DeclareMathVersion{boldup}
\DeclareMathVersion{sans}

%\SetSymbolFont{operators}{sans}{OT1}{jkpss}{m}{n}
%	% From http://mirrors.ctan.org/fonts/kpfonts/latex/kpfonts.sty
\SetSymbolFont{operators}   {sans}{OT1}{mdbch}{m}{n}
\SetSymbolFont{letters}     {sans}{OML}{jkpss}{m}{it}
	% From http://mirrors.ctan.org/fonts/kpfonts/latex/kpfonts.sty
%\SetSymbolFont{letters}     {sans}{OML}{cmbrm}{m}{it}
%\SetSymbolFont{symbols}     {sans}{OMS}{cmbrs}{m}{n}
\SetSymbolFont{symbols}     {sans}{OMS}{jkp}  {m}{n}
	% From http://mirrors.ctan.org/fonts/kpfonts/latex/kpfonts.sty
\DeclareSymbolFont{extrasymbols}  {OMS}{cmbrs}{m}{n}
\SetSymbolFont{extrasymbols}{sans}{OMS}{cmbrs}{m}{n}
	% Some symbols (e.g., \prime) look weird in kpfonts.
	% This provides the option to replace them by symbols from mathdesign-charter.

\SetMathAlphabet{\mathit} {sans}{T1}{\savesffamily}{\savesfmdseries}{it}
\SetMathAlphabet{\mathbf} {sans}{T1}{\savesffamily}{\savesfbfseries}{n}
\SetMathAlphabet{\mathtt} {sans}{OT1}{cmtl}{m}{n}
\SetMathAlphabet{\mathcal}{sans}{OMS}{ntxsy}{m}{n}
	% See https://tex.stackexchange.com/questions/231583/import-mathcal-symbols-from-txfonts
%\SetSymbolFont{largesymbols}{sans}{OMX}{jkpss}{m}{n}
%	% From http://mirrors.ctan.org/fonts/kpfonts/latex/kpfonts.sty
\SetSymbolFont{largesymbols} {sans}{OMX}{mdbch}{m}{n}
	% Using symbols like \int, \left(, etc. from mathdesign-charter because they look better than the ones included in kpfonts

\DeclareMathVersion{sansup}
\SetSymbolFont{letters}  {sansup}{OML}{jkpss}{m}{it}
\SetSymbolFont{symbols}  {sansup}{OMS}{jkp}  {m}{n}

\DeclareMathVersion{boldsans}
%\SetSymbolFont{operators}{boldsans}{OT1}{jkpss}{b}{n}
%	% From http://mirrors.ctan.org/fonts/kpfonts/latex/kpfonts.sty
\SetSymbolFont{operators}{boldsans}{OT1}{mdbch}{bx}{n}
\SetSymbolFont{letters}  {boldsans}{OML}{jkpss}{bx}{it}
	% From http://mirrors.ctan.org/fonts/kpfonts/latex/kpfonts.sty
%\SetSymbolFont{letters}  {boldsans}{OML}{mdbch}{bx}{it}
%\SetSymbolFont{letters}{boldsans}{OML}{cmbrm}{b}{it}
\SetSymbolFont{symbols}  {boldsans}{OMS}{jkp}  {bx}{n}
	% From http://mirrors.ctan.org/fonts/kpfonts/latex/kpfonts.sty
%\SetMathAlphabet{\mathrm}{boldsans}{OT1}{\savesffamily}{\savesfbfseries}{n}
\SetMathAlphabet{\mathit} {boldsans}{T1}{\savesffamily}{\savesfbfseries}{it}
\SetMathAlphabet{\mathtt} {boldsans}{T1}{cmtl}{b}{n}
\SetMathAlphabet{\mathcal}{boldsans}{OMS}{ntxsy}{b}{n}
%\SetSymbolFont{largesymbols}{boldsans}{OMX}{jkpss}{bx}{n}
%	% From http://mirrors.ctan.org/fonts/kpfonts/latex/kpfonts.sty
\SetSymbolFont{largesymbols}{boldsans}{OMX}{mdbch}{bx}{n}
	% Using symbols like \int, \left(, etc. from mathdesign-charter because they look better than the ones included in kpfonts

\DeclareMathVersion{boldsansup}
\SetSymbolFont{letters}{boldsansup}{OML}{jkpss}{bx}{it}
\SetSymbolFont{symbols}{boldsansup}{OMS}{jkp}  {bx}{n}

% Using glyphs for math mode from the custom sansserif font
\DeclareSymbolFont{uprightglyphs}{T1}{\savermfamily}{\savermmdseries}{n}
\SetSymbolFont{uprightglyphs}{normal}    {T1}{\savermfamily}{\savermmdseries}{n}
\SetSymbolFont{uprightglyphs}{normalup}  {T1}{\savermfamily}{\savermmdseries}{n}
\SetSymbolFont{uprightglyphs}{bold}      {T1}{\savermfamily}{\savermbfseries}{n}
\SetSymbolFont{uprightglyphs}{boldup}    {T1}{\savermfamily}{\savermbfseries}{n}
\SetSymbolFont{uprightglyphs}{sans}      {T1}{\savesffamily}{\savesfmdseries}{n}
\SetSymbolFont{uprightglyphs}{sansup}    {T1}{\savesffamily}{\savesfmdseries}{n}
\SetSymbolFont{uprightglyphs}{boldsans}  {T1}{\savesffamily}{\savesfbfseries}{n}
\SetSymbolFont{uprightglyphs}{boldsansup}{T1}{\savesffamily}{\savesfbfseries}{n}
\DeclareSymbolFont{italicglyphs} {T1}{\savermfamily}{\savermmdseries}{it}
\SetSymbolFont{italicglyphs} {normal}    {T1}{\savermfamily}{\savermmdseries}{it}
\SetSymbolFont{italicglyphs} {normalup}  {T1}{\savermfamily}{\savermmdseries}{n}
\SetSymbolFont{italicglyphs} {bold}      {T1}{\savermfamily}{\savermbfseries}{it}
\SetSymbolFont{italicglyphs} {boldup}    {T1}{\savermfamily}{\savermbfseries}{n}
\SetSymbolFont{italicglyphs} {sans}      {T1}{\savesffamily}{\savesfmdseries}{it}
\SetSymbolFont{italicglyphs} {sansup}    {T1}{\savesffamily}{\savesfmdseries}{n}
\SetSymbolFont{italicglyphs} {boldsans}  {T1}{\savesffamily}{\savesfbfseries}{it}
\SetSymbolFont{italicglyphs} {boldsansup}{T1}{\savesffamily}{\savesfbfseries}{n}

% Syntax of \DeclareMathSymobl:
% \DeclareMathSymbol {<symbol>} {<type>} {<sym-font>} {<slot>}
% Type              Meaning	            Example
% 0 or \mathord     Ordinary             $\alpha$
% 1 or \mathop      Large operator       $\sum$
% 2 or \mathbin     Binary operation     $\times$
% 3 or \mathrel     Relation             $\leq$
% 4 or \mathopen    Opening              $\langle$
% 5 or \mathclose   Closing              $\rangle$
% 6 or \mathpunct   Punctuation          ;
% 7 or \mathalpha   Alphabet character   A
% Example declaration:
% \DeclareMathSymbol{b}{0}{letters}{`b}

% Digits
\DeclareMathSymbol{0}{\mathalpha}{uprightglyphs}{`0}
\DeclareMathSymbol{1}{\mathalpha}{uprightglyphs}{`1}
\DeclareMathSymbol{2}{\mathalpha}{uprightglyphs}{`2}
\DeclareMathSymbol{3}{\mathalpha}{uprightglyphs}{`3}
\DeclareMathSymbol{4}{\mathalpha}{uprightglyphs}{`4}
\DeclareMathSymbol{5}{\mathalpha}{uprightglyphs}{`5}
\DeclareMathSymbol{6}{\mathalpha}{uprightglyphs}{`6}
\DeclareMathSymbol{7}{\mathalpha}{uprightglyphs}{`7}
\DeclareMathSymbol{8}{\mathalpha}{uprightglyphs}{`8}
\DeclareMathSymbol{9}{\mathalpha}{uprightglyphs}{`9}
% Operators and punctuation
\DeclareMathSymbol{+}{\mathbin}  {operators}    {`+}
	% Not from uprightglyphs due to bad spacing
\DeclareMathSymbol{=}{\mathrel}  {operators}    {`=}
	% Not from uprightglyphs due to bad spacing
\DeclareMathSymbol{.}{\mathord}  {uprightglyphs}{`.}
\DeclareMathSymbol{,}{\mathpunct}{uprightglyphs}{`,}
\DeclareMathSymbol{;}{\mathpunct}{uprightglyphs}{`;}
\DeclareMathSymbol{/}{\mathord}  {uprightglyphs}{`/}
%\DeclareMathSymbol{/}{\mathop}   {uprightglyphs}{`/}
%	% This would icrease the spacing around the division slash slightly
%\DeclareMathSymbol{(}{\mathopen} {uprightglyphs}{`(}
%\DeclareMathSymbol{)}{\mathclose}{uprightglyphs}{`)}
%\DeclareMathSymbol{[}{\mathopen} {uprightglyphs}{`[}
%\DeclareMathSymbol{]}{\mathclose}{uprightglyphs}{`]}
\DeclareMathSymbol{\prime}{\mathord}{extrasymbols}{"30}
	% Use \prime from mathdesign-charter because it looks better than the one in kpfonts
\DeclareMathDelimiter{(}      {\mathopen} {uprightglyphs}{`(} {largesymbols}{"00}
\DeclareMathDelimiter{)}      {\mathclose}{uprightglyphs}{`)} {largesymbols}{"01}
\DeclareMathDelimiter{[}      {\mathopen} {uprightglyphs}{`[} {largesymbols}{"02}
\DeclareMathDelimiter{]}      {\mathclose}{uprightglyphs}{`]} {largesymbols}{"03}
\DeclareMathDelimiter{\lbrace}{\mathopen} {uprightglyphs}{`\{}{largesymbols}{"08}
\DeclareMathDelimiter{\rbrace}{\mathclose}{uprightglyphs}{`\}}{largesymbols}{"09}
% Uppercase Latin characters
\DeclareMathSymbol{A}{\mathalpha}{italicglyphs}{`A}
\DeclareMathSymbol{B}{\mathalpha}{italicglyphs}{`B}
\DeclareMathSymbol{C}{\mathalpha}{italicglyphs}{`C}
\DeclareMathSymbol{D}{\mathalpha}{italicglyphs}{`D}
\DeclareMathSymbol{E}{\mathalpha}{italicglyphs}{`E}
\DeclareMathSymbol{F}{\mathalpha}{italicglyphs}{`F}
\DeclareMathSymbol{G}{\mathalpha}{italicglyphs}{`G}
\DeclareMathSymbol{H}{\mathalpha}{italicglyphs}{`H}
\DeclareMathSymbol{I}{\mathalpha}{italicglyphs}{`I}
\DeclareMathSymbol{J}{\mathalpha}{italicglyphs}{`J}
\DeclareMathSymbol{K}{\mathalpha}{italicglyphs}{`K}
\DeclareMathSymbol{L}{\mathalpha}{italicglyphs}{`L}
\DeclareMathSymbol{M}{\mathalpha}{italicglyphs}{`M}
\DeclareMathSymbol{N}{\mathalpha}{italicglyphs}{`N}
\DeclareMathSymbol{O}{\mathalpha}{italicglyphs}{`O}
\DeclareMathSymbol{P}{\mathalpha}{italicglyphs}{`P}
\DeclareMathSymbol{Q}{\mathalpha}{italicglyphs}{`Q}
\DeclareMathSymbol{R}{\mathalpha}{italicglyphs}{`R}
\DeclareMathSymbol{S}{\mathalpha}{italicglyphs}{`S}
\DeclareMathSymbol{T}{\mathalpha}{italicglyphs}{`T}
\DeclareMathSymbol{U}{\mathalpha}{italicglyphs}{`U}
\DeclareMathSymbol{V}{\mathalpha}{italicglyphs}{`V}
\DeclareMathSymbol{W}{\mathalpha}{italicglyphs}{`W}
\DeclareMathSymbol{X}{\mathalpha}{italicglyphs}{`X}
\DeclareMathSymbol{Y}{\mathalpha}{italicglyphs}{`Y}
\DeclareMathSymbol{Z}{\mathalpha}{italicglyphs}{`Z}
% lowercase Latin characters
\DeclareMathSymbol{a}{\mathalpha}{italicglyphs}{`a}
\DeclareMathSymbol{b}{\mathalpha}{italicglyphs}{`b}
\DeclareMathSymbol{c}{\mathalpha}{italicglyphs}{`c}
\DeclareMathSymbol{d}{\mathalpha}{italicglyphs}{`d}
\DeclareMathSymbol{e}{\mathalpha}{italicglyphs}{`e}
\DeclareMathSymbol{f}{\mathalpha}{italicglyphs}{`f}
\DeclareMathSymbol{g}{\mathalpha}{italicglyphs}{`g}
\DeclareMathSymbol{h}{\mathalpha}{italicglyphs}{`h}
\DeclareMathSymbol{i}{\mathalpha}{italicglyphs}{`i}
\DeclareMathSymbol{\imath}{\mathalpha}{italicglyphs}{"19}
\DeclareMathSymbol{j}{\mathalpha}{italicglyphs}{`j}
\DeclareMathSymbol{\jmath}{\mathalpha}{italicglyphs}{"1A}
\DeclareMathSymbol{k}{\mathalpha}{italicglyphs}{`k}
\DeclareMathSymbol{l}{\mathalpha}{italicglyphs}{`l}
\DeclareMathSymbol{m}{\mathalpha}{italicglyphs}{`m}
\DeclareMathSymbol{n}{\mathalpha}{italicglyphs}{`n}
\DeclareMathSymbol{o}{\mathalpha}{italicglyphs}{`o}
\DeclareMathSymbol{p}{\mathalpha}{italicglyphs}{`p}
\DeclareMathSymbol{q}{\mathalpha}{italicglyphs}{`q}
\DeclareMathSymbol{r}{\mathalpha}{italicglyphs}{`r}
\DeclareMathSymbol{s}{\mathalpha}{italicglyphs}{`s}
\DeclareMathSymbol{t}{\mathalpha}{italicglyphs}{`t}
\DeclareMathSymbol{u}{\mathalpha}{italicglyphs}{`u}
\DeclareMathSymbol{v}{\mathalpha}{italicglyphs}{`v}
\DeclareMathSymbol{w}{\mathalpha}{italicglyphs}{`w}
\DeclareMathSymbol{x}{\mathalpha}{italicglyphs}{`x}
\DeclareMathSymbol{y}{\mathalpha}{italicglyphs}{`y}
\DeclareMathSymbol{z}{\mathalpha}{italicglyphs}{`z}

%% Sansserif Greek letters
%\DeclareSymbolFont{lgrgreek}{LGR}{\savesffamily}{\savesfmdseries}{it}
%\SetSymbolFont{lgrgreek}{sans}    {LGR}{\savesffamily}{\savesfmdseries}{it}
%\SetSymbolFont{lgrgreek}{boldsans}{LGR}{\savesffamily}{\savesfbfseries}{it}

% The following is taken from
% https://tex.stackexchange.com/questions/116389/automatic-upright-math-when-text-is-in-italic/116399#116399
% Filling in ``missing'' Greek glyphs for completeness
% (not really necessary, since they look identical to Latin glyphs and are thus almost never used)
% ==>
\newcommand{\omicron}{o}
\newcommand{\Digamma}{F}
\newcommand{\Alpha}  {A}
\newcommand{\Beta}   {B}
\newcommand{\Epsilon}{E}
\newcommand{\Zeta}   {Z}
\newcommand{\Eta}    {H}
\newcommand{\Iota}   {I}
\newcommand{\Kappa}  {K}
\newcommand{\Mu}     {M}
\newcommand{\Nu}     {N}
\newcommand{\Omicron}{O}
\newcommand{\Rho}    {P}
\newcommand{\Tau}    {T}
\newcommand{\Chi}    {X}
\newcommand{\omicronup}{\mathup{o}}
\newcommand{\Digammaup}{\mathup{F}}
\newcommand{\Alphaup}  {\mathup{A}}
\newcommand{\Betaup}   {\mathup{B}}
\newcommand{\Epsilonup}{\mathup{E}}
\newcommand{\Zetaup}   {\mathup{Z}}
\newcommand{\Etaup}    {\mathup{H}}
\newcommand{\Iotaup}   {\mathup{I}}
\newcommand{\Kappaup}  {\mathup{K}}
\newcommand{\Muup}     {\mathup{M}}
\newcommand{\Nuup}     {\mathup{N}}
\newcommand{\Omicronup}{\mathup{O}}
\newcommand{\Rhoup}    {\mathup{P}}
\newcommand{\Tauup}    {\mathup{T}}
\newcommand{\Chiup}    {\mathup{X}}
% <==

% Save original definitions of the Greek letters
% ==>
\makeatletter
\@for\@tempa:=%
	alpha,beta,gamma,delta,epsilon,zeta,eta,theta,iota,kappa,lambda,mu,nu,xi,%
	omicron,pi,rho,sigma,varsigma,tau,upsilon,phi,chi,psi,omega,digamma,%
	Alpha,Beta,Gamma,Delta,Epsilon,Zeta,Eta,Theta,Iota,Kappa,Lambda,Mu,Nu,Xi,%
	Omicron,Pi,Rho,Sigma,Tau,Upsilon,Phi,Chi,Psi,Omega,Digamma%
	\do{%
		\expandafter\let\csname\@tempa orig\expandafter\endcsname\csname\@tempa\endcsname%
		\expandafter\let\csname\@tempa uporig\expandafter\endcsname\csname\@tempa up\endcsname%
	}%
\makeatother
% <==

% LGR-encoded Greek letters
% ==>
\newcommand{\textformath}[1]{%
	\IfInBoldMode%
		\IfInUpMode\textbf{#1}\else\textit{\bfseries #1}\fi\relax%
	\else
		\IfInUpMode\textup{#1}\else\textit{#1}\fi\relax%
	\fi\relax%
}
% The double curly braces in this section are necessary to be able to use Greek letters
% in subscripts and superscripts without having to enclose theme in curly braces;
% for example, $\sigma_\epsilon$ instead of $\sigma_{\epsilon}$.
% Uppercase
\newcommand{\AlphaLGR}   {{\mathord{\textformath{\fontencoding{LGR}\selectfont A}}}}
\newcommand{\BetaLGR}    {{\mathord{\textformath{\fontencoding{LGR}\selectfont B}}}}
\newcommand{\GammaLGR}   {{\mathord{\textformath{\fontencoding{LGR}\selectfont G}}}}
\newcommand{\DeltaLGR}   {{\mathord{\textformath{\fontencoding{LGR}\selectfont D}}}}
\newcommand{\EpsilonLGR} {{\mathord{\textformath{\fontencoding{LGR}\selectfont E}}}}
\newcommand{\ZetaLGR}    {{\mathord{\textformath{\fontencoding{LGR}\selectfont Z}}}}
\newcommand{\EtaLGR}     {{\mathord{\textformath{\fontencoding{LGR}\selectfont H}}}}
\newcommand{\ThetaLGR}   {{\mathord{\textformath{\fontencoding{LGR}\selectfont J}}}}
\newcommand{\IotaLGR}    {{\mathord{\textformath{\fontencoding{LGR}\selectfont I}}}}
\newcommand{\KappaLGR}   {{\mathord{\textformath{\fontencoding{LGR}\selectfont K}}}}
\newcommand{\LambdaLGR}  {{\mathord{\textformath{\fontencoding{LGR}\selectfont L}}}}
\newcommand{\MuLGR}      {{\mathord{\textformath{\fontencoding{LGR}\selectfont M}}}}
\newcommand{\NuLGR}      {{\mathord{\textformath{\fontencoding{LGR}\selectfont N}}}}
\newcommand{\XiLGR}      {{\mathord{\textformath{\fontencoding{LGR}\selectfont X}}}}
\newcommand{\OmicronLGR} {{\mathord{\textformath{\fontencoding{LGR}\selectfont O}}}}
\newcommand{\PiLGR}      {{\mathord{\textformath{\fontencoding{LGR}\selectfont P}}}}
\newcommand{\RhoLGR}     {{\mathord{\textformath{\fontencoding{LGR}\selectfont R}}}}
\newcommand{\SigmaLGR}   {{\mathord{\textformath{\fontencoding{LGR}\selectfont S}}}}
\newcommand{\TauLGR}     {{\mathord{\textformath{\fontencoding{LGR}\selectfont T}}}}
\newcommand{\UpsilonLGR} {{\mathord{\textformath{\fontencoding{LGR}\selectfont U}}}}
\newcommand{\PhiLGR}     {{\mathord{\textformath{\fontencoding{LGR}\selectfont F}}}}
\newcommand{\ChiLGR}     {{\mathord{\textformath{\fontencoding{LGR}\selectfont Q}}}}
\newcommand{\PsiLGR}     {{\mathord{\textformath{\fontencoding{LGR}\selectfont Y}}}}
\newcommand{\OmegaLGR}   {{\mathord{\textformath{\fontencoding{LGR}\selectfont W}}}}
\newcommand{\DigammaLGR} {{\mathord{\textformath{\fontencoding{LGR}\selectfont \char195}}}}
% lowercase
\newcommand{\alphaLGR}   {{\mathord{\textformath{\fontencoding{LGR}\selectfont a}}}}
\newcommand{\betaLGR}    {{\mathord{\textformath{\fontencoding{LGR}\selectfont b}}}}
\newcommand{\gammaLGR}   {{\mathord{\textformath{\fontencoding{LGR}\selectfont g}}}}
\newcommand{\deltaLGR}   {{\mathord{\textformath{\fontencoding{LGR}\selectfont d}}}}
\newcommand{\epsilonLGR} {{\mathord{\textformath{\fontencoding{LGR}\selectfont e}}}}
\newcommand{\zetaLGR}    {{\mathord{\textformath{\fontencoding{LGR}\selectfont z}}}}
\newcommand{\etaLGR}     {{\mathord{\textformath{\fontencoding{LGR}\selectfont h}}}}
\newcommand{\thetaLGR}   {{\mathord{\textformath{\fontencoding{LGR}\selectfont j}}}}
\newcommand{\iotaLGR}    {{\mathord{\textformath{\fontencoding{LGR}\selectfont i}}}}
\newcommand{\kappaLGR}   {{\mathord{\textformath{\fontencoding{LGR}\selectfont k}}}}
\newcommand{\lambdaLGR}  {{\mathord{\textformath{\fontencoding{LGR}\selectfont l}}}}
\newcommand{\muLGR}      {{\mathord{\textformath{\fontencoding{LGR}\selectfont m}}}}
\newcommand{\nuLGR}      {{\mathord{\textformath{\fontencoding{LGR}\selectfont n}}}}
\newcommand{\xiLGR}      {{\mathord{\textformath{\fontencoding{LGR}\selectfont x}}}}
\newcommand{\omicronLGR} {{\mathord{\textformath{\fontencoding{LGR}\selectfont o}}}}
\newcommand{\piLGR}      {{\mathord{\textformath{\fontencoding{LGR}\selectfont p}}}}
\newcommand{\rhoLGR}     {{\mathord{\textformath{\fontencoding{LGR}\selectfont r}}}}
\newcommand{\sigmaLGR}   {{\mathord{\textformath{\fontencoding{LGR}\selectfont s\noboundary}}}}
	% \noboundary prevents sigma from being replaced by the word-end sigma (varsigma),
	% see http://mirrors.ctan.org/macros/latex/contrib/textgreek/textgreek.pdf
\newcommand{\varsigmaLGR}{{\mathord{\textformath{\fontencoding{LGR}\selectfont c}}}}
\newcommand{\tauLGR}     {{\mathord{\textformath{\fontencoding{LGR}\selectfont t}}}}
\newcommand{\upsilonLGR} {{\mathord{\textformath{\fontencoding{LGR}\selectfont u}}}}
\newcommand{\phiLGR}     {{\mathord{\textformath{\fontencoding{LGR}\selectfont f}}}}
\newcommand{\chiLGR}     {{\mathord{\textformath{\fontencoding{LGR}\selectfont q}}}}
\newcommand{\psiLGR}     {{\mathord{\textformath{\fontencoding{LGR}\selectfont y}}}}
\newcommand{\omegaLGR}   {{\mathord{\textformath{\fontencoding{LGR}\selectfont w}}}}
\newcommand{\digammaLGR} {{\mathord{\textformath{\fontencoding{LGR}\selectfont \char147}}}}
% Uppercase, upright
\newcommand{\AlphaupLGR}   {{\mathord{\textup{\fontencoding{LGR}\selectfont A}}}}
\newcommand{\BetaupLGR}    {{\mathord{\textup{\fontencoding{LGR}\selectfont B}}}}
\newcommand{\GammaupLGR}   {{\mathord{\textup{\fontencoding{LGR}\selectfont G}}}}
\newcommand{\DeltaupLGR}   {{\mathord{\textup{\fontencoding{LGR}\selectfont D}}}}
\newcommand{\EpsilonupLGR} {{\mathord{\textup{\fontencoding{LGR}\selectfont E}}}}
\newcommand{\ZetaupLGR}    {{\mathord{\textup{\fontencoding{LGR}\selectfont Z}}}}
\newcommand{\EtaupLGR}     {{\mathord{\textup{\fontencoding{LGR}\selectfont H}}}}
\newcommand{\ThetaupLGR}   {{\mathord{\textup{\fontencoding{LGR}\selectfont J}}}}
\newcommand{\IotaupLGR}    {{\mathord{\textup{\fontencoding{LGR}\selectfont I}}}}
\newcommand{\KappaupLGR}   {{\mathord{\textup{\fontencoding{LGR}\selectfont K}}}}
\newcommand{\LambdaupLGR}  {{\mathord{\textup{\fontencoding{LGR}\selectfont L}}}}
\newcommand{\MuupLGR}      {{\mathord{\textup{\fontencoding{LGR}\selectfont M}}}}
\newcommand{\NuupLGR}      {{\mathord{\textup{\fontencoding{LGR}\selectfont N}}}}
\newcommand{\XiupLGR}      {{\mathord{\textup{\fontencoding{LGR}\selectfont X}}}}
\newcommand{\OmicronupLGR} {{\mathord{\textup{\fontencoding{LGR}\selectfont O}}}}
\newcommand{\PiupLGR}      {{\mathord{\textup{\fontencoding{LGR}\selectfont P}}}}
\newcommand{\RhoupLGR}     {{\mathord{\textup{\fontencoding{LGR}\selectfont R}}}}
\newcommand{\SigmaupLGR}   {{\mathord{\textup{\fontencoding{LGR}\selectfont S}}}}
\newcommand{\TauupLGR}     {{\mathord{\textup{\fontencoding{LGR}\selectfont T}}}}
\newcommand{\UpsilonupLGR} {{\mathord{\textup{\fontencoding{LGR}\selectfont U}}}}
\newcommand{\PhiupLGR}     {{\mathord{\textup{\fontencoding{LGR}\selectfont F}}}}
\newcommand{\ChiupLGR}     {{\mathord{\textup{\fontencoding{LGR}\selectfont Q}}}}
\newcommand{\PsiupLGR}     {{\mathord{\textup{\fontencoding{LGR}\selectfont Y}}}}
\newcommand{\OmegaupLGR}   {{\mathord{\textup{\fontencoding{LGR}\selectfont W}}}}
\newcommand{\DigammaupLGR} {{\mathord{\textup{\fontencoding{LGR}\selectfont \char195}}}}
% lowercase, upright
\newcommand{\alphaupLGR}   {{\mathord{\textup{\fontencoding{LGR}\selectfont a}}}}
\newcommand{\betaupLGR}    {{\mathord{\textup{\fontencoding{LGR}\selectfont b}}}}
\newcommand{\gammaupLGR}   {{\mathord{\textup{\fontencoding{LGR}\selectfont g}}}}
\newcommand{\deltaupLGR}   {{\mathord{\textup{\fontencoding{LGR}\selectfont d}}}}
\newcommand{\epsilonupLGR} {{\mathord{\textup{\fontencoding{LGR}\selectfont e}}}}
\newcommand{\zetaupLGR}    {{\mathord{\textup{\fontencoding{LGR}\selectfont z}}}}
\newcommand{\etaupLGR}     {{\mathord{\textup{\fontencoding{LGR}\selectfont h}}}}
\newcommand{\thetaupLGR}   {{\mathord{\textup{\fontencoding{LGR}\selectfont j}}}}
\newcommand{\iotaupLGR}    {{\mathord{\textup{\fontencoding{LGR}\selectfont i}}}}
\newcommand{\kappaupLGR}   {{\mathord{\textup{\fontencoding{LGR}\selectfont k}}}}
\newcommand{\lambdaupLGR}  {{\mathord{\textup{\fontencoding{LGR}\selectfont l}}}}
\newcommand{\muupLGR}      {{\mathord{\textup{\fontencoding{LGR}\selectfont m}}}}
\newcommand{\nuupLGR}      {{\mathord{\textup{\fontencoding{LGR}\selectfont n}}}}
\newcommand{\xiupLGR}      {{\mathord{\textup{\fontencoding{LGR}\selectfont x}}}}
\newcommand{\omicronupLGR} {{\mathord{\textup{\fontencoding{LGR}\selectfont o}}}}
\newcommand{\piupLGR}      {{\mathord{\textup{\fontencoding{LGR}\selectfont p}}}}
\newcommand{\rhoupLGR}     {{\mathord{\textup{\fontencoding{LGR}\selectfont r}}}}
\newcommand{\sigmaupLGR}   {{\mathord{\textup{\fontencoding{LGR}\selectfont s\noboundary}}}}
	% \noboundary prevents sigma from being replaced by the word-end sigma (varsigma),
	% see http://mirrors.ctan.org/macros/latex/contrib/textgreek/textgreek.pdf
\newcommand{\varsigmaupLGR}{{\mathord{\textup{\fontencoding{LGR}\selectfont c}}}}
\newcommand{\tauupLGR}     {{\mathord{\textup{\fontencoding{LGR}\selectfont t}}}}
\newcommand{\upsilonupLGR} {{\mathord{\textup{\fontencoding{LGR}\selectfont u}}}}
\newcommand{\phiupLGR}     {{\mathord{\textup{\fontencoding{LGR}\selectfont f}}}}
\newcommand{\chiupLGR}     {{\mathord{\textup{\fontencoding{LGR}\selectfont q}}}}
\newcommand{\psiupLGR}     {{\mathord{\textup{\fontencoding{LGR}\selectfont y}}}}
\newcommand{\omegaupLGR}   {{\mathord{\textup{\fontencoding{LGR}\selectfont w}}}}
\newcommand{\digammaupLGR} {{\mathord{\textup{\fontencoding{LGR}\selectfont \char147}}}}
% <==

% Based on description of the TS1 encoding in
% http://ctan.math.illinois.edu/macros/latex/doc/encguide.pdf:
%\let \oldpm    \pm
%\let \oldtimes \times
%\let \olddiv   \div
%\makeatletter
%\newcommand{\pmsf}   {\mathbin{\text{\usefont{TS1}{\sfdefault}{\f@series}{n}\char"B1}}}
%\newcommand{\timessf}{\mathbin{\text{\usefont{TS1}{\sfdefault}{\f@series}{n}\char"D6}}}
%\newcommand{\divsf}  {\mathbin{\text{\usefont{TS1}{\sfdefault}{\f@series}{n}\char"F6}}}
%\makeatother

% Use LGR-encoded Greek letters for \mathversion{sans}
% ==>
\makeatletter

% Save original definition of \varepsilon etc.
\@for\@tempa:=%
epsilon,theta,pi,rho,phi%
\do{%
	\expandafter\let\csname var\@tempa orig\expandafter\endcsname\csname var\@tempa\endcsname%
}

\newcommand*{\sansmath}{%
	\@for\@tempa:=%
		alpha,beta,gamma,delta,epsilon,zeta,eta,theta,iota,kappa,lambda,mu,nu,xi,%
		omicron,pi,rho,sigma,varsigma,tau,upsilon,phi,chi,psi,omega,digamma,%
		Alpha,Beta,Gamma,Delta,Epsilon,Zeta,Eta,Theta,Iota,Kappa,Lambda,Mu,Nu,Xi,%
		Omicron,Pi,Rho,Sigma,Tau,Upsilon,Phi,Chi,Psi,Omega,Digamma%
	\do{%
		\expandafter\let\csname\@tempa\expandafter\endcsname\csname\@tempa LGR\endcsname%
		\expandafter\let\csname\@tempa up\expandafter\endcsname\csname\@tempa upLGR\endcsname%
		\expandafter\let\csname up\@tempa\expandafter\endcsname\csname\@tempa upLGR\endcsname%
	}%
	\@for\@tempa:=%
		epsilon,theta,pi,rho,phi%
	\do{%
		\expandafter\let\csname var\@tempa\expandafter\endcsname\csname\@tempa\endcsname%
	}%
	%\renewcommand{\pm}{\pmsf}%
	%\renewcommand{\times}{\timessf}%
	%\renewcommand{\div}{\divsf}%
}
% <==

% Switch back to the original Greek letters for \mathversion{normal}, i.e., the serif font
% ==>
\newcommand*{\unsansmath}{%
	\@for\@tempa:=%
		alpha,beta,gamma,delta,epsilon,zeta,eta,theta,iota,kappa,lambda,mu,nu,xi,%
		omicron,pi,rho,sigma,varsigma,tau,upsilon,phi,chi,psi,omega,digamma,%
		Alpha,Beta,Gamma,Delta,Epsilon,Zeta,Eta,Theta,Iota,Kappa,Lambda,Mu,Nu,Xi,%
		Omicron,Pi,Rho,Sigma,Tau,Upsilon,Phi,Chi,Psi,Omega,Digamma%
	\do{%
		\expandafter\let\csname\@tempa\expandafter\endcsname\csname\@tempa orig\endcsname%
		\expandafter\let\csname\@tempa up\expandafter\endcsname\csname\@tempa uporig\endcsname%
		\expandafter\let\csname up\@tempa\expandafter\endcsname\csname\@tempa uporig\endcsname%
	}%
	\@for\@tempa:=%
		epsilon,theta,pi,rho,phi%
	\do{%
		\expandafter\let\csname var\@tempa\expandafter\endcsname\csname var\@tempa orig\endcsname%
	}%
	%\renewcommand{\pm}{\oldpm}%
	%\renewcommand{\times}{\oldtimes}%
	%\renewcommand{\div}{\olddiv}%
}
% <==

%% If you would like to use LGR-encoded Greek letters also for the serif font
%% ==>
%\renewcommand*{\unsansmath}{%
%	\@for\@tempa:=%
%		alpha,beta,gamma,delta,epsilon,zeta,eta,theta,iota,kappa,lambda,mu,nu,xi,%
%		omicron,pi,rho,sigma,varsigma,tau,upsilon,phi,chi,psi,omega,digamma,%
%		Alpha,Beta,Gamma,Delta,Epsilon,Zeta,Eta,Theta,Iota,Kappa,Lambda,Mu,Nu,Xi,%
%		Omicron,Pi,Rho,Sigma,Tau,Upsilon,Phi,Chi,Psi,Omega,Digamma%
%		\do{%
%			\expandafter\let\csname\@tempa\expandafter\endcsname\csname\@tempa LGR\endcsname%
%			\expandafter\let\csname\@tempa up\expandafter\endcsname\csname\@tempa upLGR\endcsname%
%			\expandafter\let\csname up\@tempa\expandafter\endcsname\csname\@tempa upLGR\endcsname%
%		}%
%	%\renewcommand{\pm}{\pmsf}%
%	%\renewcommand{\times}{\timessf}%
%	%\renewcommand{\div}{\divsf}%
%}
%% <==

\newcommand*{\upgreekletters}{%
	\@for\@tempa:=%
		alpha,beta,gamma,delta,epsilon,zeta,eta,theta,iota,kappa,lambda,mu,nu,xi,%
		omicron,pi,rho,sigma,varsigma,tau,upsilon,phi,chi,psi,omega,digamma,%
		Alpha,Beta,Gamma,Delta,Epsilon,Zeta,Eta,Theta,Iota,Kappa,Lambda,Mu,Nu,Xi,%
		Omicron,Pi,Rho,Sigma,Tau,Upsilon,Phi,Chi,Psi,Omega,Digamma%
		\do{%
			\expandafter\let\csname\@tempa\expandafter\endcsname\csname\@tempa up\endcsname%
		}%
}
\newcommand*{\itgreekletters}{%
	\@for\@tempa:=%
		alpha,beta,gamma,delta,epsilon,zeta,eta,theta,iota,kappa,lambda,mu,nu,xi,%
		omicron,pi,rho,sigma,varsigma,tau,upsilon,phi,chi,psi,omega,digamma,%
		Alpha,Beta,Gamma,Delta,Epsilon,Zeta,Eta,Theta,Iota,Kappa,Lambda,Mu,Nu,Xi,%
		Omicron,Pi,Rho,Sigma,Tau,Upsilon,Phi,Chi,Psi,Omega,Digamma%
		\do{%
			\expandafter\let\csname\@tempa\expandafter\endcsname\csname\@tempa orig\endcsname%
		}%
}

\makeatother

%\makeatletter
%	\@for\@tempa:=%
%	%alpha,beta,gamma,delta,epsilon,zeta,eta,theta,iota,kappa,lambda,mu,nu,xi,%
%	%pi,rho,sigma,varsigma,tau,upsilon,phi,chi,psi,omega,digamma,%
%	Gamma,Delta,Theta,Lambda,Xi,Pi,Sigma,Upsilon,Phi,Psi,Omega%
%	\do{\expandafter\let\csname\@tempa\expandafter\endcsname\csname other\@tempa\endcsname}%
%\makeatother

% Fix the \bm command so that it also works properly in the sans mathversions
% ==>
\let \bmorig \bm
\renewcommand{\bm}[1]{%
	\IfInSansMode%
		\textbf{\mathversion{boldsans}\(#1\)}%
	\else%
		\bmorig{#1}%
	\fi\relax%
}
% <==
\renewcommand{\mathbf}[1]{\bm{#1}}
\renewcommand{\boldsymbol}[1]{\bm{#1}}
\newcommand{\mathbfit}[1]{\mathbf{\mathit{#1}}}
\renewcommand{\mathcal}[1]{\mathscr{#1}}

% Apply sansmath etc. automagically
% ==>
\newif\IfInSansMode
\newif\IfInBoldMode
\newif\IfInUpMode
\let \oldsf \sffamily
\renewcommand*{\sffamily}{%
	\oldsf\sansmath\InSansModetrue%
	\IfInBoldMode\mathversion{boldsans}\else\mathversion{sans}\fi\relax%
}
\let \oldbf \bfseries
\renewcommand*{\bfseries}{%
	\oldbf\InBoldModetrue%
	\IfInSansMode\sansmath\mathversion{boldsans}\else\mathversion{bold}\fi\relax%
}
\let \oldmd \mdseries
\renewcommand*{\mdseries}{%
	\oldmd\InBoldModefalse%
	\IfInSansMode\sansmath\mathversion{sans}\else\mathversion{normal}\fi\relax%
}
\let \oldnorm \normalfont
\renewcommand*{\normalfont}{%
	\oldnorm\InSansModefalse\InBoldModefalse\mathversion{normal}%
	\unsansmath%
}
\let \oldrm \rmfamily
\renewcommand*{\rmfamily}{%
	\oldrm\InSansModefalse%
	\IfInBoldMode\mathversion{bold}\else\mathversion{normal}\fi\relax%
	\unsansmath%
}
% <==

% Make \mathnormal obey the currently active \mathversion ==>
\let \mathnormalorig \mathnormal
\renewcommand{\mathnormal}[1]{%
	\IfInSansMode%
		\IfInBoldMode%
			\mathversion{boldsans}%
			{\textbf{\(#1\)}}%
		\else%
			\mathversion{sans}%
			{\textmd{\(#1\)}}%
		\fi\relax%	
	\else%
		\mathnormalorig{#1}%
	\fi\relax%
}
% <==

% Adjust \mathrm to the curretly active \mathversion.
% We set it up such that also in sansserif mode, \mathrm activates the serif font.
% ==>
\let \mathrmorig \mathrm
\renewcommand{\mathrm}[1]{%
	\IfInSansMode%
		{\textrm{%
			\IfInBoldMode%
				\mathversion{bold}%
				\(\mathrmorig{#1}\)%
			\else%
				\mathversion{normal}%
				\(\mathrmorig{#1}\)%
			\fi\relax%
		}}%
	\else%
		\mathrmorig{#1}%
	\fi\relax%
}
% <==

% Define \mathup to activate \upshape without switching to the serif font
% (in contrast to \mathrm)
% ==>
\newcommand{\mathup}[1]{%
	\IfInSansMode%
		{\textup{%
			\InUpModetrue%
			\IfInBoldMode%
				\mathversion{boldsansup}%
				\(#1\)%
			\else%
				\mathversion{sansup}%
				\(#1\)%
			\fi\relax%
		}}%
	\else%
		{\upgreekletters\mathrm{#1}\itgreekletters}%
	\fi\relax%
}
\newcommand{\mathbfup}[1]{%
	\IfInSansMode%
		{\mathbf{\mathup{#1}}}%
	\else%
		{\upgreekletters\mathbf{\mathrm{#1}}\itgreekletters}%
	\fi\relax%
}
% <==

%% If you would like to redefine \mathup also for the serif font:
%% ==>
%\renewcommand{\mathup}[1]{%
%	\IfInSansMode%
%		{\textup{%
%			\InUpModetrue%
%			\IfInBoldMode%
%				\mathversion{boldsansup}%
%				\(#1\)%
%			\else%
%				\mathversion{sansup}%
%				\(#1\)%
%			\fi\relax%
%		}}%
%	\else%
%		{\textup{%
%			\InUpModetrue%
%			\IfInBoldMode%
%				\mathversion{boldup}%
%				\(#1\)%
%			\else%
%				\mathversion{normalup}%
%				\(#1\)%
%			\fi\relax%
%		}}%
%	\fi\relax%
%}
%\renewcommand{\mathbfup}[1]{%
%	\IfInSansMode%
%		{\mathbf{\mathup{#1}}}%
%	\else%
%		{\upgreekletters\mathbf{\mathup{#1}}\itgreekletters}%
%	\fi\relax%
%}
%% <==

% Make the LaTeX-defined operators obey sansserif math
% ==>
\let \operatornameorig \operatorname
\renewcommand{\operatorname}[1]{%
	\operatornameorig{\mathup{#1}}%
}
\makeatletter
\@for\@tempa:=%
	arccos,arccot,arccsc,arcsec,arcsin,arctan,arg,cos,cosh,cot,coth,csc,%
	deg,det,dim,exp,gcd,hom,inf,ker,lg,lim,liminf,limsup,ln,log,max,min,%
	Pr,sec,sin,sinh,sup,tan,tanh%
	\do{%
		\expandafter\let\csname\@tempa\endcsname\relax%
	}%
\makeatother
\DeclareMathOperator {\arccos}{\mathup{arccos}}
\DeclareMathOperator {\arccot}{\mathup{arccot}}
\DeclareMathOperator {\arccsc}{\mathup{arccsc}}
\DeclareMathOperator {\arcsec}{\mathup{arcsec}}
\DeclareMathOperator {\arcsin}{\mathup{arcsin}}
\DeclareMathOperator {\arctan}{\mathup{arctan}}
\DeclareMathOperator {\arg}   {\mathup{arg}}
\DeclareMathOperator {\cos}   {\mathup{cos}}
\DeclareMathOperator {\cosh}  {\mathup{cosh}}
\DeclareMathOperator {\cot}   {\mathup{cot}}
\DeclareMathOperator {\coth}  {\mathup{coth}}
\DeclareMathOperator {\csc}   {\mathup{csc}}
\DeclareMathOperator {\deg}   {\mathup{deg}}
\DeclareMathOperator {\det}   {\mathup{det}}
\DeclareMathOperator {\dim}   {\mathup{dim}}
\DeclareMathOperator {\exp}   {\mathup{exp}}
\DeclareMathOperator {\gcd}   {\mathup{gcd}}
\DeclareMathOperator*{\hom}   {\mathup{hom}}
\DeclareMathOperator*{\inf}   {\mathup{inf}}
\DeclareMathOperator {\ker}   {\mathup{ker}}
\DeclareMathOperator {\lg}    {\mathup{lg}}
\DeclareMathOperator*{\lim}   {\mathup{lim}}
\DeclareMathOperator*{\liminf}{\mathup{lim\,inf}}
\DeclareMathOperator*{\limsup}{\mathup{lim\,sup}}
\DeclareMathOperator {\ln}    {\mathup{ln}}
\DeclareMathOperator {\log}   {\mathup{log}}
\DeclareMathOperator*{\max}   {\mathup{max}}
\DeclareMathOperator*{\min}   {\mathup{min}}
\DeclareMathOperator {\Pr}    {\mathup{Pr}}
\DeclareMathOperator {\sec}   {\mathup{sec}}
\DeclareMathOperator {\sin}   {\mathup{sin}}
\DeclareMathOperator {\sinh}  {\mathup{sinh}}
\DeclareMathOperator*{\sup}   {\mathup{sup}}
\DeclareMathOperator {\tan}   {\mathup{tan}}
\DeclareMathOperator {\tanh}  {\mathup{tanh}}
% <==

% Allow for fine-grained scaling of font sizes
% ==>
\usepackage{relsize}
\renewcommand\RSpercentTolerance{1}
% Enabling slightly reduced font for CAPS:
\newcommand{\caps}[1]{\textscale{0.96}{\textls[35]{\MakeUppercase{#1}}}}
% <==




%%%%%%%%%%%%%%%%%%%%%%%%%%%%%%%%%%%%%%%%%%%%%%%%
%%  ADJUSTING THE DESIGN OF THE PRESENTATION  %%
%%%%%%%%%%%%%%%%%%%%%%%%%%%%%%%%%%%%%%%%%%%%%%%%


\setbeamerfont{title}
	{parent=structure, size=\LARGE, series=\bfseries}
\setbeamerfont{subtitle}
	{parent=structure, size=\Large, series=\mdseries}
\setbeamerfont{frametitle}
	{parent=structure, size=\large}

% For convenience, let the ``paperheight'' of the slides be identical,
% no matter what the aspect ratio of the slides ==>
\makeatletter
% 4:3 aspect ratio:
\@ifclasswith{beamer}{aspectratio=43}{%
	\beamer@paperwidth 12.00cm%
	\beamer@paperheight 9.00cm%
	\setbeamerfont{title}
		{parent=structure, size*={16}{20}}
	\setbeamerfont{subtitle}
		{parent=structure, size*={13.5}{17}}
	\setbeamerfont{frametitle}
		{parent=structure, size*={11}{14}}
}{}
%% 14:9 aspect ratio: Nothing needs to be changed
%\@ifclasswith{beamer}{aspectratio=149}{%
%	\beamer@paperwidth 14.00cm%
%	\beamer@paperheight 9.00cm%
%}{}
%% 16:9 aspect ratio: Nothing needs to be changed
%\@ifclasswith{beamer}{aspectratio=169}{%
%	\beamer@paperwidth 16.00cm%
%	\beamer@paperheight 9.00cm%
%}{}
% 16:10 aspect ratio:
\@ifclasswith{beamer}{aspectratio=1610}{%
	\beamer@paperwidth 14.40cm%
	\beamer@paperheight 9.00cm%
}{}
\makeatother
% <==

% Margins ==>
\newcommand{\margintop}{12.5pt}
\newcommand{\marginleft}{17.5pt}
\newcommand{\marginright}{\marginleft}
\setbeamersize{text margin left=\marginleft}
\setbeamersize{text margin right=\marginright}
\setlength{\textwidth}{\paperwidth-\marginleft-\marginright}
% <==

\setlength{\parskip}{\medskipamount}
% Inserts some space between paragraphs

% Enable hyphenation on Beamer slides:
\usepackage{ragged2e}
\let \raggedright \RaggedRight
\sloppy
\hyphenpenalty=500

\setbeamerfont{alerted text}{series=\bfseries}

%% Lorenz Götte's color scheme:
%\usecolortheme{whale}
%\definecolor{beamer@blendedblue}{rgb}{0.137,0.466,0.741}
%\setbeamercolor{structure}{fg=beamer@blendedblue}
%\definecolor{SpotColor}{rgb}{0.1,0.4,0.7} % Lorenz' Blue
\definecolor{UBonnBlue}   {RGB}{  7,  82, 154}
\definecolor{UBonnYellow} {RGB}{234, 185,  12}
\definecolor{UBonnGray}   {RGB}{144, 144, 133}
\definecolor{darkgray}    {RGB}{102, 102, 102}
\definecolor{darkred}     {RGB}{191,   0,   0}
\definecolor{neutralgreen}{RGB}{ 28, 166,   0}

\colorlet{SpotColor}   {UBonnBlue}
\colorlet{AlertColor}  {darkred}
\colorlet{ExampleColor}{neutralgreen}

\setbeamercolor{structure}{fg=SpotColor}
\setbeamercolor{alerted text}{fg=AlertColor}
\setbeamercolor{author in head/foot}{fg=white, bg=SpotColor}
\setbeamercolor{button}{bg=SpotColor,fg=white}
\setbeamercolor{headline}{bg=SpotColor,fg=white}
\setbeamercolor{headlinecover}{bg=white,fg=white}

\newcommand{\highlight}[1]{\textcolor{SpotColor}{#1}}
\newcommand{\heading}[1]{%
	\par%
	\textbf{\usebeamerfont{structure}\usebeamercolor[fg]{frametitle}#1}%
	\par%
}
\newcommand{\sigstar}{\highlight{*}}

\renewcommand{\alert}[1]{\highlight{\textbf{#1}}}

\newlength{\rulelength}
\setlength{\rulelength}{\paperwidth - \marginleft - \marginright}
\newcommand{\sliderule}{%
	\textcolor{SpotColor}{\rule{\rulelength}{.35pt}}%
}

% Remove navigation symbols from the slide footer
\beamertemplatenavigationsymbolsempty

% Adjust layout of frametitle
% ==>
\setbeamertemplate{frametitle}{%
	\vspace{\margintop}\vspace{-5pt}%
	{\large\usebeamercolor{frametitle}\usebeamerfont{frametitle}\insertframetitle\\[-1.2ex]}%
	\sliderule\vspace{-2.5pt}%
}
% <==

\newlength{\navigationwidth}
\BeforeBeginEnvironment{frame}{%
	% Add a slide number (and [short]title) to the footer of each slide
	% ==>
	\setbeamertemplate{footline}{\vskip-15pt%
		\hspace{\marginleft}%
		\sliderule \vspace{1.1ex} \linebreak
		\mbox{\hspace{\marginleft}}%
		\textmd{\insertshortauthor: \enquote{\insertshorttitle}}
		\hfill%
		\textbf{\insertframenumber/\inserttotalframenumber}%
		\hspace{\marginright}%
		\vspace{12pt}%
	}%
	% <==
	% Outline with current section highlighted in the ``headline'' ==>
	\ifnum \showsectionnavigation=0
	\setbeamertemplate{headline}{}
	\else
	\setbeamertemplate{headline}{%
		\hspace{\marginleft}%
		\begin{beamercolorbox}[wd=\linewidth, ht=5pt, dp=2pt]{headline}%
			\insertsectionnavigationhorizontal{\linewidth}{\hspace{0.04\linewidth}}{\hspace{0.04\linewidth}}%
		\end{beamercolorbox}%
		\newline
		\begin{beamercolorbox}[wd=\paperwidth, ht=15pt, dp=0pt]{headlinecover}%
			% Adds a white bar to cover stuff that protrudes from the above color box.
		\end{beamercolorbox}%
	}
	\setlength{\fboxrule}{0.25pt}\setlength{\fboxsep}{7.5pt}%
	\setbeamertemplate{section in head/foot}{%
		\settowidth{\navigationwidth}{\bfseries\insertsectionhead}%
		\fcolorbox{SpotColor!25}{SpotColor}{%
			\makebox[\navigationwidth]{\color{white}\bfseries\insertsectionhead}%
		}%
	}
	\setbeamertemplate{section in head/foot shaded}{%
		\settowidth{\navigationwidth}{\bfseries\insertsectionhead}%
		\fcolorbox{SpotColor}{SpotColor}{%
			\makebox[\navigationwidth]{\color{white}\mdseries\insertsectionhead}%
		}
	}
	\fi
	% <==
}

% See https://tex.stackexchange.com/questions/427257/how-to-remove-footer-for-specific-type-of-slides-frames
% ==>
\makeatletter
\define@key{beamerframe}{standout}[true]{%
	\setbeamertemplate{footline}{\vskip-15pt%
		\hspace{\marginleft}%
		\sliderule \vspace{1.1ex} \linebreak
		\mbox{\hspace{\marginleft}}%
		\textmd{\phantom{\insertshortauthor: \enquote{\insertshorttitle}}}
		\hfill%
		\phantom{\insertframenumber/\inserttotalframenumber}%
		\hspace{\marginright}%
		\vspace{15pt}%
	}%
	\setbeamertemplate{headline}{}%
}
\makeatother
% <==

\frenchspacing  % Prevent increased whitespace after periods and colons

\setbeamertemplate{button}{%
	\tikz
	\node[
	inner xsep=3pt,
	inner ysep=2pt,
	draw=structure!100,
	fill=structure!100,
	rounded corners=1.5pt
	]{\raisebox{1.5pt}{\usebeamerfont{structure}\insertbuttontext}};%
}

%\renewcommand{\texteuro}{\fontencoding{TS1}\selectfont\char"BF\fontencoding{T1}\selectfont}
%\newcommand{\euro}{\texteuro}

% Adding framenumbers automatically to the frametitles
\newcommand{\pageinsection}{\number\numexpr\insertpagenumber-\insertsectionstartpage+1}
% From https://tex.stackexchange.com/questions/308343/how-to-create-mini-sections-mini-subsections-and-mini-frames-in-beamer-presenta
% ==>
%\usepackage{etoolbox}  % Automatically loaded by beamer
\makeatletter
\newcount\beamer@sectionstartframe
\beamer@sectionstartframe=1
\apptocmd{\beamer@section}{%
	\addtocontents{nav}{\protect\headcommand{%
			\protect\beamer@sectionframes{\the\beamer@sectionstartframe}{\the\c@framenumber}}}%
}{}{}
\apptocmd{\beamer@section}{%
	\beamer@sectionstartframe=\c@framenumber\advance\beamer@sectionstartframe by1\relax%
}{}{}
\AtEndDocument{%
	\immediate\write\@auxout{\string\@writefile{nav}%
		{\noexpand\headcommand{\noexpand\beamer@sectionframes{\the\beamer@sectionstartframe}{\the\c@framenumber}}}}%
}{}{}
\def\beamer@startframeofsection{1}
\def\beamer@endframeofsection{1}
\def\beamer@sectionframes#1#2{%
	\ifnum\c@framenumber<#1%
	\else%
	\ifnum\c@framenumber>#2%
	\else%
	\gdef\beamer@startframeofsection{#1}%
	\gdef\beamer@endframeofsection{#2}%
	\fi%
	\fi%
}
\newcommand\insertsectionstartframe{\beamer@startframeofsection}
\newcommand\insertsectionendframe{\beamer@endframeofsection}
\makeatother
% <==
\newcommand{\frameinsection}{\number\numexpr\insertframenumber-\insertsectionstartframe+1}
% https://tex.stackexchange.com/questions/228684/two-counters-for-beamer-presentations
%\newcommand{\titleprefix}{\insertsection~{\pageinsection}}
\newcommand{\titleprefix}{\insertsection~{\frameinsection}}
% <==

% Continuation counter for frames with the ``allowframebreaks'' option.
% ==>
%% Inspired by https://tex.stackexchange.com/questions/275044/how-do-i-insert-the-total-continuation-count-in-the-allowframbreaks-frame-title:
%\newcounter{totalcontinuationcount}
%\makeatletter
%\setbeamertemplate{frametitle continuation}{%
%	\setcounter{totalcontinuationcount}{\beamer@endpageofframe}%
%	\addtocounter{totalcontinuationcount}{1}%
%	\addtocounter{totalcontinuationcount}{-\beamer@startpageofframe}%
%	\ifnum \value{totalcontinuationcount} > 1
%		\textmd{(\insertcontinuationcount/\arabic{totalcontinuationcount})}%
%	\fi
%}
%\makeatother
% More elegant version based on https://github.com/josephwright/beamer/issues/423#issuecomment-456494500:
\makeatletter
\defbeamertemplate*{frametitle continuation}{only if multiple}{%
	\ifnum \numexpr\beamer@endpageofframe+1-\beamer@startpageofframe\relax > 1
	\textmd{(%
		\insertcontinuationcount/%
		\the\numexpr\beamer@endpageofframe+1-\beamer@startpageofframe%
		)}%
	\fi%
}
\makeatother
% <==




%%%%%%%%%%%%%%%%%%%%%%%%%%%%%%%%%
%%  LAYOUT OF THE TITLE SLIDE  %%
%%%%%%%%%%%%%%%%%%%%%%%%%%%%%%%%%


% Make the title page left-aligned
% ==>
%\setbeamertemplate{title page}[default][left]

% \setbeamerfont{title} and \setbeamerfont{subtitle} already executed above!
\setbeamerfont{author}
	{parent=normal text, size=\large, series=\mdseries, shape=\upshape}
\setbeamercolor{author}
	{parent=normal text}
\setbeamerfont{institute}
	{parent=normal text, size=\small, series=\mdseries, shape=\itshape}
\setbeamercolor{institute}
	{parent=normal text}
\setbeamerfont{date}
	{parent=structure, size=\normalsize, series=\bfseries}
\setbeamercolor{date}
	{parent=normal text}

\makeatletter
\setbeamertemplate{title page}{
	\begin{minipage}[c][6.75cm]{\textwidth}%
		\raggedright%
		\ifx\inserttitlegraphic\@empty\else%
			\inserttitlegraphic%
		\fi
		\vfill
		\begingroup
			\ifx\inserttitle\@empty\else%
				\usebeamerfont{title}\usebeamercolor[fg]{title}\inserttitle\par%
			\fi
			\ifx\insertsubtitle\@empty\else%
				\medskip
				\usebeamerfont{subtitle}\usebeamercolor[fg]{subtitle}\insertsubtitle\par%
			\fi
			\ifx\beamer@shortauthor\@empty\else%
				\vspace{\fill}
				\usebeamerfont{author}\usebeamercolor[fg]{author}\insertauthor\par%
			\fi
			\ifx\insertinstitute\@empty\else%
				\vspace{\fill}
				\usebeamerfont{institute}\usebeamercolor[fg]{institute}\insertinstitute\par%
			\fi
			\ifx\insertdate\@empty\else%
				\vspace{\fill}
				\usebeamerfont{date}\usebeamercolor[fg]{date}\insertdate%
			\fi
		\endgroup
		\vfill
		\vspace{0cm}  % For the \vfill to actually have an effect
	\end{minipage}%
}
\makeatother
% <==

\makeatletter
\renewcommand{\beamer@insttitle}[1]{\highlight{\textsuperscript{\kern.75pt \textit{#1}}}}
\renewcommand{\beamer@instinst}[1]{\beamer@insttitle{#1}\ignorespaces}
\renewcommand{\beamer@andinst}{\\[0.33\baselineskip]}
\makeatother




%%%%%%%%%%%%%%%%%%%%%%%%%%%%%%%%%%%%%%%
%%  LAYOUT OF THE TABLE OF CONTENTS  %%
%%%%%%%%%%%%%%%%%%%%%%%%%%%%%%%%%%%%%%%


\useinnertheme{circles}
\setbeamertemplate{section in toc}{%
	\leavevmode\leftskip=3.3ex%
	\llap{%
		\usebeamerfont*{section number projected}%
		\usebeamercolor{section number projected}%
		\begin{pgfpicture}{-1ex}{0ex}{1ex}{2ex}
			\color{bg}
			\pgfpathcircle{\pgfpoint{0pt}{.75ex}}{1.2ex}
			\pgfusepath{fill}
			\pgftext[base]{\color{fg}\raisebox{0.07ex}{%
					% \addfontfeatures{Numbers={Lining, Monospaced}}%
					\usebeamerfont{structure}%
					\small\inserttocsectionnumber}%
			}
		\end{pgfpicture}\kern1.5ex%
	}%
	\usebeamerfont{normal text}%
	\inserttocsection\par%
}
\setbeamertemplate{subsection in toc}{%
	\leavevmode\leftskip=2em$\bullet$\hskip1em\inserttocsubsection\par%
}

\makeatletter
\patchcmd{\beamer@sectionintoc}
{\vfill}
{\medskip}
{}
{}
\makeatother




%%%%%%%%%%%%%%
%%  BLOCKS  %%
%%%%%%%%%%%%%%


% Put border around block, exampleblock, and alertblock
% ==>

\usepackage{mdframed}
\usepackage{changepage}

\newlength{\blockinnermargin}
\setlength{\blockinnermargin}{4pt}

\newcommand{\framedblockcolor}{SpotColor}  % This allows us to change the color mid-presentation
\newcommand{\PutBorderAroundBlock}[2]{
	\renewcommand{\framedblockcolor}{#2}%
	\setbeamercolor*{block title}{fg=white, bg=\framedblockcolor}%
	\setbeamercolor*{block body} {fg=black, bg=}%
	\addtobeamertemplate{block begin}{%
		\setbeamercolor*{item}{fg=\framedblockcolor}%
		\begin{mdframed}[linecolor=\framedblockcolor, linewidth=0.35pt, leftmargin=0pt, innerleftmargin=0.375em, innerrightmargin=0.375em, innertopmargin=-0.01ex, innerbottommargin=4pt]%
			\let \insertblocktitleorig \insertblocktitle%
			\renewcommand{\insertblocktitle}{%
				\RaggedRight%
				\leftskip=\blockinnermargin%
				\rightskip=\blockinnermargin%
				\insertblocktitleorig\strut\vspace{-1.5pt}%
					% \strut\vspace{-1.5pt} so that all single-line block titles have identical hight
			}%
	}{%
		\renewcommand{\insertblocktitle}{\insertblocktitleorig}%
		\begin{adjustwidth}{\blockinnermargin}{\blockinnermargin}%
			\RaggedRight%
	}%
	\addtobeamertemplate{block end}{%
		\end{adjustwidth}%
	}{%
		\end{mdframed}%
	}%
}
\PutBorderAroundBlock{}{SpotColor}  % Default color of blocks is SpotColor

\newcommand{\PutBorderAroundBlockExAlert}[2]{%
	\def\temp{#1}%
	\ifx\temp\empty
		\def\blocktype{block }     % Trailing space is crucial!
	\else
		\def\temp{ #1}
		\def\blocktype{block #1 }  % Trailing space is crucial!
	\fi
	\setbeamercolor*{block title\temp}{fg=white, bg=#2}
	\setbeamercolor*{block body\temp} {fg=black, bg=}
	\addtobeamertemplate{\blocktype begin}{%
		\begin{mdframed}[linecolor=#2, linewidth=0.35pt, leftmargin=0pt, innerleftmargin=0.375em, innerrightmargin=0.375em, innertopmargin=-0.01ex, innerbottommargin=4pt]%
			\let \insertblocktitleorig \insertblocktitle%
			\renewcommand{\insertblocktitle}{%
				\RaggedRight%
				\leftskip=\blockinnermargin%
				\rightskip=\blockinnermargin%
				\insertblocktitleorig\strut\vspace{-1.5pt}%
					% \strut\vspace{-1.5pt} so that all single-line block titles have identical hight
			}%
	}{%
		\renewcommand{\insertblocktitle}{\insertblocktitleorig}%
		\begin{adjustwidth}{\blockinnermargin}{\blockinnermargin}%
			\RaggedRight%
	}%
	\addtobeamertemplate{\blocktype end}{%
		\end{adjustwidth}%
	}{%
		\end{mdframed}%
	}%
}
\PutBorderAroundBlockExAlert{example}{ExampleColor}
\PutBorderAroundBlockExAlert{alerted}{AlertColor}

% <==

% Style definition, theorem, lemma, corollary, and proof with out border
% ==>

% Inspired by https://github.com/schnorr/infufrgs/blob/master/slides/beamerthemeInf.sty
\setbeamercolor{theorem title}{fg=SpotColor}
\setbeamerfont{theorem title}{parent=structure}
\setbeamercolor{theorem body}{fg=black}
\setbeamerfont{theorem body}{parent=normal text, shape=\itshape, series=\mdseries}
% Body of theorem, lemma, and corollary are italicized
\AtBeginEnvironment{definition}{%
	\setbeamerfont{theorem body}{parent=normal text, shape=\upshape, series=\mdseries}
	% Body of definition is typeset in upright font
}

\setbeamertemplate{theorem begin}{%
	\usebeamerfont{theorem title}%
	\usebeamercolor[fg]{theorem title}%
	\inserttheoremname%~\inserttheoremnumber%
	\ifx\inserttheoremaddition\empty\else\ (\inserttheoremaddition)\fi%
	%\inserttheorempunctuation\hspace{1ex}%
	\\[\smallskipamount]
	\usebeamerfont{theorem body}%
	\usebeamercolor[fg]{theorem body}%
}
\setbeamertemplate{theorem end}{}

\setbeamercolor{proof title}{fg=SpotColor}
\setbeamerfont {proof title}{parent=structure}
\setbeamercolor{proof body} {fg=black}
\setbeamerfont {proof body} {parent=normal text, shape=\upshape, series=\mdseries}
% Remove period after ``Proof''
% See https://tex.stackexchange.com/questions/31354/how-to-remove-the-in-the-proof-environment
% -->
\makeatletter
\let\@addpunct\@gobble
\makeatother
% <--
\renewcommand{\qedsymbol}{\small\textcolor{SpotColor}{$\blacksquare$}}
\setbeamertemplate{proof begin}{%
	\usebeamerfont{proof title}%
	\usebeamercolor[fg]{theorem title}%
	\insertproofname\\[\smallskipamount]
	\usebeamerfont{proof body}%
	\usebeamercolor[fg]{proof body}%
}
\setbeamertemplate{proof end}{}

% <==




%%%%%%%%%%%%%
%%  LISTS  %%
%%%%%%%%%%%%%


\AtBeginEnvironment{itemize}{%
	\setlength{\partopsep}{-\smallskipamount}%
	\setlength{\labelsep}{0.55em}%
	\setlength{\labelwidth}{\widthof{9.}}%
}
\AtBeginEnvironment{enumerate}{%
	\setlength{\partopsep}{-\smallskipamount}%
	\setlength{\labelsep}{0.5em}%
	\setlength{\labelwidth}{\widthof{9.}}%
}

% Adjust spacing of lists ==>
% Unfortunately, the ``enumitem'' package is incompatible with the ``beamer'' class.
% Hence, we need to adjust left margins etc. manually:
\setlength{\leftmargini}{\widthof{9.}+\labelsep}
\setlength{\leftmarginii}{\widthof{g.}+\labelsep}
\setlength{\leftmarginiii}{\widthof{iii.}+\labelsep}
%\setlength{\rightmargin}{0in}
%\setlength{\itemindent}{0in}
%\usepackage{enumitem}
%\setlist[enumerate, 1]{
%	leftmargin=-\parindent, listparindent=\parindent, labelsep=0.42\parindent, itemsep=\smallskipamount, parsep=0pt
%}
%
%\setlist[itemize, 1]{
%	leftmargin=0pt, listparindent=\parindent, labelsep=0.75em, itemsep=\smallskipamount, parsep=0pt, topsep=1.2ex, label=\usebeamerfont*{itemize item}	\usebeamercolor[fg]{itemize item} \usebeamertemplate{itemize item}, before={\RaggedRight \hyphenpenalty=1000}
%}
%\setlist[itemize, 2]{
%	leftmargin=1.2em, listparindent=\parindent, labelsep=0.6em, itemsep=\smallskipamount, parsep=0pt, topsep=0.6ex, label=\usebeamercolor[fg]{itemize item} --, before={\RaggedRight \hyphenpenalty=1000}
%}
% <==

% ``Medium'' as the bold series:
%\makeatletter
%\def\bfseries@sf{mb}
%\makeatother

%\useinnertheme{circles}

\setbeamertemplate{enumerate items}
[default]
\setbeamertemplate{enumerate subitem}
{\alph{enumii}.}
\setbeamertemplate{enumerate subsubitem}
{\roman{enumiii}.}

\setbeamertemplate{itemize item}
{\raisebox{-0.5pt}{\scalebox{0.95}{\textrm{\textbf{\textbullet}}}}\hspace{-0.06em}}
\setbeamertemplate{itemize subitem}
{\usebeamerfont{normal text}--\hspace{0.15em}}
\setbeamertemplate{itemize subsubitem}
{\raisebox{1.5pt}{\tiny$\blacktriangleright$}\hspace{0.15em}}

\setbeamerfont*{itemize/enumerate body}{size=\normalsize}
\setbeamerfont*{itemize/enumerate subbody}{parent=itemize/enumerate body}
\setbeamerfont*{itemize/enumerate subsubbody}{parent=itemize/enumerate body}

\renewenvironment{quote}{%
	\list{}{\leftmargin\leftmarginii \rightmargin\leftmarginii}%
	\RaggedLeft
	\itshape%
	\item\relax%
}
{\endlist}




%%%%%%%%%%%%%%%%%%%%%%%%%%
%%  FIGURES AND TABLES  %%
%%%%%%%%%%%%%%%%%%%%%%%%%%


\setbeamertemplate{caption}[numbered]
\setbeamertemplate{caption label separator}{. }
\usepackage[justification=centering]{caption}
\ifnum \serifbodyfont=0
	\captionsetup{font={sf, small}}
\else
	\captionsetup{font={rm, small}}
\fi
\ifnum \serifheadingfont=0
	\captionsetup{labelfont={sf, small, bf, color=SpotColor}}
\else
	\captionsetup{labelfont={rm, small, bf, color=SpotColor}}
\fi

\usepackage{tabularx}	% Provides environment tabularx to adjust width of tables
% \usepackage{tabulary}	% For some reason, tabulary doesn't obey the width argument ...
% Emulate the "tabulary" column types:
\newcolumntype{C}{>{\centering\arraybackslash}X}
\newcolumntype{J}{>{\arraybackslash}X}
\newcolumntype{L}{>{\RaggedRight\arraybackslash}X}
\newcolumntype{R}{>{\RaggedLeft\arraybackslash}X}
% \usepackage[flushleft]{threeparttable}	% Provides the tablenotes environment


% Packages for creating better-looking tables
\usepackage{booktabs}
\setlength{\cmidrulewidth}{.35pt}
\setlength{\lightrulewidth}{.35pt}
\setlength{\heavyrulewidth}{.70pt}
\setlength{\abovetopsep}{-8pt}
\addtolength{\aboverulesep}{1pt}	% Make tables a little more spacious
\addtolength{\belowrulesep}{1pt}
% \setlength{\belowbottomsep}{-2pt}
\usepackage{colortbl} % e.g., for colored rules
\arrayrulecolor{SpotColor} % Color all table rules blue

\usepackage{siunitx}
% Allows, among others, for alignment of decimal numbers in tables at the decimal point.
\sisetup{
	detect-all,
	round-integer-to-decimal = true,
	group-digits             = true,
	group-minimum-digits     = 5,
	group-separator          = {\kern 1pt},
	table-align-text-pre     = false,
	table-align-text-post    = false,
	input-signs              = + -,
	input-symbols            = {*} {**} {***} \sigstar,
	input-open-uncertainty 	 = ,
	input-close-uncertainty  = ,
	retain-explicit-plus
}
\newcolumntype{T}[1]
{@{}S[table-format = #1, table-space-text-pre = {***}, table-space-text-post = {***}]}
% Loading the ``siunitx'' package generates a ``too many math alphabets'' used error.
% See https://tex.stackexchange.com/questions/452682/limit-math-alphabets-created-by-siunitx?noredirect=1&lq=1 and
%https://tex.stackexchange.com/questions/40874/kpfonts-siunitx-and-math-alphabets
% Hence, redefine the (virtually) never used \mathtt command:
\renewcommand{\mathtt}[1]{#1}
% From https://tex.stackexchange.com/questions/213605/siunitx-does-not-detect-semi-bold-font
% ==>
\ExplSyntaxOn\makeatletter
\cs_generate_variant:Nn \tl_if_in:nnTF { noTF }
\cs_set_protected:Npn \__siunitx_detect_font_weight_text: {
	\tl_if_in:noTF { sb bf xb } { \f@series }
	{
		\cs_set:Nn \__siunitx_font_weight: { \boldmath \use:c { \f@series series } }
	}
	{
		\cs_set:Nn \__siunitx_font_weight: { \use:c { \f@series series } }
	}
}
\makeatother\ExplSyntaxOff
% <==

\AtBeginEnvironment{table}{\vspace{-1.2ex}\small}
% Reduces font size in tables slightly and
% reduces excessive white space above tables




%%%%%%%%%%%%%%%%%%%%%%%%%%%%%
%%  BIBLIOGRAPHY SETTINGS  %%
%%%%%%%%%%%%%%%%%%%%%%%%%%%%%


%\usepackage[round]{natbib}

\newlength{\bibindent}
\newlength{\bibitemsep}
\newlength{\bibsep}

\setbeamertemplate{bibliography item}{}
\setbeamercolor{bibliography entry author}  {fg=black}
\setbeamercolor{bibliography entry title}   {fg=black}
\setbeamercolor{bibliography entry location}{fg=black}
\setbeamercolor{bibliography entry note}    {fg=black}

% !TeX program = pdflatex
% !TeX TXS-program:compile = txs:///pdflatex/
% !TeX TS-program = pdflatex
% !BIB program = biber
% !TeX TXS-program:bibliography = txs:///biber




%%%%%%%%%%%%%%%%%%%%%%%%%%%%%%%%%%%%%%%%%%%%%%%%
%%  CITATION COMMANDS AND BIBLIOGRAPHY STYLE  %%
%%%%%%%%%%%%%%%%%%%%%%%%%%%%%%%%%%%%%%%%%%%%%%%%


% AER/JEL/JEP style

\usepackage[backend=biber, natbib=true, bibencoding=inputenc, bibstyle=authoryear, citestyle=authoryear-comp, mincitenames=1, maxcitenames=3, minbibnames=99, maxbibnames=99, uniquename=false, uniquelist=true, backref=true, backrefstyle=three, doi=true, isbn=false, dashed=false, sorting=ynt, sortcites=true, mergedate=true, dateabbrev=false, abbreviate=false, citetracker=true]{biblatex}
% sortcites sorts the in-text citations by year of publication
\DeclareBibliographyAlias{newspaper}{article}

% Full author list on first citation, ``et al.'' only from second citation onwards:
% https://tex.stackexchange.com/questions/48846/biblatex-et-al-beginning-from-second-citation
% ==>
%\AtEveryCitekey{\ifciteseen{}{\clearfield{namehash}}}  % Uncomment this ``so that an entry won't be part of a compact citation list the first time it is cited.''
\xpatchbibmacro{cite}
	{\printnames{labelname}}
	{\ifciteseen
		{\printnames{labelname}}
		{\printnames[][1-5]{labelname}}%
	}
	{}
	{}
\xpatchbibmacro{textcite}
	{\printnames{labelname}}
	{\ifciteseen
		{\printnames{labelname}}
		{\printnames[][1-5]{labelname}}%
	}
	{}
	{}
% <==

\let\citeorig\cite
\renewcommand{\cite}{\citet}
\renewcommand{\citealp}{\citeorig}

\defbibheading{subbibliography}[\refname]{%
	\section*{#1}%
	\sectionmark{#1}%
	\addcontentsline{toc}{section}{#1}%
}

\renewcommand{\bibfont}{\sffamily\small}
	% Reduce font size for the bibliography, make the font sans-serif.
\setlength{\bibindent}{\parindent}
\setlength{\bibitemsep}{0pt}

\DefineBibliographyStrings{english}{%
	andothers = {et~al\adddot},
	volume = {vol\adddot}
}

% Remove parentheses from the date:
\xpatchbibmacro{date+extradate}{%
	\printtext[parens]%
}{%
	\setunit*{\mkbibbold{\addperiod\space}}%
	\mdseries\selectfont\printtext%
}{}{}
% Make the author/editor name(s) bold ==>
% See https://tex.stackexchange.com/questions/41468/biblatex-textcite-author-formatting-vs-bibliography-author-formatting
\xpatchbibmacro{author}{\printnames{author}}{\mkbibbold{\printnames{author}\addperiod}}{}{}
%\xpretobibmacro{author}{\mkbibbold\bgroup}{}{}
%\xapptobibmacro{author}{\egroup}{}{}
\xpretobibmacro{editor+others}{\mkbibbold\bgroup}{}{}%
\xapptobibmacro{editor+others}{\egroup\clearname{editor}}{}{}%
\xpretobibmacro{translator+others}{\mkbibbold\bgroup}{}{}%
\xapptobibmacro{translator+others}{\egroup\clearname{translator}}{}{}
% Set up page range compression (e.g., 1034-1067 => 1034-67)
\setcounter{mincomprange}{100}
\setcounter{maxcomprange}{100000}
\setcounter{mincompwidth}{10}
\DeclareFieldFormat{pages}{\mkcomprange{#1}}
% Removes ``pp.'' from pages and compresses page ranges:
\DeclareFieldFormat[article, inbook, incollection, inproceedings, patent, thesis, unpublished, newspaper]{pages}{{\nopp\mkcomprange{#1}}} 
% Include comma also after first name in the case of only two authors:
\AtEveryBibitem{
	\renewcommand*{\finalnamedelim}{\finalandcomma\addspace\bibstring{and}\space}
	\renewcommand*{\finallistdelim}{\finalandcomma\addspace\bibstring{and}\space}
}
\xpretobibmacro{byeditor+others}  % Undo this for the listing of editors
	{%
		\renewcommand*{\finalnamedelim}{%
			\ifnumgreater{\value{liststop}}{2}{\finalandcomma}{}%
			\addspace\bibstring{and}\space%
		}%
	}
	{}
	{\DoNotContinue}

% \renewcommand*{\newunitpunct}{\addcomma\space}
\renewcommand*{\bibnamedash}{---{\kern -2.25pt}---{\kern -2.25pt}---\addcomma\space}

%\DeclareCiteCommand{\citejournal}
%  {\usebibmacro{prenote}}
%  {\usebibmacro{citeindex}%
%   \usebibmacro{journal}}
%  {\multicitedelim}
%  {\usebibmacro{postnote}}

% Remove ``in'' for journal and newspaper articles:
\renewcommand*{\intitlepunct}{\space}
\renewbibmacro{in:}{%
	\ifentrytype{article}%
		{}%
		{\ifentrytype{newspaper}%
			{}%
			{\printtext{\bibstring{in}\intitlepunct}}%
		}%
}

% How to handle DOIs and URLs
% ==>
\renewbibmacro*{doi+eprint+url}
{%
	\iffieldundef{doi}
	{}
	{\printfield{doi}}
	\newunit\newblock
	\iftoggle{bbx:eprint}
	{\usebibmacro{eprint}}
	{}%
	\newunit\newblock
	\iffieldundef{doi}  % If no DOI provided, use URL
	{\usebibmacro{url+urldate}}
	{}
}
\DeclareFieldFormat{url}{%
	% \setlength{\Urlmuskip}{0mu}%
	\Urlmuskip = 0mu\relax%
	\mathchardef\UrlBigBreakPenalty=1\relax%
	\mathchardef\UrlBreakPenalty=2\relax%
		% See https://tex.stackexchange.com/questions/22854/url-line-breaks-with-biblatex
	\caps{URL}\addcolon\space\url{#1}%
}
% Link DOIs automatically to https://doi.org/<DOI>:
\DeclareFieldFormat{doi}{%
	% \setlength{\Urlmuskip}{0pt}%
	\Urlmuskip = 0mu\relax
	\mathchardef\UrlBigBreakPenalty=1\relax%
	\mathchardef\UrlBreakPenalty=2\relax%
		% See https://tex.stackexchange.com/questions/22854/url-line-breaks-with-biblatex
	\caps{DOI}\addcolon\space\href{https://doi.org/#1}{\nolinkurl{#1}}%
}
% <==

% Make journal title italic:
\DeclareFieldFormat[article]{journaltitle}{%
	\mkbibemph{#1}%
	\iffieldundef{volume}{\addcomma}{}%
}

% Handling of newspaper articles
\DeclareFieldFormat[newspaper]{journaltitle}{%
	\mkbibemph{#1} \mkbibparens{\thefield{edition}}\addcomma\addspace%
	\iflanguage{ngerman}{%
		\addnbspace\thefield{day}\addperiod\addnbspace\mkbibmonth{\thefield{month}}\addspace\thefield{year}%
	}{% else: default to the USenglish version
		\mkbibmonth{\thefield{month}}\addnbspace\thefield{day}\addcomma\addspace\thefield{year}%
	}%
	\iffieldundef{volume}{\addcolon}{\addcomma}%
}
\DeclareFieldFormat[newspaper]{date}{%
	\thefield{year}%
}

% Format the ``month'' field:
\DeclareFieldFormat{month}{\mkbibmonth{#1}}

% Put the title of articles in quotation marks:
\DeclareFieldFormat[thesis, unpublished, report, misc, newspaper]{title}{\mkbibquote{#1}}
%% Make title of books italic:
%\DeclareFieldFormat[book]{title}{\mkbibemph{#1}\nopunct}

% Makes first character of document type uppercase:
\DeclareFieldFormat[thesis, unpublished, report, misc]{type}{\MakeCapital{#1}}

%% Remove punctuation after ``series'' field:
%\DeclareFieldFormat[incollection]{series}{#1\nopunct}

% Add ``vol.'' for books etc.:
\DeclareFieldFormat[book, inbook, incollection, inproceedings]{volume}{\bibstring{volume}\addnbspace#1\addcomma}
% Formatting VV\,(NN):
\DeclareFieldFormat[article]{volume}{%
	\iffieldundef{volume}{}{%
		\addspace#1%
		\iffieldundef{number}{\addcolon}{\nopunct}%
	}
}
\DeclareFieldFormat[article]{number}{\addnbthinspace\mkbibparens{#1}\addcolon}
\DeclareFieldFormat[article]{date}{#1}

% \AtEveryCitekey{\clearfield{month}}
% Replace number by month if number is undefined:
% ==>
\DeclareSourcemap{
    \maps[datatype=bibtex]{
        \map{
            \pertype{article}
			\step[notfield=number, final]
			\step[fieldsource=month, fieldtarget=number]
            \step[fieldset=month, null]
            \step[fieldset=month_numeric, null]
        }
    }
}
% <==

% Adjust formatting of back-references
% ==>
\DefineBibliographyStrings{english}{%
	backrefpage  = {},	% originally ``cit. on p.''
	backrefpages = {}	% originally ``cit. on pp.''
}
\DefineBibliographyStrings{ngerman}{%
	backrefpage  = {},	% originally ``Siehe Seite''
	backrefpages = {}	% originally ``Siehe Seiten''
}
\renewcommand*{\finentrypunct}{}
\renewbibmacro*{pageref}{%
	\addperiod
	\iflistundef{pageref}
	{}
	{\printtext[brackets]{%
			\ifnumgreater{\value{pageref}}{1}
				{\bibstring{backrefpages}}
				{\bibstring{backrefpage}}%
			\printlist[pageref][-\value{listtotal}]{pageref}%
		}%
	}%
}
% <==

% Move notes to the end of an entry by copying the ``note'' information into ``addendum'': -->
\DeclareSourcemap{
	\maps[datatype=bibtex]{
		% Copy values of the Mendeley-created ``annote'' field to the ``note'' field:
		\map[overwrite]{
		 	\step[fieldsource=annote]
		 	\step[fieldset=note, origfieldval, append]
			\step[fieldset=annote, null]
		}
		\map{
			\step[fieldsource=note, final]
			\step[fieldset=addendum, origfieldval, final]
			\step[fieldset=note, null]
		}
	}
}
% \DeclareFieldFormat{addendum}{(#1)} % Enclose addendum/note in parentheses.
% <--

\AtEveryBibitem{%
	\ifentrytype{newspaper}%
		{\clearfield{addendum}}% then
		{\clearfield{month}}% else
}

% Add ``working paper'' as the default value for type of ``techreports'' ==>
% see https://tex.stackexchange.com/questions/212362/how-to-use-declaresourcemap-to-add-default-value-to-a-field
\DeclareSourcemap{
	\maps[datatype=bibtex]{
		\map{% Will overwrite fields without the ``overwrite'' option
			\pertype{techreport}
			\step[fieldset=type, fieldvalue={Working paper}]
		}
	}
}
% <==

% Add “doctoral dissertation” as the default value for type of “phdthesis” ==>
% see https://tex.stackexchange.com/questions/212362/how-to-use-declaresourcemap-to-add-default-value-to-a-field
\DeclareSourcemap{
	\maps[datatype=bibtex]{
		\map{% Will overwrite fields without the ``overwrite'' option
			\pertype{phdthesis}
			\step[fieldset=type, fieldvalue={Doctoral dissertation}]
		}
	}
}
% <==

% Remove superfluous ``The'' from journal names ==>
\DeclareSourcemap{ 
	\maps[datatype=bibtex]{
		\map{
			\step[fieldsource=journal, match={\regexp{^The\s}}, replace={}]
		}
	}      
}
% <==

% Replace (Mendeley's) month strings (``jan'', ``feb'', etc.) by full names ==>
\DeclareSourcemap{ 
	\maps[datatype=bibtex]{
		\map{
			\step[fieldsource=number, match={\regexp{^jan$}}, replace=\regexp{\\mkbibmonth\{1\}}]
			\step[fieldsource=number, match={\regexp{^feb$}}, replace=\regexp{\\mkbibmonth\{2\}}]
			\step[fieldsource=number, match={\regexp{^mar$}}, replace=\regexp{\\mkbibmonth\{3\}}]
			\step[fieldsource=number, match={\regexp{^apr$}}, replace=\regexp{\\mkbibmonth\{4\}}]
			\step[fieldsource=number, match={\regexp{^may$}}, replace=\regexp{\\mkbibmonth\{5\}}]
			\step[fieldsource=number, match={\regexp{^jun$}}, replace=\regexp{\\mkbibmonth\{6\}}]
			\step[fieldsource=number, match={\regexp{^jul$}}, replace=\regexp{\\mkbibmonth\{7\}}]
			\step[fieldsource=number, match={\regexp{^aug$}}, replace=\regexp{\\mkbibmonth\{8\}}]
			\step[fieldsource=number, match={\regexp{^sep$}}, replace=\regexp{\\mkbibmonth\{9\}}]
			\step[fieldsource=number, match={\regexp{^oct$}}, replace=\regexp{\\mkbibmonth\{10\}}]
			\step[fieldsource=number, match={\regexp{^nov$}}, replace=\regexp{\\mkbibmonth\{11\}}]
			\step[fieldsource=number, match={\regexp{^dec$}}, replace=\regexp{\\mkbibmonth\{12\}}]
		}
	}      
}
% <==

%% !TeX program = pdflatex
% !TeX TXS-program:compile = txs:///pdflatex/
% !TeX TS-program = pdflatex
% !BIB program = biber
% !TeX TXS-program:bibliography = txs:///biber




%%%%%%%%%%%%%%%%%%%%%%%%%%%%%%%%%%%%%%%%%%%%%%%
%%  CITATION COMMANDS AND BIBLIOGRAPHY STYLE %%
%%%%%%%%%%%%%%%%%%%%%%%%%%%%%%%%%%%%%%%%%%%%%%%


% For APA:
\usepackage[style=apa, backend=biber, natbib=true, backref=true]{biblatex}

\let\citeorig\cite
\renewcommand{\cite}{\citet}
\renewcommand{\citealp}{\citeorig}

\renewcommand{\bibfont}{\sffamily\small}
	% Reduce font size for the bibliography, make the font sans-serif.
\AtBeginDocument{%
	\setlength{\bibindent}{\parindent}%
}
\setlength{\bibitemsep}{0pt}

\DefineBibliographyStrings{english}{%
	backrefpage = {},	% originally ``cit. on p.''
	backrefpages = {}	% originally ``cit. on pp.''
}
\DefineBibliographyStrings{ngerman}{%
	backrefpage  = {},	% originally ``Siehe Seite''
	backrefpages = {}	% originally ``Siehe Seiten''
}

\renewcommand*{\finentrypunct}{}
\renewbibmacro*{pageref}{%
	\addperiod
	\iflistundef{pageref}
	{}
	{\printtext[brackets]{% NEW
			\ifnumgreater{\value{pageref}}{1}
			{\bibstring{backrefpages}}
			{\bibstring{backrefpage}}%
			\printlist[pageref][-\value{listtotal}]{pageref}%
		}%
	}%
}% NEW

% Link DOIs automatically to https://doi.org/<DOI>, in line with the
% APA's DOI Display Guidelines Update from March 2017:
% https://blog.apastyle.org/apastyle/digital-object-identifier-doi/
% ==>
\DeclareFieldFormat{doi}{%
	\Urlmuskip = 0mu\relax%
	\mathchardef\UrlBigBreakPenalty=1\relax%
	\mathchardef\UrlBreakPenalty=2\relax%
		% See https://tex.stackexchange.com/questions/22854/url-line-breaks-with-biblatex
	\url{https://doi.org/#1}%
}
% <==
\DeclareFieldFormat{url}{%
	\Urlmuskip = 0mu\relax%
	\mathchardef\UrlBigBreakPenalty=1\relax%
	\mathchardef\UrlBreakPenalty=2\relax%
		% See https://tex.stackexchange.com/questions/22854/url-line-breaks-with-biblatex
	\url{#1}%
}

\bibliography{Library}

\renewcommand{\bibfont}{\scriptsize}

\renewbibmacro*{pageref}{%
	\addperiod
	\iflistundef{pageref}
	{}
	{\color{UBonnGray}%
		\printtext[brackets]{\caps{PDF}~%
			\ifnumgreater{\value{pageref}}{1}
			{pp.}
			{p.}~%
			\printlist[pageref][-\value{listtotal}]{pageref}%
		}%
	}%
}

\setcounter{biburllcpenalty}{7000}
\setcounter{biburlucpenalty}{8000}




%%%%%%%%%%%%%%%%%%%%%%%%%%%%%%%%%
%%  APPENDIX-RELATED SETTINGS  %%
%%%%%%%%%%%%%%%%%%%%%%%%%%%%%%%%%


\usepackage{appendixnumberbeamer}
\AtBeginEnvironment{appendix}{%
	\let \insertframenumberorig \insertframenumber
	\renewcommand{\insertframenumber}{Appendix Slide \insertframenumberorig}
	%\let \inserttotalframenumberorig \inserttotalframenumber
	%\renewcommand{\inserttotalframenumber}{A-\inserttotalframenumberorig}
}




%%%%%%%%%%%%%%%%%%%%%%%%%%
%%  FOR DEBUGGING ONLY  %%
%%%%%%%%%%%%%%%%%%%%%%%%%%


\usepackage{blindtext}
\blindmathtrue

% An auxiliary command to display the current font settings ==>
\makeatletter
\newcommand{\showfont}{%
	\textit{encoding:} \f@encoding{},
	\textit{family:} \f@family{},
	\textit{series:} \f@series{},
	\textit{shape:} \f@shape{},
	\textit{size:} \f@size{}
}
\makeatother
% <==
