\section{Conclusion}
\label{sec:conclusion}

We propose a simulation based model of infectious disease transmission that is designed
to predict the effects of fine-grained social distancing policies. In particular, the
model can be used to model policies such as several ways of splitting school classes or
work reduction policies that affect essential and non-essential workers differently.
Both policies would be hard to implement in standard SEIR or agent based simulation
models.

To predict the effects of such policies, it is not only important to have a way of
expressing such flexible policies in terms of model quantities, but also to incorporate
heterogeneity in disease progression as well as meeting patterns. We calibrate age
dependent disease progression parameters from the medical literature and age dependent
contact frequencies from contact diaries. Moreover, we distinguish ten types of contacts
out of which some are only relevant for certain age groups.

The model has a good fit on past German case numbers and passes an out of sample
validation despite a drastic change in the policy environment between the estimation
period and the validation period.

Despite these encouraging results we still see the model as work in progress and plan to
implement more features such as a detailed model of testing and contact tracing.
Moreover, the estimation of the infection probabilities and the model fit will improve
as more data becomes available.

W invite researchers from any discipline, but particularly epidemiologists to provide
feedback on the model and welcome collaborations.
