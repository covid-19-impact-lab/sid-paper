\noindent
In order to slow the spread of the CoViD-19 pandemic, governments around the world have
enacted a wide set of policies limiting the transmission of the disease. Initially,
these focused on non-pharmaceutical interventions; more recently, vaccinations and
large-scale rapid testing have started to play a major role. The objective of this study
is to explain the quantitative effects of these policies on determining the course of
the pandemic, allowing for factors like seasonality or virus strains with different
transmission profiles. To do so, the study develops an agent-based simulation model,
which explicitly takes into account test demand and behavioral changes following
positive tests. The model is estimated using data for the second and the third wave of
the CoViD-19 pandemic in Germany. The paper finds that during a period where vaccination
rates rose from 5\% to 40\%, seasonality and rapid testing had the largest effect on
reducing infection numbers. Frequent large-scale rapid testing should remain part of
strategies to contain CoViD-19; it can substitute for many non-pharmaceutical
interventions that come at a much larger cost to individuals, society, and the economy.

\vspace{1cm}
\noindent \textbf{JEL Classification:} C63, I18

\noindent \textbf{Keywords:} CoViD-19, agent based simulation model, rapid testing,
non-pharmaceutical interventions