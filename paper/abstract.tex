\noindent
Governments worldwide have been adopting various and nuanced policy measures
to reduce the number of Covid-19 cases.
However, epidemiological models usually lack sufficient detail in meeting
patterns to credibly predict the effects such policies, much less predict
complementarities or substitution effects between them.
We propose a novel simulation-based model to address these shortcomings.
We build on state-of-the-art agent-based simulation models, greatly increasing
the amount of detail and realism with which contacts take place.
Firstly, we allow for heterogeneity in the types of contacts (e.g. recurrent and
non-recurrent) and in the infectiousness of each contact type.
Secondly, we strictly separate the number of contacts from the probabilities
that a contact leads to an infection.
Thus, social distancing policies can affect the number of contacts while the
infection probabilities remain invariant.
This allows us to model many types of targeted policies that cannot be
incorporated into other models.
\textcolor{red}{
To validate our model, we show that it can predict the effect of the German
November lockdown even if no similar policy has been observed during the time
period that was used to estimate the model parameters.
}

\vspace{1cm}
\noindent JEL Classification: C63, I18

\noindent Keywords: Covid-19, agent based simulation model, public health measures
