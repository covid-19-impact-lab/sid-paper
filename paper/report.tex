% Reports (up to ~2500 words including references, notes and captions–corresponds to ~3
% printed pages in the journal) present important new research results of broad
% significance. Reports should include an abstract, an introductory paragraph, up to
% four figures or tables, and about 30 references. Materials and Methods should be
% included in supplementary materials, which should also include information needed to
% support the paper's conclusions.

% Main Text is not divided into sub-headings for Reports.

% The manuscript should start with a brief introduction describing the paper’s
% significance. The introduction should provide sufficient background information to
% make the article intelligible to readers in other disciplines, and sufficient context
% that the significance of the experimental findings is clear.

% \clearpage

Since early 2020, the CoViD-19 pandemic has presented an enormous challenge to humanity
on many dimensions. The development of highly effective vaccines holds the promise of
containment in the medium term. However, most countries find themselves many months---and
often years---away from reaching vaccination-induced herd immunity
\citep{Swaminathan2021}.\comment[id=K]{WHO Press Conference saying we won't reach herd
immunity in 2021} In the meantime, it is of utmost importance to employ an effective mix
of strategies for containing the virus. The most frequent initial response was a set of
non-pharmaceutical interventions (NPIs) to reduce contacts between individuals. While
this has allowed some countries to sustain equilibria with very low infection
numbers\footnote{See \citet{Contreras2021} for a theoretical equilibrium at low case
numbers which is sustained with test-trace-and-isolate policies.}, most have seen large
fluctuations of infection rates. Containment measures have become increasingly diverse
and now include testing, more nuanced NPIs, and contact tracing. Neither these policies'
effect nor the influence of seasonal patterns or more infectious virus strains are well
understood in quantitative terms. This paper develops a model incorporating all these
factors. The framework allows to combine a wide variety of data in a timely fashion,
making it useful to predict the effects of various interventions. We apply the model to
Germany and show that rapid testing had the largest impact on the reduction in infections
by almost 80\% during the month of May 2021. We conclude that rapid tests have a large
role to play at least as long as vaccinations have not been offered to an entire
population.

At the core of our agent-based model are physical contacts between heterogeneous agents
(Figure~\ref{fig:broad_model_description}).\footnote{We provide a detailed comparison to
    other approaches in \ref{sec:literature_review}. The model most closely related to ours
    is described in \citet{Hinch2020}.} Each contact between an infectious individual and
somebody susceptible to the disease bears the risk of transmitting the virus. Contacts
occur in the household, at work, at school, or in other settings (leisure activities,
grocery shopping, medical appointments, etc.). Some contacts recur regularly, others
occur at random. Random contacts are typically assortative in age and geographical
location. Empirical applications can take the population structure from census data and
the types and frequency of contacts from diary data measuring contacts before the
pandemic \citep[e.g.][]{Mossong2008}.\footnote{\citet{Hoang2019} provide access to
    multiple data sets on contact types and frequencies at
    \url{http://www.socialcontactdata.org/} covering countries from all continents except
    North America and Australia.} The dimensions are chosen so that the most common NPIs can
be modeled in great detail by reducing the number of contacts in a particular setting or
the risk of transmitting the disease for a type of contact. For example, a mandate to
work from home will reduce the number of work contacts to zero for a fraction of the
working population. Schools and daycare can be closed entirely, operate at reduced
capacity---including an alternating schedule---,or implement mitigation measures like
masking requirements or air filters \citep{Lessler2021}. Curfews may reduce the number
of contacts in non-work/non-school settings. In any setting, measures like masking
requirements would reduce the probability of infection associated with a contact.

\begin{figure}
    \centering

    \begin{subfigure}[b]{0.425\textwidth}
        \centering
        \includegraphics[width=\textwidth]{../figures/model-graph-top-left}
        \caption{{\small Model description}}
        \label{fig:broad_model_description}
    \end{subfigure}
    \hfill
    \begin{subfigure}[b]{0.425\textwidth}
        \centering
        \includegraphics[width=\textwidth]{../figures/model-graph-top-right}
        \vskip4ex


        \caption{Disease progression}
        \label{fig:disease_progression}
    \end{subfigure}
    \vskip3ex
    \begin{subfigure}[b]{0.425\textwidth}
        \centering

        \includegraphics[width=\textwidth]{../figures/model-graph-bottom-left}
        \caption{{\small Model for antigen tests}}
        \label{fig:antigen_tests}
    \end{subfigure}
    \hfill
    \begin{subfigure}[b]{0.425\textwidth}
        \centering
        \includegraphics[width=\textwidth]{../figures/model-graph-bottom-right}
        \caption{{\small Model for undetected cases}}
        \label{fig:model_for_official_cases}
    \end{subfigure}

    \caption{Model description}
    \label{fig:model-description}

    \floatfoot{\noindent Note: ...}
\end{figure}

In our model, susceptibility to contracting the SARS-CoV-2 virus is dependent on age. A
possible infection progresses as shown in Figure~\ref{fig:disease_progression}. We
differentiate between an initial period of infection without being infectious or showing
symptoms, being infectious (presymptomatic or asymptomatic), showing symptoms, requiring
intensive care, and recovery or death as for example also modeled in \cite{Grimm2021}.
The probabilities of transitioning between these states depend on age; their duration is
random within intervals calibrated to medical literature (for a detailed description see
Section~\ref{sec:medical_params}). Conditional on the type of contact, infectiousness is
independent of age \citep{Jones2021}.

The model includes several other features, which are crucial to describe the evolution
of the pandemic in 2020-2021. New virus strains with different profiles regarding
infectiousness can be introduced. Agents may receive a vaccination. With a probability
of 75\% \citep{Hunter2021}, vaccinated agents become immune and they do not transmit the
virus \citep{Petter2021, LevineTiefenbrun2021, Pritchard2021}.\footnote{75\% is lower
    than what is usually reported for after the second dose of the Biontech/Pfizer vaccine,
    which is most commonly used in Germany. We choose it because our model neither includes
    booster shots, nor does it allow vaccinated individuals who became immune to transmit
    the disease\citep{Petter2021, LevineTiefenbrun2021, Pritchard2021}.} During the vaccine
roll-out, priority may depend on age and occupation.

We include two types of tests. Polymerase chain reaction (PCR) tests directly reveal
whether an individual is infected or not. PCR tests require some days to be processed
and there are always aggregate capacity constraints. In contrast, rapid antigen tests
yield immediate results. Specificity and sensitivity of these tests is set according to
data analysed in \cite{Bruemmer2021, Smith2021}; sensitivity depends on the timing of
the test relative to the start of infectiousness. Figure~\ref{fig:antigen_tests} shows
our model for rapid test demand. Schools may require students to be tested regularly.
Rapid tests may be offered by employers for on-site workers. Individuals may demand
tests for private reasons, which include having plans to meet other people\footnote{A
    positive test will make them reduce their contacts; this is why tests impact the actual
    contacts in Figure~\ref{fig:model-description}.}, showing symptoms of CoViD-19, and
because a household member tested positively for the virus. We endow each agent with an
individual compliance parameter. This parameter determines whether she takes up rapid
tests offered by employers or follows up on private reasons. The thresholds are lower for
tests in a private setting than for tests at the workplace.\comment[id=HM]{True? @Janos}

Modelling a population of agents according to actual demographic characteristics means
that we can use a wide array of data to identify and estimate the model's many
parameters.\footnote{See section S.XXX of the supplementary materials for an
    overview.} Mobility data is used to
model the reductions in work contacts. School and daycare policies are incorporated
directly from official directives. Infection rates by age and geographical region are
estimated to match to officially recorded numbers; so is the prevalence of virus
strains. The model yields total infections, but only a fraction of those will be
officially recorded. We model the fraction of known cases as depicted in
Figure~\ref{fig:model_for_official_cases}.\comment[id=HM]{Add a sentence.}
\footnote{EQUATIONS FOR THE FOURTH PICTURE (find names!):

    The probabilities for Figure~\subref{fig:model_for_official_cases} are the
    following:

    \begin{align*}
        x & = \frac{n\_newly\_infected \times share\_known\_cases \times P(PCR | S)}{n\_symptomatic\_and\_untested}                    \\
        y & = \frac{n\_newly\_infected \times share\_known\_cases \times P(PCR | S)}{n\_asymptoptomatic\_and\_infected\_and\_untested} \\
        z & = \frac{P(PCR | RT)}{survey}
    \end{align*}
    where untested means that individuals are not waiting on the result of a previously
    administered rapid test.} A further advantage is that the simulated data have a
structure that resembles datasets used for regression models, which allows
additional plausibility checks by re-running the same models on the model-generated
data.\comment[id=HM]{Only keep if we actually do something like that.}

We apply this model to Germany. In March and April 2020, the country broke the first
wave of the pandemic fairly quickly. Between May and mid-September, daily new infections
were below 20 per Million and day \citep{owidcoronavirus}. We model the period
mid-September 2020 to the end of May 2021. We pick the starting date for two reasons.
First, we do not include the first wave because the environment was very different
(e.g., aggregate PCR test capacity was much lower and we would require a very different
model for calculating the share of known cases) Second, a large fraction cases during
summer of 2020 were traced to international travel \citep{KochInstitut2021,Hodcroft2021}
but the precise number is difficult to model.

Figure~\ref{fig:pandemic_drivers_model_fit} describes the evolution of the
pandemic and of its drivers. The black line in Figure~\ref{fig:aggregated_fit}
shows officially recorded cases; the black line in
Figure~\ref{fig:stringency_index} the Oxford Response Stringency Index
\citep{Hale2020}, which tracks the tightness of non-pharmaceutical
interventions. We transform the index so that lower values represent higher
levels of restrictions. A value of zero means all measures incorporated in the
index are turned on. The value 1 represents the situation in mid-September, with
restrictions on gatherings and public events, masking requirements, but open
schools and workplaces (the raw value of the index at that point is 49.5). In
the seven weeks between mid September and early November, cases increased by a
factor of 10. Restrictions were somewhat tightened in mid-October and again in
early November. New infections remained constant throughout November, before
rising again in December, which prompted the most stringent lockdown to this
date. Schools and daycare centers were closed again, so were customer-facing
businesses except for grocery and drug stores. From the peak of the second wave
just before Christmas until the trough in mid-February, newly detected cases
decreased by almost three quarters. The third wave in the spring of 2021 is
associated with the B.1.1.7 strain, which became dominant in March. See
Figure~\ref{fig:share_b117}. In early March, some NPIs were being relaxed; e.g.,
hairdressers and home improvement stores were allowed to open again to the
public.


\begin{figure}[!tp]
    \centering

    \begin{subfigure}[b]{0.475\textwidth}
        \centering
        \includegraphics[width=\textwidth]{../figures/results/figures/scenario_comparisons/combined_fit/full_new_known_case}
        \caption{{\small Recorded cases: Empirical and simulated}}
        \label{fig:aggregated_fit}
    \end{subfigure}
    \hfill
    \begin{subfigure}[b]{0.475\textwidth}
        \centering
        \includegraphics[width=\textwidth]{../figures/results/figures/data/stringency2_with_seasonality}

        \caption{{\small Stringency of NPIs and changes in infectious contacts by type}}
        \label{fig:stringency_index}
    \end{subfigure}

    \vskip3ex

    \begin{subfigure}[b]{0.475\textwidth}
        \centering

        \includegraphics[width=\textwidth]{../figures/results/figures/scenario_comparisons/combined_fit/full_share_rapid_test_in_last_week_and_vaccinated}

        \caption{{\small Tests and vaccinations}}
        \label{fig:antigen_tests_vaccinations}
    \end{subfigure}
    \hfill
    \begin{subfigure}[b]{0.475\textwidth}
        \centering

        \includegraphics[width=\textwidth]{../figures/results/figures/scenario_comparisons/combined_fit/full_share_b117}

        \caption{Fraction of B.1.1.7 strain among measured infections}
        \label{fig:share_b117}
    \end{subfigure}

    \caption{Evolution of the pandemic, its drivers, and model fit, September 2020 to May 2021}
    \label{fig:pandemic_drivers_model_fit}

    \floatfoot{\noindent Note: All aggregates; See S.XXX for statistics by age group and
        by geographical region. Also more disaggregated data.

        Sources: ...}

\end{figure}

By this time, the set of policy instruments had become much more diverse. Around the
turn of the year, the first people were vaccinated with a focus on older age groups and
medical personnel (Figure~\ref{fig:antigen_tests_vaccinations}. By the end of May, just
over 40\% had received at least one dose of a vaccine. Around the same time, rapid tests
started to replace PCR tests in many medical and nursing facilities. These had to be
administered by medical doctors or in pharmacies. At-home tests approved by authorities
became available in mid-March, rapid test centers were opened and one test per person
and week was made available free of charge. Depending on the state, customers were only
allowed to enter certain stores with a recent negative rapid test result. These
developments are characteristic of many countries: The initial focus on NPIs to slow the
spread of the disease has been accompanied by vaccines and a growing acceptance and use
of rapid tests. At broadly similar points in time, novel strains of the virus have
started to pose additional challenges.

Our model is able to track these developments very well. The blue line in
Figure~\ref{fig:aggregated_fit} shows our model's predictions are very close to
officially recorded cases in the aggregate. This is also true for infections by
age and geographical region, which are shown in the supplementary materials
(Figures~\ref{fig:age_group_fit} and \ref{fig:state_fit}, respectively). We can
disentangle various mechanisms due to the distinct temporal variation in the
drivers of the pandemic. Next to the stringency index, the three lines in
Figure~\ref{fig:stringency_index} summarize how contact reductions, increased
hygiene regulations, and seasonality evolved since early September for each of
the three broad contact networks. For example, a value of 0.75 for the work
multiplier means that if the environment was the same as in September (levels of
infection rates, no rapid tests or vaccinations, only the wildtype virus
present), infections at the workplace would be reduced by 25\%. This reduction
is the product of the effect of contact reductions, increased hygiene
regulations, and seasonality. Along with the levels of infections, these
measures determine the spread of SARS-CoV-2 in 2020. See
Appendix~\comment[id=HM]{Reference} for separate plots of the three factors. Two
aspects are particularly interesting. First, all lines broadly follow the
stringency index and they would do so even more if we left out seasonality and
school vacations (roughly the last two weeks of October, two weeks each around
Christmas and Easter, and some days in late May). Second, the most stringent
regulations are associated with the period of strong decreases in new infection
between late December 2020 and mid-February 2021. The measures were not enough,
however, to stop the B.1.1.7 variant from spreading in the subsequent period.
The steep drop in recorded cases during May 2021 is associated with at least
weekly rapid tests to around 42~percent of the population, a vaccination rate
that rose from 28\% to 43\%, and mostly seasonality impacting a fall in the
relative infectiousness of contacts outside of work and school.

Figure~\ref{fig:2021_scenarios_broad} consider the relative effects of rapid
tests, vaccinations, and of seasonality during 2021, assuming NPIs to have
evolved the same way as in the baseline scenario.
Figure~\ref{fig:2021_scenarios_recorded} shows the model fit (the blue line,
same as in Figure~\ref{fig:aggregated_fit}), a scenario without any of the three
factors (red line), and three scenarios turning these factors off one by one.
Figure~\ref{fig:2021_scenarios_newly_infected} does the same for total
infections in the model. Figure~\ref{fig:2021_scenarios_decomposition} employs
Shapley values to decompose the difference in total infections between the
scenario without any of the three factors and our main specification.

\begin{figure}[!tp]
    \centering

    \begin{subfigure}[b]{0.475\textwidth}
        \centering
        \includegraphics[width=\textwidth]{../figures/results/figures/scenario_comparisons/effect_of_channels_on_pessimistic_scenario/full_new_known_case}
        \caption{{\small Recorded cases: 2021 scenarios}}
        \label{fig:2021_scenarios_recorded}
    \end{subfigure}
    \hfill
    \begin{subfigure}[b]{0.475\textwidth}
        \centering
        \includegraphics[width=\textwidth]{../figures/results/figures/scenario_comparisons/effect_of_channels_on_pessimistic_scenario/full_newly_infected}
        \caption{{\small Total cases: 2021 scenarios}}
        \label{fig:2021_scenarios_newly_infected}
    \end{subfigure}

    \begin{subfigure}[b]{0.475\textwidth}
        \centering
        \includegraphics[width=\textwidth]{../figures/full_decomposition}

        \vskip2ex

        \caption{Decomposition of effects for
            Figure~\ref{fig:2021_scenarios_newly_infected}.}
        \label{fig:2021_scenarios_decomposition}
    \end{subfigure}
    \hfill
    \begin{subfigure}[b]{0.475\textwidth}
        \centering

        \includegraphics[width=\textwidth]{../figures/results/figures/scenario_comparisons/effect_of_rapid_tests/full_newly_infected}
        \caption{{\small Effects of different types of testing}}
        \label{fig:2021_scenarios_decomposition_tests}
    \end{subfigure}

    \caption{The effect of different interventions on recorded and actual infections}
    \label{fig:2021_scenarios_broad}

    \floatfoot{\noindent Note: All aggregates; See S.XXX for statistics by age group and
        by geographical region.

        The decomposition is based on Shapley values where the individual contribution
        of a channel is its average contribution over different sizes of coalitions
        (combinations with other channels). The individual contribution to a coalition
        is the difference between the effect size of the coalition with the particular
        channel and without.}
\end{figure}

Until mid-March, there is no visible difference between the different scenarios.
Seasonality hardly changes, and only few vaccinations or rapid tests were administered.
Even thereafter, the effect of the vaccination campaign is surprisingly small at first
sight. Whether considering recorded or total infections, the final level is always the
highest in case the vaccination campaign had been running in isolation (red lines). The
Shapley value decomposition shows that vaccinations contribute about
15\%\comment[id=HM]{check!} to the cumulative difference between scenarios. Reasons for
this are the slow start---it took until 24~March until 10\% of the population had
received their first vaccination, the 20\% mark was reached on April 19th ---and the
focus on older individuals. These groups contribute less to the spread of the disease
than others due to a lower number of contacts, see~\ref{fig:assortativity}. It is
important to note that the initial focus of the campaign was to prevent deaths and
severe disease; the case fatality was rate considerably lower during the third wave when
compared to the second (4.4\% between October and February and 1.4\% between March and
June). It is important to note that by the end of our study period, when first-dose
vaccination rates reached around 40\% of the population, the numbers of new cases would
have started to decline.

Seasonality has a large effect in slowing the spread of SARS-CoV-2. By May 31, both
observed and recorded cases would be reduced by a factor of four if only seasonality
mattered. However, in this period, cases would have kept on rising throughout, just at a
much lower pace. Nevertheless, we estimate it to be a quantitatively important factor
determining the evolution of the pandemic, explaining most of the early changes and
almost 40\% of the cumulative difference by the end of May.

The largest effect---almost one half when considering the decompositions---comes from
rapid testing. Here, it is crucial to differentiate between recorded cases and actual
cases. Additional testing means that infections become known which would otherwise
remain undetected. Figure~\ref{fig:2021_scenarios_recorded} shows that this means that
until late April, recorded cases are higher than in the scenario where none of the three
mechanisms is turned on. Compared to the scenario with vaccinations only, this point is
reached only around mid May and it would be June for the comparison with the
seasonality-only scenario. The effect on total cases, however, kicks in immediately and
strongly. Despite the fact that only a small fraction of the population performed weekly
rapid tests in March (X\%\comment[id=HM]{Put in}), the rise in new infections would be
limited by XXX\% relative to the scenario without vaccinations, tests, or seasonality.

Tests



\clearpage

Two particularly important areas:
\begin{itemize}
    \item Schooling: large costs to pupils, ... social inequalities
    \item Work: Q2 / 2020 saw the largest drop in GDP in a long time;
\end{itemize}

Results
\begin{itemize}
    \item Advantage of testing in both cases: Recurrent. Small cost relative to other
          stuff. Certain publicness.
    \item Mandatory tests: Screening effect: ...
\end{itemize}


\begin{figure}[!tp]
    \centering

    \begin{subfigure}[b]{0.475\textwidth}
        \centering
        \includegraphics[width=0.9 \textwidth]{../figures/results/figures/scenario_comparisons/school_scenarios/full_newly_infected}
        \caption{{\small Effects of different schooling scenarios after Easter}}
        \label{fig:schooling_scenarios_easter}
    \end{subfigure}
    \hfill
    \begin{subfigure}[b]{0.475\textwidth}
        \centering
        % \includegraphics[width=0.9
        % \textwidth]{../figures/results/figures/scenario_comparisons/}
        \caption{{\small Effects of different work scenarios after Easter}}
        \label{fig:to_be_determined}
    \end{subfigure}
    \vskip3ex

    \caption{Effects of different scenarios for schooling and working from home}
    \label{fig:interventions_school}

    \floatfoot{\noindent Note: All aggregates; See S.XXX for statistics by age group and
        by geographical region.}

\end{figure}




\paragraph{Points to mention}
\begin{itemize}
    \item If anything too optimistic regarding vaccinations
    \item Social structure / conditional block testing in families important (?)
    \item Trump-effect: More testing = more cases true for how long?
    \item Towards the end, total tests go down: Preferences for tests remain the same,
          but infection rates lower \comment[id=HM]{Janos, please fill up}
    \item Actually performed tests increase until May, stay constant during May and then start to drop
    \item This is the combination of the following things:
    \begin{itemize}
        \item time constant willingness to take tests
        \item monotonically increasing availability of tests
        \item endogenous number of situations in which a test should be taken (mainly positive tests of household members)
    \end{itemize}
    \item This is a classic way of modeling in economics: Preferences stay constant
          (and can thus be elicited in just one survey) but outcomes change because
          circumstances change.

\end{itemize}


An important feature of the way we model rapid tests is that the number of actually
performed test is determined endogenously. The number of weekly tests increases until
the end of May and then starts to drop. This is because many  It is determined by the availability of tests,
the individual compliance with testing recommendations and the occurrence of situations
in which a test should be taken.

% Technical terms should be defined.

% Symbols, abbreviations, and acronyms should be defined the first time they are used.

% All tables and figures should be cited in numerical order.

% All data must be shown either in the main text or in the Supplementary Materials or
% must be available in an established database with accession details provided in the
% acknowledgements section.

% References to unpublished materials are not allowed to substantiate significant
% conclusions of the paper.

