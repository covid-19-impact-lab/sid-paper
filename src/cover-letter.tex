% !TEX program = pdflatex
% !TEX TS-program = pdflatex
% !TEX encoding = UTF-8 Unicode



%%%%%%%%%%%%%%%%%%%%%%%%%%%%
%%  FUNDAMENTAL SETTINGS  %%
%%%%%%%%%%%%%%%%%%%%%%%%%%%%


\documentclass[USenglish, 10pt, UBonn_letterhead]{scrlttr2}
% To change the language, set the respective class option to ngerman or UKenglish
% To increase the size of the body font, set the respective class option to 11pt or 12pt

\newkomavar{suppressanniversary}
\newkomavar{suppressbankaccount}
\newkomavar{suppresssigspacing}
\newkomavar{suppressbackaddress}



%%%%%%%%%%%%%%%%%%%%%
%%  SENDER'S DATA  %%
%%%%%%%%%%%%%%%%%%%%%


\IfLanguageName{ngerman}{%
    % German:
    \newcommand{\shortuniversity}{Universität Bonn}
    \newcommand{\higherlevelunit}{Fachbereich Wirtschafts\-wissenschaften}
    \setkomavar{higherlevelunitcaps}{\caps{\higherlevelunit}}
    \setkomavar{institute}{Institut für Angewandte Mikroökonomik}%
    \setkomavar{fromcountry}{Deutschland}
    \setkomavar{place}{Bonn}
    \setkomavar{fromname}{Prof.~Dr.~Hans-Martin v. Gaudecker}
    \setkomavar{signature}{\\[-3ex]\includegraphics[width=8cm]{\string~/admin/unterschrift_lang}\\[0ex]Hans-Martin v. Gaudecker}
}{
    % US or UK English:
    \newcommand{\shortuniversity}{University of Bonn}
    \newcommand{\higherlevelunit}{Department \newline of \newline Economics}
    \setkomavar{higherlevelunitcaps}{\caps{Department} \\ \caps{of} \\ \caps{Economics}}
    \setkomavar{institute}{Institute for Applied~Microeconomics}%
    \setkomavar{fromcountry}{Germany}
    \setkomavar{place}{}
    \setkomavar{fromname}{Prof.~Dr.~Hans-Martin von Gaudecker}
    \setkomavar{signature}{\\[-3ex]\includegraphics[width=8cm]{\string~/admin/unterschrift_lang}\\[0ex]Hans-Martin von Gaudecker}
}
\setkomavar{shortuniversity}{\caps{\shortuniversity}}
\setkomavar{higherlevelunit}{\higherlevelunit}
\setkomavar{shortinstitute}{\caps{IAME}}
\setkomavar{fromaddress}{Lennéstraße~43 \\ 53113~Bonn}
\setkomavar{fromphone}{+49~228~73-9357}
\setkomavar{secretary}{Simone Jost}
\setkomavar{secretaryphone}{+49~228~73-9238}
\setkomavar{fromemail}{\email{hmgaudecker@uni-bonn.de}}
\setkomavar{fromurl}{\urlhttps{www.wiwi.uni-bonn.de/gaudecker}}
\setkomavar{refnum}{}



%%%%%%%%%%%%%%%%%%%
%%  LETTER BODY  %%
%%%%%%%%%%%%%%%%%%%


\begin{document}


\setkomavar{date}{\today}

\setkomavar{subject}{%
    The Effectiveness of Strategies to Contain SARS-CoV-2: Testing, Vaccinations, and NPIs
}

\begin{letter}{%
        To the Editors of\\
        Nature Human Behaviour \\
    }

    \opening{Dear colleague,}

    Many thanks for considering our manuscript for possible publication in
    \textit{Nature Human Behavior}. Our hope is that modelling the evolution of the
    CoViD-19 pandemic and the relative effectiveness of diverse interventions are of
    interest to a broad readership. On way to view the manuscript is as taking the same
    basic approach as the work by Aleta et al.---published a year ago in this
    journal---and demonstrating its usefulness for modelling and understanding the
    spread of the virus under different policies.

    We incorporate detailed NPIs, different tests, and vaccinations in an agent-based
    model, taking into account behavioural reactions to symptoms or positive test
    results. Importantly, we do so in a way that allows to calibrate its parameters
    based on routinely available data. We apply the model to the second and third waves
    of the CoViD-19 pandemic in Germany over a period of 9 months, achieving a good fit
    to actual infection rates.
    
    Our core substantive result is that rapid testing has been very effective relative
    to many NPIs in a low-medium vaccination environment. This is the relevant scenario
    for most countries across the world at present; it seems also applicable to countries
    with high vaccination rates under the conditions of the delta variant.

    \closing{Kind regards,}

\end{letter}


\end{document}
