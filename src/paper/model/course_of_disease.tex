\subsection{Course of the Disease}
\label{sub:disease_progression}

The disease progression in the model is fairly standard. It is depicted in
Figure~\ref{fig:model_disease_progression} and the values and source of the relevant
parameters are describes in Section~\ref{sec:medical_params}.

First, infected individuals will become infectious after one to five days. Overall,
about one third of people remain asymptomatic. The rest develop symptoms about one to
two days after they become infectious. Modeling asymptomatic and pre-symptomatic cases
is important because those people do not reduce their contacts nor do they have an
elevated probability to demand a test. Thus they can potentially infect many other
people \citep{Donsimoni2020}. The probability to develop
symptoms with Covid-19 is highly age dependent with 75\% of children not developing
clinical symptoms \citep{Davies2020}.

A small share of symptomatic people will develop strong symptoms that require intensive
care. The exact share and time span is age-dependent. An age-dependent share of intensive
care unit (ICU) patients will die after spending up to 32 days in intensive care.
Moreover, if the ICU capacity was reached, all patients who require intensive care but do
not receive it die.

It would be easy to make the course of disease even more fine-grained. For example, we
could model people who require hospitalization but not intensive care. So far we opted
against that because only the intensive care capacities are feared to become a bottleneck
in Germany.

We allow the progression of the disease to be stochastic in two ways: Firstly, state
changes only occur with a certain probability (e.g. only a fraction of infected
individuals develops symptoms). Secondly, the number of periods for which an individual
remains in a state is drawn randomly. The parameters that govern these processes are
taken from the literature \footnote{ Detailed information on the calibration of the
disease parameters is available as part of our
\href{https://sid-dev.readthedocs.io/en/latest/reference_guides/epi_params.html}{online
documentation}.}.
%&%&%& \comment[id=J]{Replace this by a link to the relevant table with parameters in the
%data appendix, once we have that table.}


