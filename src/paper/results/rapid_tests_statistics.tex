
\subsection{Rapid Tests}
\label{subsec:appendix_rapid_tests}


\begin{figure}[ht] % Ever Rapid Test
  \centering
  \includegraphics[width=\textwidth]{figures/results/figures/scenario_comparisons/combined_fit/full_share_ever_rapid_test}
  \caption{Share of Individuals Who Have Ever Done a Rapid Test}
  \label{fig:fit_share_ever_rapid_test}
  \floatfoot{\noindent \textit{Note:} This figure clearly shows that overall our
  assumptions on rapid tests are conservative. In our model, the share of individuals who
  have ever done a rapid test lies  consistently 10 percentage points below the share
  reported in the COSMO study \citep{Betsch2021}. The main reason for this is that we
  introduce rapid tests only at the start of 2021. However, using the available data to
  calibrate our rapid test parameters and estimating the remaining parameters to fit the
  official case numbers we arrive at a shape that is very similar to that implied by the
  survey results. This together with fitting the share of Germans with weekly rapid tests
  well (see Figure~\ref{fig:fit_share_rapid_test_within_last_week} makes us confident
  that our rapid test model is a good -- and especially not over-confident --
  representation of rapid testing in Germany.}
\end{figure}


\begin{figure}[ht] % Tested in the last week
  \centering
  \includegraphics[width=\textwidth]{figures/results/figures/scenario_comparisons/combined_fit/full_share_rapid_test_in_last_week}
  \caption{Share of Individuals Who Have Done a Rapid Test in the Last Week}
  \label{fig:fit_share_rapid_test_within_last_week}
  \floatfoot{\noindent \textit{Note:} Our modelling of rapid tests leads to very similar rates of individuals that are tested at least weekly as in the COSMO study \citep{Betsch2021}.
  In that study, individuals were asked if they performed rapaid tests at least weekly in
  the last four weeks. In our model we cannot verify the rhythm over the last four
  weeks but just observe the last time of a rapid test. However, both work and school
  rapid tests which make up the largest share of rapid tests (see
  Figure~\ref{fig:rapid_test_demand_by_channel}) are both required on a weekly or twice
  weekly basis while the private rapid test demand which are partly just triggered by
  events make up only a small fraction of tests.}
\end{figure}


\begin{figure}[ht] % Share tested per day
  \centering
  \includegraphics[width=\textwidth]{figures/results/figures/rapid_test_statistics/popshare_tested}
  \caption{Share of the Population Demanding a Rapid Test Because of Different Channels}
  \label{fig:rapid_test_demand_by_channel}
  \floatfoot{\noindent \textit{Note:}
    Rapid tests in the education setting are demanded by teachers (nursery, preschool and
    school) as well as school pupils. After Easter the required frequency of tests is
    increased from once per week to twice per week. Work rapid tests are demanded by
    individuals that still have work contacts, i.e. do not work from home. The share of
    employers offering rapid tests increases over the time frame and the frequency of
    testing is also increased. Tests are demanded by individuals for one of three private
    reasons: having developed symptoms without access to a PCR test, having a household
    member that has tested positive or developed symptoms or having planned weekly
    meeting with friends. }
\end{figure}


\begin{figure}   % Number of True Positive / False Positive / True Negative / False Negative
    \centering
    \begin{subfigure}[b]{0.425\textwidth}
        \centering
        \includegraphics[width=\textwidth]{figures/results/figures/rapid_test_statistics/number_true_positive}
        \caption{Number of Discovered Cases Due to Rapid Tests by Channel}
        \label{fig:rapid_tests_number_true_positive}
    \end{subfigure}
    \hfill
    \begin{subfigure}[b]{0.425\textwidth}
        \centering
        \includegraphics[width=\textwidth]{figures/results/figures/rapid_test_statistics/number_false_positive}
        \caption{Number of False Positive Rapid Tests by Channel}
        \label{fig:rapid_tests_number_false_positive}
    \end{subfigure}
    \vskip3ex
    \begin{subfigure}[b]{0.425\textwidth}
        \centering
        \includegraphics[width=\textwidth]{figures/results/figures/rapid_test_statistics/number_true_negative}
        \caption{Number of True Negative Rapid Tests by Channel}
        \label{fig:rapid_tests_number_true_negative}
    \end{subfigure}
    \hfill
    \begin{subfigure}[b]{0.425\textwidth}
        \centering
        \includegraphics[width=\textwidth]{figures/results/figures/rapid_test_statistics/number_false_negative}
        \caption{Number of False Negative Rapid Tests by Channel}
        \label{fig:rapid_tests_number_false_negative}
    \end{subfigure}
    \vskip3ex
    \caption{Rapid Test Results}
    \label{fig:rapid_test_results_numbers}

    \floatfoot{\noindent \textit{Note:}
      The number of rapid tests of each category are upscaled to the full German
      population. \textcolor{red}{To be written.}
    }
\end{figure}


\begin{figure} % True Positive / False Positive / True Negative / False Negative Rate
    \centering
    \begin{subfigure}[b]{0.425\textwidth}
        \centering
        \includegraphics[width=\textwidth]{figures/results/figures/rapid_test_statistics/true_positive_rate}
        \caption{Rate of True Positive Rapid Tests by Channel}
        \label{fig:rapid_tests_true_positive_rate}
    \end{subfigure}
    \hfill
    \begin{subfigure}[b]{0.425\textwidth}
        \centering
        \includegraphics[width=\textwidth]{figures/results/figures/rapid_test_statistics/false_positive_rate}
        \caption{Rate of False Positive Rapid Tests by Channel}
        \label{fig:rapid_tests_false_positive_rate}
    \end{subfigure}
    \vskip3ex
    \begin{subfigure}[b]{0.425\textwidth}
        \centering
        \includegraphics[width=\textwidth]{figures/results/figures/rapid_test_statistics/true_negative_rate}
        \caption{Rate of True Negative Rapid Tests by Channel}
        \label{fig:rapid_tests_true_negative_rate}
    \end{subfigure}
    \hfill
    \begin{subfigure}[b]{0.425\textwidth}
        \centering
        \includegraphics[width=\textwidth]{figures/results/figures/rapid_test_statistics/false_negative_rate}
        \caption{Rate of False Negative Rapid Tests by Channel}
        \label{fig:rapid_tests_false_negative_rate}
    \end{subfigure}
    \vskip3ex
    \caption{Rapid Test Rates by Channel}
    \label{fig:rapid_test_results_rates}

    \floatfoot{\noindent \textit{Note:}
      \textcolor{red}{To be written.}
    }
\end{figure}

\begin{figure} % Share of Tests That Are True Positive / False Positive / True Negative / False Negative
    \centering
    \begin{subfigure}[b]{0.425\textwidth}
        \centering
        \includegraphics[width=\textwidth]{figures/results/figures/rapid_test_statistics/true_positive_rate}
        \caption{Share of Tests That Are True Positive by Channel}
        \label{fig:rapid_tests_true_positive_rate}
    \end{subfigure}
    \hfill
    \begin{subfigure}[b]{0.425\textwidth}
        \centering
        \includegraphics[width=\textwidth]{figures/results/figures/rapid_test_statistics/false_positive_rate}
        \caption{Share of Tests That Are False Positive by Channel}
        \label{fig:rapid_tests_false_positive_rate}
    \end{subfigure}
    \vskip3ex
    \begin{subfigure}[b]{0.425\textwidth}
        \centering
        \includegraphics[width=\textwidth]{figures/results/figures/rapid_test_statistics/true_negative_rate}
        \caption{Share of Tests That Are True Negative by Channel}
        \label{fig:rapid_tests_true_negative_rate}
    \end{subfigure}
    \hfill
    \begin{subfigure}[b]{0.425\textwidth}
        \centering
        \includegraphics[width=\textwidth]{figures/results/figures/rapid_test_statistics/false_negative_rate}
        \caption{Share of Tests That Are False Negative by Channel}
        \label{fig:rapid_tests_false_negative_rate}
    \end{subfigure}
    \vskip3ex
    \caption{Share of Rapid Tests by Outcome and True Infection Status}
    \label{fig:rapid_test_results_test_shares}

    \floatfoot{\noindent \textit{Note:}
      \textcolor{red}{To be written.}
    }
\end{figure}






\FloatBarrier