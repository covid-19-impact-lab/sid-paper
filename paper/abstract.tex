\noindent
% Abstracts of Research Articles and Reports should explain to the general reader why
% the research was done, what was found and why the results are important. 
% 
% They should start with some brief BACKGROUND information: a sentence giving a broad
% introduction to the field comprehensible to the general reader, and then a sentence of
% more detailed background specific to your study. 
In order to slow the spread of the CoViD-19 pandemic, governments around the world have
enacted a wide set of policies limiting the transmission of the disease. Initially,
these focused on non-pharmaceutical interventions; more recently, vaccinations and
large-scale rapid testing have started to play a major role. The objective of this study
is to explain the quantitative effects of these policies on determining the course of
the pandemic, allowing for factors like seasonality or virus strains with different
transmission profiles. To do so, the study develops an agent-based simulation
model, which is estimated using data for the second and the third wave of the CoViD-19
pandemic in Germany. The paper finds that during a period where vaccination rates rose
from 5\% to 40\%, large-scale rapid testing had the largest effect on reducing infection
numbers. 
% The final sentence should outline the main CONCLUSIONS of the study, in terms that
% will be comprehensible to all our readers. The Abstract is distinct from the main body
% of the text, and thus should not be the only source of background information critical
% to understanding the manuscript. Please do not include citations or abbreviations in
% the Abstract. 
Frequent large-scale rapid testing should remain part of strategies to contain
CoViD-19; it can substitute for many non-pharmaceutical interventions that come at a
much larger cost to individuals, society, and the economy.

% The abstract should be 125 words or less.

\vspace{1cm}
\noindent \textbf{JEL Classification:} C63, I18

\noindent \textbf{Keywords:} CoViD-19, agent based simulation model, rapid testing,
non-pharmaceutical interventions