\noindent
In the second year of the pandemic, governments worldwide have been adopting diverse and nuanced policy measures to contain the spread of Covid-19.
Examples are targeted contact reduction policies, mask mandates and other hygiene measures, vaccinations and large scale rapid testing.
Standard epidemiological models are not designed to predict the effect of very fine grained policies, nor to model the effect of many interacting measures at a time.
%
We propose a novel simulation-based model to address these shortcomings. We build on
state-of-the-art agent-based simulation models, greatly increasing the amount of detail
and realism with which contacts take place. Firstly, we allow for different contact
types (such as work, school, households or leisure), distinguish recurrent and
non-recurrent contacts and allow the infectiousness of meetings to vary between contact
types.
% Secondly, we strictly separate the number of contacts from the probabilities that a
% contact leads to an infection.
Secondly, we allow agents to seek tests and react to information, such as experiencing
symptoms, receiving a positive test or a known case among their contacts, by reducing
their own contacts.
% Thus, social distancing policies can affect the number of contacts while the infection
% probabilities remain invariant.
The model is able to explain the surprising drop in case numbers observed in many european countries in the spring of 2021 as the combined effect of seasonality,
vaccinations, contact reduction policies and rapid testing and allows to quantify
the contribution each policy measure has.


\vspace{1cm}
\noindent JEL Classification: C63, I18

\noindent Keywords: Covid-19, agent based simulation model, public health measures
