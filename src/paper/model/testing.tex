\subsection{Testing} % (fold)
\label{sub:testing}

Having a realistic model of PCR and rapid tests is crucial for two reasons: Firstly, only
via a testing model can the simulated infections from the model be made comparable to
official case numbers. Secondly, individuals with undetected or not yet detected
infections are an important driver of the pandemic.

In principle, our modeling approach is flexible enough to incorporate mechanistic test
demand, allocation and processing models. However, there is not enough data available to
calibrate such a mechanistic model.

Therefore, we build a simpler model of PCR and rapid tests that can be calibrated with
available data on test demand and availability and -- nevertheless
-- can produce a share of undetected cases that varies over time and across age groups
and agrees with other estimates over the time periods where they are available.

PCR tests are modeled since the beginning of the simulation and determine whether a
infection is officially recorded. Rapid tests are only added at the beginning of 2021.
Positive rapid tests do not enter official case numbers directly, but most people with a
positive rapid tests demand a confirmatory PCR test. However, positive rapid tests can
have a strong effect on the infection dynamics because they trigger contact reductions
and additional rapid tests.

During 2020 people can demand PCR tests either because they have symptoms or randomly.
The probability that a PCR test is performed in each of the two situations depends on the
number of new infections and the number of available tests. Thus, it varies strongly over
time and is unknown.

To distribute the correct number of PCR tests among symptomatic and asymptomatic
infections without knowing explicit test demand probabilities, we use the following
approach: First, we calculate the total number of positive PCR tests by multiplying the
number of newly infected individuals with an estimate of the share of detected cases
from the Dunkelzifferradar project \citep{Dunkelzifferradar2020}. Next, we determine how
many of these tests should go to symptomatic and asymptomatic individuals from data by
the RKI \citep{ARS2020}. Then, we sample the individuals to which those tests
are allocated from the pools of symptomatic and asymptomatic infected but not yet tested
individuals.

Sampling uniformly from the pool of symptomatic individuals ensures that age groups who
are more likely to develop symptoms are also more likely to receive tests. Thus, the
share of detected cases is much higher for the elderly than for children in time periods
where many tests are done because of symptoms.

At the beginning of 2021, two challenges arise: Firstly, the externally estimated share
of detected cases from Dunkelzifferradar project \citep{Dunkelzifferradar2020} can no
longer be used because it is based on the case fatality rate which changes drastically
due to vaccinations. Secondly, rapid tests become available at a large scale.

We solve the first challenge by assuming that the share of detected cases would have
remained at the level it reached before Christmas if rapid tests had not become
available. While this is only an approximation to reality, changes in the share of
detected cases that would have happened without rapid tests are very likely to be small
compared to the changes caused by rapid tests.

The second challenge is solved by mechanistic rapid test demand models for the
workplace, schools and by private individuals. The calibration of these models is
described in Section~\ref{subsec:rapid_test_demand}.
Figure~\ref{fig:antigen_tests_vaccinations} shows that the number of performed rapid
tests in the model fits the empirical data well (where empirical data is available).

In contrast to PCR tests, rapid tests are not perfect and can be falsely positive or
falsely negative. While the specificity of rapid tests is calibrated at 99.4\%
\citep{Bruemmer2021}, their sensitivity strongly depends on the timing of the rapid test
relative to the start of infectiousness \citep{Smith2021}. Before the onset of
infectiousness the sensitivity is very low (35\%). On the first day of infectiousness it
is much higher (88\%) but still lower than during the remaining infectious period (92\%).
After infectiousness stops, the sensitivity drops to 50 \%.

Modeling the diagnostic gap before and at the beginning of infectiousness is very
important to address concerns that rapid tests are too unreliable to serve as screening
devices.

We do not distinguish between self administered rapid tests and those that are
administered by medical personnel. While there were concerns that self administered
tests are less reliable, a recent study has found no basis for this
concern \citep{Lindner2020}.

While rapid tests do not directly enter official case numbers, 82\%
($\chi_{confirmation}$) of positively tested individuals seek a PCR test
\citep{Betsch2021}. Importantly, those PCR tests are made in addition to the tests that
would have been done otherwise. Section~\ref{subsec:results_share_known_cases} discusses
the effect of rapid tests on the share of detected cases.

