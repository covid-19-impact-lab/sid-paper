\subsection{Estimated Parameters}
\label{subsec:estimated_params}

%%% The free parameters are the infection probabilities by contact category, one hygiene
%%% multiplier for school and work contacts after November 2nd 2020, approx. ten parameters
%%% that govern the reduction of other contacts\comment[id=K]{Must be updated when the
%%% estimation is finished.}, the number of extra contacts during holidays, the share of
%%% detected cases around holidays and the fade in speed of the rapid tests as well as one
%%% parameter that governs the import of B.1.1.7 cases in January 2021.

\FloatBarrier

We estimate parameters that cannot be calibrated outside of the model with the
method of simulated moments \citep{McFadden1989} by minimizing the distance between
simulated and observed infection rates and fatality rates (disaggregated by region
and age groups). Since our model includes a lot of randomness, we
average simulated infection rates over several model runs.

We fit our model to data for Germany from October 2020 until June 2021. We do not use
earlier periods for three reasons. Firstly, in the beginning PCR tests were highly
limited and therefore it would be difficult to find good initial conditions for our
simulations. In addition during the summer the case numbers were extremely low. This
could lead to the epidemic going extinct in our simulation. Additionally, our model does
not include international travel or other imports of cases. These would be important but
difficult to model during the summer months.

% infection probabilities

To avoid over-fitting and simplify the numerical optimization problem, we only allow for
five different infection probabilities: 1) for contacts in schools 2) contacts in
preschools and nurseries. 3) for work contacts. 4) for households. 5) for other
contacts.\comment[id=J]{Write that the exact numbers are hard to interpret and show the
infections by channel heatmap more prominently than the infection probabilities}

% policies
We also estimate a parameter that reflects the effect of hygiene measures
after November 2020 at work and in educational facilities. This parameter reduces
infectiousness of contacts by one third. In total those are 10 parameters. The
breakpoints the contact reduction changes are not determined from data but from
announcements of policy changes. Moreover, we constrain the estimated contact reduction
to follow the shape of the stringency index. The resulting contact reduction can be
seen in Figure~\ref{fig:other_multiplier}


% b117 rate
Finally we estimate one parameter that governs the introduction of the B.1.1.7 virus
variant in January 2021. This parameter implies that at the end of January roughly one
case per 100 000 individuals per day is imported. After January we do not model imported
cases of B.1.1.7 anymore because they are negligible compared to the endogenous growth
of that virus variant.

% The infection probabilities estimated from our model are as shown in
% Table~\ref{tab:infection_probs}.\comment[id=K]{Missing to add data for the other
% estimated parameters (import of b117 etc)}\comment[id=K]{Update table after the new
% estimation is finished}

% \begin{table}[tb]
%     \caption{Estimated Infection Probabilities}
%     \label{tab:infection_probs}
%     \centering
%     \begin{tabular}{ll}
%         \toprule
%         Contact Type & Infection Probability \\
%         \midrule
%         preschool and nursery & 00.500\% \\
%         school & 01.200\% \\
%         household & 10.000\% \\
%         work & 14.500\% \\
%         other & 15.875\% \\
%         \bottomrule
%     \end{tabular}
%     \tablenotes{}
% \end{table}


