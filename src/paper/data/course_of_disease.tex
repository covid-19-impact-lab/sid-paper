\subsection{Course of Disease}
\label{sub:data_course_of_disease}

This section discusses the parameters governing the course of disease, their sources and
how we arrived at the distributions used in the paper.
See Figure~\ref{fig:model_disease_progression} for a summary of our
disease progression model.


% Duration until Infectiousness and Symptoms ----------------------------------

% Latency Period
We denote the latent period---i.e., the time span between infection and the start of
infectiousness---by $\gamma_{infectious}$. \cite{Zhao2021} estimate the latent period to
last 3.3 days (95\% CI: 0.2, 7.9) on average. In line with this estimate our latent
period lasts one to five days.

% Asymptomatic Share
Once individuals become infectious, a share of them goes on to develop symptoms while
others remain asymptomatic. We rely on data by \cite{Davies2020} for the age-dependent
probability to develop symptoms. It varies from 25\% for children and young adults to
nearly 70\% for the elderly.
% Length of the presymptomatic stage
Similar to \citet{Peak2020} and in line with \citet{He2020} we set the length of the
presymptomatic stage of age group $a$, $\gamma_{symptoms,\:a}$ to be one or two days. The
probability to become symptomatic for age group $a$ is split equally between one and two
days. This combined with our latency period leads to an incubation period that is in line
with the meta analysis by \citet{McAloon2020}.

% Duration of Infectiousness ----------------------------------------------------------

We assume that the duration of infectiousness ($\gamma_{stop\:infectious}$) is the same
for both symptomatic and asymptomatic individuals as evidence suggests little differences
in the transmission rates between symptomatic and asymptomatic patients (\citet{Yin2020})
and that the viral load between symptomatic and asymptomatic individuals are similar
(\citet{Zou2020}, \citet{Byrne2020}, \citet{Singanayagam2020}). Our distribution of the
duration of infectiousness is based on \citet{Byrne2020}. For symptomatic cases they
arrive at zero to five days before symptom onset (see their figure 2) and three to eight
days of infectiousness afterwards.\footnote{Viral loads may be detected much later but
eight days seems to be the time after which most people are culture negative, as also
reported by \citet{Singanayagam2020}.} Excluding the most extreme combinations, we arrive
at 3 to 11 days as the duration of infectiousness.

% Duration of Symptoms ----------------------------------------------------------

We use the duration to recovery of mild and moderate cases reported by \cite[Figure~S3,
Panel~2]{Bi2020} for the duration of symptoms for non-ICU requiring symptomatic cases
($\gamma_{stop\:symptoms}$). We only disaggregate by age how likely individuals
are to require intensive care. %

For the time from symptom onset until need for intensive care we rely on data by
\cite{Stokes2020}) and \cite{Hinch2021} ($\gamma_{icu,\:a}$). For those who will require
intensive care we follow \citet{Chen2020} who estimate the time from symptom onset to ICU
admission as $8.5 \pm 4$ days. This aligns well with numbers reported for the time from
first symptoms to hospitalization: \citet{Gaythorpe2020} report a mean of 5.76 with a
standard deviation of four. We assume that the time between symptom onset and ICU takes
four, six, eight or ten days with equal probabilities.

% Death and Recovery from ICU ----------------------------------------------------------
We take the survival probabilities and time to death and time until recovery
($\gamma_{stop\:icu\:a}$ and $\gamma_{dead,\:a}$) from intensive care from
\citet{Hinch2021}. They report time until death to have a mean of 11.74 days and a
standard deviation of 8.79 days. To match this we discretize that 41\% of individuals who
will die from Covid-19 do so after one day in intensive care, 22\% day after twelve days,
29\% after 20 days and 7\% after 32 days. Again, we rescale this for every age group
among those that will not survive. For survivors \cite{Hinch2020} reports a mean duration
of 18.8 days until recovery and a standard deviation of 12.21 days. We discretize this
such that of those who recover in intensive care, 22\% do so after one day, 30\% after 15
days, 28\% after 25 days and 18\% after 45 days.

\FloatBarrier
