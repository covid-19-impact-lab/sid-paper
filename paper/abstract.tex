\noindent
Governments worldwide have been adopting diverse and nuanced policy measures
to contain the spread of Covid-19.
However, epidemiological models usually lack the detailed representation of
human meeting patterns to credibly predict the effects such policies.
%
We propose a novel simulation-based model to address these shortcomings.
We build on state-of-the-art agent-based simulation models, greatly increasing
the amount of detail and realism with which contacts take place.
Firstly, we allow for different contact types (such as work, school, households
or leisure), distinguish recurrent and non-recurrent contacts and allow the
infectiousness of meetings to vary between contact types.
% Secondly, we strictly separate the number of contacts from the probabilities
% that a contact leads to an infection.
Secondly, we allow agents to seek tests and react to information, such as
experiencing symptoms, receiving a positive test or a known case among their
contacts, by reducing their own contacts.
% Thus, social distancing policies can affect the number of contacts while the
% infection probabilities remain invariant.
This allows us to model the effects of a wide array very targeted policies such
as split classes, mandatory work from home schemes or test-and-trace policies.
To validate our model, we show that it can predict the effect of the German
November lockdown even if no similar policy has been observed during the time
period that was used to estimate the model parameters.


\vspace{1cm}
\noindent JEL Classification: C63, I18

\noindent Keywords: Covid-19, agent based simulation model, public health measures
