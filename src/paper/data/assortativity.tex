\subsection{Assortativity}
\label{subsec:assortativity}

As mentioned in section \ref{sec:matching}, the probability that two individuals are
matched can depend on background characteristics. In particular, we allow this
probability to depend on age and county of residence. While we do not have good data on
geographical assortativity and just roughly calibrate it such that 80\% of contacts are
within the same county, we can calibrate the assortative mixing by age from the same data
we use to calibrate the number of contacts.

\begin{figure}[ht]
    \centering
    \includegraphics[width=0.9 \textwidth]{figures/results/figures/data/assortativity_other_non_recurrent}
    \caption{Distribution of Non Recurrent Other Contacts by Age Group}
    \label{fig:assortativity_other}
    \floatfoot{\noindent \textit{Note:} The figure shows the distribution of non
        recurrent other contacts by age group. A row shows the share of contacts a
        certain age group has with all other age groups. Higher values are colored in
        darker red tones. The diagonal represents the share of contacts with individuals
        from the same age group.}
\end{figure}


Figure~\ref{fig:assortativity_other} shows that assortativity by age is especially strong
for children and younger adults. For older people, the pattern becomes more dispersed
around their own age group, but within-age-group contacts are still the most common
contacts.

\begin{figure}[ht]
    \centering
    \includegraphics[width=0.9 \textwidth]{figures/results/figures/data/assortativity_work_non_recurrent}
    \caption{Distribution of Random Work Contacts by Age Group}
    \label{fig:assortativity_work}
    \floatfoot{\noindent \textit{Note:} The figure shows the distribution of non
        recurrent work contacts by age group. A row shows the share of contacts a certain
        age group has with all other age groups. Higher values are colored in darker red
        tones. The diagonal represents the share of contacts with individuals from the
        same age group. We only show age groups that have a significant fraction of
        working individuals.}
\end{figure}

Figure~\ref{fig:assortativity_work} shows that assortativity by age is also important
among work contacts.

Our two other types of contacts, households and schools, get their assortativity by
construction. Schools are groups where the same children of the mostly same age and
county meet with teachers every day. Household composition follows directly from the
German microcensus data we use to construct our synthetic population.

\FloatBarrier