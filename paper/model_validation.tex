\section{Model Validation}
\label{sec:model_validation}

We validate our model in two ways: 1) We look at the in-sample fit over the estimation
period. 2) We look at the out-of-sample fit for November. The last one is a challenging
test for our model because there was a strong policy change between the estimation
period and November. The model convincingly passes both tests.


\subsection{In-Sample Fit}
\label{sub:in_sample_fit}

Despite fitting only four infection probabilities, the in-sample fit is very good.
The best fit is achieved in the largest age groups.
This is so mechanically, because we weigh the
deviations between simulated and observed infection rates by group sizes. The worst fit
is achieved for the 80 to 100 years old.
There are three reasons for this: Firstly, we only model private households at
the moment, meaning that nursing homes are not part of the synthetic population we
simulate. Since community housing residents are part of the German Mikrozensus
\citep{FDSAeDBUDL2018} which forms the basis for our synthetic population, we can and
plan to include community housing residents in the near future.
Secondly, the elderly, especially nursing home residents, are tested more often than
the general population leading to more cases being detected in this age group.
We are currently working to include age-variant testing strategies,
using \href{https://ars.rki.de/Content/COVID19/Main.aspx}{data by the RKI}
\citep{Seifried2020}.

\begin{figure}[h]
    \centering
    \includegraphics[width=1.1\textwidth]{../figures/goodness_of_fit_by_age_group}
    \caption{Reported vs. simulated weekly incidence rates of infections.}
    \label{fig:goodness_of_fit}
    \figurenotes{
        The figure shows the weekly incidence rates per 100,000 people for the reported
        (red line) versus the simulated infections rates (blue line) for age groups
        available in the data provided by the Robert-Koch Institut.
    }
\end{figure}

\FloatBarrier

\subsection{Out-of-Sample Fit}
\label{sub:out_of_sample_fit}

We can assess the out-of-sample fit by projecting the effect of the lockdown light and
comparing it to case numbers until mid of November. It is important to note that this
is not just a simple extrapolation of a time trend because the lockdown light only started
after the estimation period. The out-of-sample fit can be assessed in
Figure~\ref{fig:out-of-sample-fit}.

\begin{figure}[!tp]
    \centering
    \includegraphics[width=\textwidth]{../figures/out_of_sample_validation}
    \caption{Predicted effect of the "Lockdown Light" on infection rates.}
    \label{fig:out-of-sample-fit}
    \figurenotes{
        For the time period until the beginning of November, the figure shows the weekly
        incidence rates of infections per 100,000 people from reported (black) versus
        simulated (blue) data. With the start of November, the projections of the three
        scenarios, optimistic (blue), neutral (red), and pessimistic (mint green), are
        shown until the beginning of the Christmas holidays. The actual incidence rates
        (black) are reported until the 24th November.
    }
\end{figure}

The model correctly predicts the effect of the lockdown light with reasonable accuracy.
In particular, the actual case numbers are between our neutral and pessimistic
projection. The plot also shows that ending the lockdown light on November 30, as was
originally planned, would lead to an explosive growth in case numbers in all scenarios.
