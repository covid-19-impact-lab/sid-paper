\documentclass[fleqn,10pt]{wlscirep}
\usepackage[utf8]{inputenc}
\usepackage[T1]{fontenc}
\title{
    The Effectiveness of Testing, Vaccinations and Contact Restrictions for Containing
    the CoViD-19 Pandemic
}

\author[1, 2, +]{Janoś Gabler}
\author[3, +]{Tobias Raabe}
\author[1, +]{Klara Röhrl}
\author[2, 4, *, +]{Hans-Martin von Gaudecker}

\affil[1]{Bonn Graduate School of Economics, Bonn, 53113, Germany}
\affil[2]{IZA Institute of Labor Economics, Bonn, 53113, Germany}
\affil[3]{Private sector}
\affil[4]{Rheinische Friedrich-Wilhelms-Universität Bonn, Bonn, 53113, Germany}

\affil[*]{hmgaudecker@uni-bonn.de}

\affil[+]{These authors contributed equally to this work}

\keywords{
    CoViD-19, agent based simulation model, rapid testing, non-pharmaceutical
    interventions
}

\begin{abstract}
    In order to slow the spread of the CoViD-19 pandemic, governments around the world
    have enacted a wide set of policies limiting the transmission of the disease.
    Initially, these focused on non-pharmaceutical interventions; more recently,
    vaccinations and large-scale rapid testing have started to play a major role. The
    objective of this study is to explain the quantitative effects of these policies on
    determining the course of the pandemic, allowing for factors like seasonality or
    virus strains with different transmission profiles. To do so, the study develops an
    agent-based simulation model, which explicitly takes into account test demand and
    behavioral changes following positive tests. The model is estimated using data for
    the second and the third wave of the CoViD-19 pandemic in Germany. The paper finds
    that during a period where vaccination rates rose from 5\% to 40\%, seasonality and
    rapid testing had the largest effect on reducing infection numbers. Frequent
    large-scale rapid testing should remain part of strategies to contain CoViD-19; it
    can substitute for many non-pharmaceutical interventions that come at a much larger
    cost to individuals, society, and the economy.
\end{abstract}

\begin{document}

\flushbottom
\maketitle
\thispagestyle{empty}

\section*{Introduction}

Since early 2020, the CoViD-19 pandemic has presented an enormous challenge to humanity
on many dimensions. The development of highly effective vaccines holds the promise of
containment in the medium term. However, most countries find themselves many
months---and often years---away from reaching vaccination levels that would end the
pandemic or even protect the most vulnerable \cite{Mathieu2021}. In the meantime, it is
of utmost importance to employ an effective mix of strategies for containing the virus.
The most frequent initial response was a set of non-pharmaceutical interventions (NPIs)
to reduce contacts between individuals. While this has allowed some countries to sustain
equilibria with very low infection numbers (see \cite{Contreras2021} for a theoretical
equilibrium at low case numbers which is sustained with test-trace-and-isolate
policies), most have seen large fluctuations of infection rates over time. Containment
measures have become increasingly diverse and now include rapid testing, more nuanced
NPIs, and contact tracing. Neither these policies' effects nor the influence of seasonal
patterns or of more infectious virus strains are well understood in quantitative terms.

This paper develops a quantitative model incorporating these factors simultaneously. The
framework allows to combine a wide variety of data and mechanisms in a timely fashion,
making it useful to predict the effects of various interventions. Behavioral reactions
to symptoms or positive tests are explicitly taken into account. We apply the model to
Germany, where new infections fell by almost 80\% during May 2021. Our analysis shows
that, aside from seasonality, frequent and large-scale rapid testing caused the bulk of
this decrease, which is in line with prior predictions \cite{Mina2021}. We conclude
that it should have a large role for at least as long as vaccinations have not been
offered to an entire population.

\section{Model description}

At the core of our agent-based model---we review more literature in Supplementary
Material~B.1 \cite{Aleta2020,Hinch2021a}---are physical contacts between heterogeneous
agents (Figure~1a). Each contact between an infectious individual and somebody
susceptible to the disease bears the risk of transmitting the virus. Contacts occur in
up to four networks: Within the household, at work, at school, or in other settings
(leisure activities, grocery shopping, medical appointments, etc.). Some contacts recur
regularly, others occur at random. Empirical applications can take the population and
household structure from census data and the network-specific frequencies of contacts
from diary data measuring contacts before the pandemic \cite{Mossong2008,Hoang2019}.
Within each network, meeting frequencies depend on age and geographical location (see
Supplementary Material~A.4).

The four contact networks are chosen so that the most common NPIs can be modeled in
great detail. NPIs affect the number of contacts or the risk of transmitting the disease
upon having physical contact. The effect of different NPIs will generally vary across
contact types. For example, a mandate to work from home will reduce the number of work
contacts to zero for a fraction of the working population. Schools and daycare can be
closed entirely, operate at reduced capacity---including an alternating schedule---, or
implement mitigation measures like masking requirements or air filters
\cite{Lessler2021}. Curfews may reduce the number of contacts in settings outside of
work and school. In any setting, measures like masking requirements would reduce the
probability of infection associated with a contact \cite{Cheng2021}.

\begin{figure}[tb]
    \centering
    \includegraphics[width=\textwidth]{../../bld/panel_figures/fig-model-description.pdf}
    \caption{
        A~description of the model can be found in Supplementary Material~B. Figure~1a
        shows the influence of an agent's contacts to other agents on infections.
        Demographic characteristics set the baseline number of contacts in different
        networks ($\eta$). The agents may reduce the number of contacts due to NPIs,
        showing symptoms, or testing positively for SARS-CoV-2 ($\tau$). Infections may
        occur when a susceptible agent meets an infectious agent; the probability
        depends on the type of contact ($\beta_c$), on seasonality ($\kappa_c$), and on
        NPIs ($\rho_{c,\:t}$). If infected, the infection progresses as depicted in
        Figure~1b. If rapid tests are available, agents' demand is modeled as in
        Figure~1c. All reasons trigger a test only for a fraction of individuals
        depending on an individual compliance parameter; the thresholds for triggering
        test demand differ across reasons and they may depend on calendar time
        ($\pi_{c,\:t}$ and $\tau_{c,\:t}$). Figure~1d shows the model of translating all
        infections in the simulated data to age-specific recorded infections. The model
        uses data on the aggregate share of recorded cases ($\psi$), the share of
        positive PCR tests triggered by symptoms ($\chi_{symptom}$), and the false
        positive rate of rapid tests ($p_{positive|infected,\;i,\:t}$). The lower part
        of the graph is relevant only for periods where rapid tests are available. All
        parameters are explained in Supplementary Material~A.11.
    }
    \label{fig:model_description}
\end{figure}

In the model, susceptibility to contracting the SARS-CoV-2 virus is dependent on age
\cite{Davies2020,Goldstein2021}. A possible infection progresses as shown in Figure~1b.
We differentiate between an initial period of infection without being infectious or
showing symptoms, being infectious (presymptomatic or asymptomatic), showing symptoms,
requiring intensive care, and recovery or death similar to \cite{Aleta2020}. The
probabilities of transitioning between these states depend on age; their duration is
random and calibrated to medical literature (for a detailed description see
Supplementary Material~A.1). Conditional on the type of contact, infectiousness is
independent of age \cite{Jones2021}.

The model includes several other features, which are crucial to describe the evolution
of the pandemic in 2020-2021. New virus strains with different infectiousness profiles
may appear. Vaccines may become available. During the vaccine roll-out, priority may
depend on age and occupation; vaccine hesitancy is modelled by some individuals refusing
vaccination offers. With some probability, vaccinated agents become immune and do not
transmit the virus \cite{Hunter2021, LevineTiefenbrun2021, Petter2021, Pritchard2021}.

We include two types of tests. Polymerase chain reaction (PCR) tests reveal whether an
individual is infected or not; there is no uncertainty to the result. PCR tests require
at least one day to be processed and there are aggregate capacity constraints. In
contrast, rapid antigen tests yield immediate results. Specificity and sensitivity of
these tests is set according to data analyzed in
\cite{Bruemmer2021,Scheiblauer2021,Ozcurumez2021}; sensitivity depends on the timing of
the test relative to the onset of infectiousness as in \cite{Smith2021}. We analyse
robustness to different assumptions in Supplementary Material~A.12. After a phase-in
period, all tests that are demanded will be performed. Figure~1c shows our model for
rapid test demand. Schools may require staff and students to be tested regularly. Rapid
tests may be offered by employers to on-site workers. Individuals may demand tests for
private reasons, which include having plans to meet other people, showing symptoms of
CoViD-19, and a household member having tested positively for the virus. We endow each
agent with an individual compliance parameter. This parameter determines whether she
takes up rapid tests. (Positive test results or symptoms lead most individuals to reduce
their contacts; this is why tests impact the actual contacts in Figure~1.)

Modelling a population of agents according to actual demographic characteristics means
that we can use a wide array of data to identify and calibrate the model's many
parameters (see Supplementary Material~A for a complete description). Contact diaries
yield pre-pandemic distributions of contacts for different contact types and their
assortativity by age group. Mobility data is used to model the evolution of work
contacts. School and daycare policies can be incorporated directly from official
directives. Administrative records on the number of tests, vaccinations by age and
region, and the prevalence of virus strains are generally available. Surveys may ask
about test offers, propensities to take them up, and past tests. Other studies'
estimates of the seasonality of infections can be incorporated directly. The remaining
parameters---most notably, these include infection probabilities by contact network and
the effects of some NPIs, see Supplementary Material~A.9---will be chosen numerically so
that the model matches features of the data (see \cite{McFadden1989} for the general
method). In our application, we keep the number of free parameters low in order to avoid
overfitting. The data features to be matched include official case numbers for each age
group and region, deaths, and the share of the B.1.1.7 strain.

The main issue with official case numbers is that they will contain only a fraction of
all infections. In the German case, this specifically amounts to positive PCR tests. We
thus model recorded cases as depicted in Figure~1d. We take mortality-based aggregate
estimates of the share of detected cases and use data on the share of PCR tests
administered because of CoViD-19 symptoms. As the share of asymptomatic individuals
varies by age group, this gives us age-specific shares (see Figure~B.11). Our estimates
suggest that---in the absence of rapid testing---the detection rate is 80\% higher on
average for individuals above age 80 compared to school age children. Once rapid test
become available, confirmation of a positive result is another reason leading to
positive PCR tests.


\section*{Second and third waves of the CoViD-19 pandemic in Germany}

The model is applied to the second and third waves of the CoViD-19 pandemic in Germany,
covering the period mid-September 2020 to the end of May 2021. Figure~2a describes the
evolution of the pandemic and of its drivers. The black line in Figure~2a shows
officially recorded cases; the black line in Figure~2b the Oxford Response Stringency
Index \cite{Hale2020}, which tracks the tightness of non-pharmaceutical interventions.
The index is shown for illustration of the NPIs, we never use it directly. For
legibility reasons, we transform the index so that lower values represent higher levels
of restrictions. A value of zero means all measures incorporated in the index are turned
on. The value one represents the situation in mid-September, with restrictions on
gatherings and public events, masking requirements, but open schools and workplaces. In
the seven weeks between mid September and early November, cases increased by a factor of
ten. Restrictions were somewhat tightened in mid-October and again in early November.
New infections remained constant throughout November before rising again in December,
prompting the most stringent lockdown to this date. Schools and daycare centers were
closed, so were customer-facing businesses except for grocery and drug stores. From the
peak of the second wave just before Christmas until the trough in mid-February, newly
detected cases decreased by almost three quarters. The third wave in the spring of 2021
is associated with the B.1.1.7 (Alpha) strain, which became dominant in March
(Figure~2c)---B.1.617.2 (Delta) was first detected in Germany in April; at the end of
our simulation period it accounted for less than 5\% of cases. In early March, some NPIs
were relaxed; e.g., hairdressers and home improvement stores were allowed to open again
to the public. There were many changes in details of regulations afterwards, but they
did not change the overall stringency index.

\begin{figure}[tb]
    \centering
    \includegraphics[width=\textwidth]{../../bld/panel_figures/fig-evolution-pandemic.pdf}
    \caption{
        Evolution of the pandemic, its drivers, and model fit, September 2020 to May
        2021: Data sources are described in Supplementary Material~A. Age- and
        region-specific analogues to Figure~2a can be found in Supplementary
        Material~B.10. For legibility reasons, all lines in Figure~2b are rolling 7-day
        averages. The Oxford Response Stringency Index is scaled as $2 \cdot (1 -  x /
        100)$, so that a value of one refers to the situation at the start of our sample
        period and zero means that all NPIs included in the index are turned on. The
        other lines in Figure~2b show the product of the effect of contact reductions,
        increased hygiene regulations, and seasonality. See Appendix~A.5 for separate
        plots of the three factors by contact type.
    }
    \label{fig:pandemic_drivers_model_fit}
\end{figure}

By March 2021, the set of policy instruments had become much more diverse. Around the
turn of the year, the first people were vaccinated with a focus on older age groups and
medical staff (Figure~2d). Until the end of May, 43\% had received at least one dose of
a vaccine. In late 2020, rapid tests started to replace regular PCR tests for staff in
many medical and nursing facilities. These had to be administered by medical doctors or
in pharmacies. At-home tests approved by authorities became available in mid-March.
Rapid test centers were opened, and one test per person and week was made available free
of charge. In several states, customers were only allowed to enter certain stores with a
recent negative rapid test result. These developments are characteristic of many
countries: The initial focus on NPIs to slow the spread of the disease has been
accompanied by vaccines and a growing acceptance and use of rapid tests. At broadly
similar points in time, novel strains of the virus have started to pose additional
challenges.


\section*{Results}

We draw simulated samples of agents from the population structure in September 2020 and
use the model to predict recorded infection rates until the end of May 2021. See
Supplementary Materials~A.2 and B.9 for details. The blue line in
Figure~2a shows that our model's predictions are very close to
officially recorded cases in the aggregate. This is also true for infections by age and
geographical region (see Supplementary Material~B.10).

The effects of various mechanisms can be disentangled due to the distinct temporal
variation in the drivers of the pandemic. Next to the stringency index, the three lines
in Figure~2b summarize how contact reductions,
increased hygiene regulations, and seasonality evolved since early September for each of
the three broad contact networks. For example, a value of 0.75 for the work multiplier
means that if the environment was the same as in September (levels of infection rates,
no rapid tests or vaccinations, only the wildtype virus present), infections at the
workplace would be reduced by 25\%. Two aspects are particularly interesting. First, all
lines broadly follow the stringency index and they would do so even more if we left out
seasonality and school vacations (roughly the last two weeks of October, two weeks each
around Christmas and Easter, and some days in late May). Second, the most stringent
regulations coincide with the period of decreasing infection rates between late December
2020 and mid-February 2021. The subsequent reversal of the trend is associated with the
spread of the B.1.1.7 variant. During the steep drop in recorded cases during May 2021,
for 42\% of the population took at least one rapid tests per week, the first-dose
vaccination rate rose from 28\% to 43\%, and seasonality lowered the relative
infectiousness of contacts.

In order to better understand the contributions of rapid tests, vaccinations, and
seasonality on the evolution of infections in 2021,
Figure~\ref{fig:2021_scenarios_broad} considers various scenarios. NPIs are always held
constant at their values in the baseline scenario. Figure~3a shows the model fit (the
blue line, same as in Figure~2a), a scenario without any of the three factors (red
line), and three scenarios turning each of these factors on individually. Figure~3b does
the same for total infections in the model. Figure~3c employs Shapley values
\cite{Shapley2016} to decompose the difference in total infections between the scenario
without any of the three factors and our main specification.

\begin{figure}[tb]
    \centering
    \includegraphics[width=\textwidth]{../../bld/panel_figures/fig-2021-scenarios.pdf}
    \caption{
        The effect of different interventions on recorded and actual infections. The
        blue line in Figure~3a is the same as in Figure~2a and refers to our baseline
        scenario, so does the blue line in Figure~3b. The red lines refer to a situation
        where NPIs evolve as in the baseline scenario and the B.1.1.7 variant is
        introduced in the same way; vaccinations, rapid tests, and seasonality remain at
        their January levels. The other scenarios turn each of these three factors on
        individually. The decompositions in Figures~3c and 3d are based on Shapley
        values, which are explained more thoroughly in Appendix~A.10. For legibility
        reasons, all lines are rolling 7-day averages.
    }
    \label{fig:2021_scenarios_broad}
\end{figure}

Until mid-March, there is no visible difference between the different scenarios.
Seasonality hardly changes, and only few vaccinations and rapid tests were administered.
Even thereafter, the effect of the vaccination campaign is surprisingly small at first
sight. Whether considering recorded or total infections with only one channel active,
the final level is always the highest in case of the vaccination campaign (orange
lines). The Shapley value decomposition shows that vaccinations contribute 16\% to the
cumulative difference between scenarios. Reasons for the low share are the slow
start---it took until March~24th until 10\% of the population had received their first
vaccination, the 20\% mark was reached on April 19th---and the focus on older
individuals. These groups contribute less to the spread of the disease than others due
to a lower number of contacts. By the end of our study period, when first-dose
vaccination rates reached 43\% of the population, the numbers of new cases would have
started to decline. It is important to note that the initial focus of the campaign was
to prevent deaths and severe disease. Indeed, the case fatality rate was considerably
lower during the third wave when compared to the second (4.4\% between October and
February and 1.4\% between March and the end of May).

Seasonality has a large effect in slowing the spread of SARS-CoV-2. By May 31, both
observed and total cases would be reduced by a factor of four if only seasonality
mattered. However, in this period, cases would have kept on rising throughout, just at a
much lower pace this is in line with results in \cite{Gavenciak2021}, which our
seasonality measure is based on. Nevertheless, we estimate seasonality to be a
quantitatively important factor determining the evolution of the pandemic, explaining
most of the early changes and 43\% of the cumulative difference by the end of May.

A similar-sized effect---42\% in the decomposition---comes from rapid testing. Here, it
is crucial to differentiate between recorded cases and actual cases. Additional testing
means that additional infections will be recorded which would otherwise remain
undetected. Figure~3a shows that this effect is large and may persist for some time.
Until late April, recorded cases are higher in the scenario with rapid testing alone
when compared to the setting where none of the three mechanisms are turned on. The
effect on total cases, however, is visible immediately in Figure~3b. Despite the fact
that only 10\% of the population performed weekly rapid tests in March on average, new
infections on April~1 would have been reduced by 53\% relative to the scenario without
vaccinations, rapid tests, or seasonality. In Supplementary Material~B.12, we provide a
detailed analysis of whether our results are robust regarding the sensitivity parameters
we assume for rapid tests. Even if we take a pessimistic stance, the effect is only
reduced from 42\% to 38\%.

So why is rapid testing so effective? In order to shed more light on this question,
Figure~3d decomposes the difference in the scenario without rapid tests and the main
specification into the three channels for rapid tests. Tests at schools have the
smallest effect, which is largely explained by schools not operating at full capacity
during our period of study and the relatively small number of students. (18\% of our
population are in the education sector, e.g., pupils, teachers; 46\% are workers outside
the education sector.) Almost 40\% come from tests at the workplace. Despite the fact
that rapid tests for private reasons are phased in only in mid-March, they make up for
more than half of the total effect. The reason lies in the fact that a substantial share
of these tests is driven by an elevated probability to carry the virus, i.e., showing
symptoms of CoViD-19 or following up on a positive test of a household member. The
latter is essentially a form of contact tracing, which has been shown to be very
effective \cite{Contreras2021, Fetzer2021,Kretzschmar2020}. Indeed, a deeper analysis in
Supplementary Material~B.15 shows that the same amount of rapid tests administered
randomly in the population would not have been nearly as effective.

\section*{Discussion and conclusions}

Having quantified the effects of various mechanisms, we now simulate hypothetical
scenarios comparing changes in NPIs and testing regimes. Two of the most contentious
NPIs concern schools and mandates to work from home. In many countries, schools switched
to remote instruction during the first wave, so did Germany. After the summer break,
they were operating at full capacity with increased hygiene measures, before being
closed again from mid-December onward. Some states started opening them gradually in
late February, but operation at normal capacity did not resume until the beginning of
June. Figure~4a shows the effects of different policies regarding schools starting after
Easter, at which point rapid tests had become widely available. We estimate the realized
scenario to have essentially the same effect as a situation with closed schools. Under
fully opened schools with mandatory tests, total infections would have been 6\% higher;
this number rises to 20\% without tests. These effect sizes are broadly in line with
empirical studies (e.g. \cite{Vlachos2021, Berger2021}, see Supplementary Material~B.11
for a comparison). In light of the large negative effects school closures have on
children and parents \cite{Luijten2021, Melegari2021}---and in particular on those with
low socio-economic status---these results in conjunction with hindsight bias suggest
that opening schools combined with a testing strategy would have been beneficial. In
other situations, and particular when rapid test are not available at scale, trade-offs
may well be different.

\begin{figure}[tb]
    \centering
    \includegraphics[width=\textwidth]{../../bld/panel_figures/fig-school-workplace-scenarios.pdf}
    \caption{
        Effects of different scenarios for policies regarding schools and workplaces.
        Blue lines in both figures refer to our baseline scenario; they are the same as
        in Figure~3b. Interventions start at Easter because there were no capacity
        constraints for rapid tests afterwards. For legibility reasons, all lines are
        rolling 7-day averages.
    }
    \label{fig:school_workplace_scenarios}
\end{figure}

Figure~4b shows that with a large fraction of workers
receiving tests, testing at the workplace has larger effects than mandating employees to
work from home. Whether the share of workers working at the usual workplace is reduced
or increased by ten percent changes infection rates by 2.5\% or less in either
direction. Making testing mandatory twice a week---assuming independent compliance by
employers and workers of 95\% each---would have reduced infections by 23\%. Reducing
rapid tests offers by employers to the level of March would have increased infections by
13\%.

Our analysis has shown that during the transition to high levels of vaccination and
possibly thereafter, large-scale rapid testing can substitute for some NPIs. This comes
at a fraction of the cost. A week of the fairly strict lockdown in early 2021 is
estimated to have cost around 50~Euros per capita \cite{Wollmershauser2021}; retail
prices for rapid tests were below one Euro in early June 2021 and below five Euros for
firms. While we do not distinguish between self-administered rapid tests and point of
care rapid tests, the former are likely to play a larger role for indication-driven
testing. Widespread availability at low prices seems important. However, they rely on
purely voluntary participation in a non-public setting. The benefit of point-of-care
rapid tests as a precondition to participate in leisure activities as well as mandatory
tests at the workplace or at school come from screening the entire population. This is
important because disadvantaged groups are less likely to be reached by testing
campaigns relying on voluntary participation (e.g. \cite{StillmanTonin2021}); at the
same time, these groups have a higher risk to contract CoViD-19
\cite{KochInstitut2021a}. Mandatory tests at school and at the workplace will extend
more into these groups. The same goes for individuals who exhibit a low level of
compliance with CoViD-19-related regulations. Compared to vaccinations, rapid testing
programmes allow a much quicker roll-out, making it arguably the most effective tool to
contain the pandemic in the short run.

\section*{Data Availability}

The source code for the epidemiological model is publicly available in this Github
repository: https://github.com/covid-19-impact-lab/sid. The source code of the research
project and the data is available in this repository:
https://github.com/covid-19-impact-lab/sid-germany.

\bibliography{references}

\section*{Acknowledgements (not compulsory)}

The authors are grateful for support by the Deutsche Forschungsgemeinschaft (DFG, German
Research Foundation) under Germany´s Excellence Strategy – EXC 2126/1– 390838866 – and
through CRC-TR 224 (Projects A02 and C01), by the IZA Institute of Labor Economics, and
by the Google Cloud CoViD-19 research credits program.

\section*{Author contributions statement}

Must include all authors, identified by initials, for example: A.A. conceived the
experiment(s),  A.A. and B.A. conducted the experiment(s), C.A. and D.A. analysed the
results.  All authors reviewed the manuscript.

\section*{Additional information}

To include, in this order: \textbf{Accession codes} (where applicable);

\textbf{Competing interests}: The authors declare no competing interests.

\end{document}
