\documentclass[a4paper,11pt]{article}
\usepackage{a4wide}
\usepackage[utf8]{inputenc}
\usepackage[T1]{fontenc}
\usepackage{afterpage,rotating,graphicx,setspace,xcolor}
\usepackage{longtable,booktabs,tabularx}
\usepackage{chngcntr}
\usepackage{eurosym,calc}
\usepackage{amsmath,amssymb,amsfonts,amsthm,delarray}
\usepackage{bm}
\usepackage{bbm}
\usepackage{caption}
\usepackage{subcaption}
\usepackage{xfrac}
\usepackage{listings}
\usepackage{adjustbox}
\usepackage{xr}
\usepackage{xr-hyper}
\usepackage[unicode=true]{hyperref}
\usepackage{afterpage}
\usepackage{clipboard}
% !TeX program = pdflatex
% !TeX TXS-program:compile = txs:///pdflatex/
% !TeX TS-program = pdflatex
% !BIB program = biber
% !TeX TXS-program:bibliography = txs:///biber




%%%%%%%%%%%%%%%%%%%%%%%%%%%%%%%%%%%%%%%%%%%%%%%%
%%  CITATION COMMANDS AND BIBLIOGRAPHY STYLE  %%
%%%%%%%%%%%%%%%%%%%%%%%%%%%%%%%%%%%%%%%%%%%%%%%%


% Science style
\usepackage[style=science, backend=biber, natbib=true, bibencoding=inputenc]{biblatex}

\openclipboard{clipboard-rev1}

\definecolor{darkblue}{rgb}{0, 0, 0.8}
\hypersetup{ colorlinks=true, linkcolor=darkblue, anchorcolor=darkblue,
    citecolor=darkblue, filecolor=darkblue, menucolor=darkblue, runcolor=darkblue,
    urlcolor=darkblue }
\usepackage{enumerate}
\usepackage{caption}
\usepackage{booktabs,caption,rotating}
\usepackage[capposition=top]{floatrow}
% \usepackage{bbm}
\usepackage{epstopdf}
\usepackage{longtable, tabularx, array}
\usepackage{colortbl}
\usepackage{placeins}

% \usepackage[british]{babel} \usepackage[useregional]{datetime2}
% \DTMlangsetup[en-GB]{showdayofmonth=false}

\usepackage{subcaption}
\usepackage{eurosym}

% \setlength{\marginparwidth}{2cm}
\usepackage[colorinlistoftodos, backgroundcolor=white, prependcaption, textsize=tiny,
    disable]{todonotes}


%%%%PROPOSITION AND PROOF ENVIRONMENTS%%%%
\theoremstyle{plain}
\newtheorem{result}{Result}
\newtheorem{assumption}{Assumption}

\DeclareMathOperator*{\argmin}{arg\,min} \DeclareMathOperator*{\argmax}{arg\,max}

% don't let footnotes split across pages
\interfootnotelinepenalty=10000 \widowpenalty=10000 \clubpenalty=10000

\hyphenation{bet-ween}
\hyphenation{with-out}

\frenchspacing

\newlength{\parindentaux}
\setlength{\parindentaux}{\parindent}

\usepackage{enumitem}
\setlist[itemize]{%
    leftmargin=\parindentaux, listparindent=\parindentaux }

% \usepackage{calc}  % Needed to be able to calculate, e.g., sum of dimensions
%\usepackage{ifthen}

\makeatletter
\newcommand*\@myenumerate{enumerate} \newcommand*\@myitemize{itemize}
\newcommand{\comment}[1]{{\color{black!60}#1}} \newcommand{\response}[1]{{%
		\color{black}%
		\par\frenchspacing%
		\ifx\@currenvir\@myenumerate%
		\addtolength{\leftskip}{-\parindentaux}%
		\rm\noindent #1\par%
		\addtolength{\leftskip}{\parindentaux}%
		\else%
		\ifx\@currenvir\@myitemize%
		\addtolength{\leftskip}{-\parindentaux}%
		\rm\noindent #1\par%
		\addtolength{\leftskip}{\parindentaux}%
		\else%
		\rm\noindent #1\par%
		\fi%
		\fi%
}}
\makeatother

\externaldocument{paper}

\renewcommand{\theequation}{\roman{equation}} \renewcommand{\thetable}{\Roman{table}}
\renewcommand{\thefigure}{\Roman{figure}} \renewcommand{\thesection}{\Roman{section}}


\setlength{\parindent}{0ex}

\bibliography{references.bib}

\begin{document}

\title{\large Reply to the Editor's and Reviewers' comments on: \\[2ex]
    \LARGE The Effectiveness of Testing, Vaccinations, and Contact Restrictions for
    Containing the Covid-19 Pandemic\\[2ex]
    \large Submitted to Scientific Reports \\(Submission ID
    db917d3c-b56a-4129-8eb1-a62a716aaede)\\[-6ex]
}

\author{}
\date{}

\maketitle

We would like to thank the editors and reviewers for their work. We comment briefly on the points made by Reviewer 1.

\section*{Reviewer 1}
\comment{
In the manuscript presented here, the authors describe model calculations on the quantitative influence of non-pharmaceutical measures, vaccination and rapid testing campaigns on the containment of the spread of SARS-CoV-2 in Germany. They also consider factors such as seasonality or virus variants with different transmission profiles.
Modeling of the kind described here has now been carried out and published in large numbers by other groups. Unfortunately, the authors here refer exclusively to calculations of the spread of the SARS-CoV-2 alpha variant in Germany. In contrast, the much more extensive spread of the delta variant and the even more significant omicron spreads are not considered.

This is a major weakness of this manuscript. Although the descriptions of the methods used seem adequate from my perspective, the analysis adds very little value. Expanding the manuscript to include data on the delta and omicron variants would greatly enhance its significance.

Despite this considerable weakness, I see no obstacles to publication, provided that the authors nevertheless wish to share the limited significance of their study with the scientific community.
\vskip1ex

\response{%
    We would like to thank the Reviewer for his/her comments. We would like to make a couple of points:

    \begin{itemize}
        \item The manuscript was finalized in June 2021 and submitted to \textit{Scientific Reports} in August 2021. At that time, $\delta$ played a minor role in Germany and omicron was not known yet.
        \item The time period includes the wild type and the $\alpha$ variant. The framework for adding different virus types is there.
        \item In our view, our scholarship lies in the construction of the model and estimating its parameters using available data, thereby demonstrating the effectiveness of different strategies at a particular point in time. The latter may well be different for other types and omicron in particular, where vaccines loose much of their power regarding infection dynamics.
        \item Extending the analysis in such a way would be a straightforward task with our model: The code is all open source, well documented, and available free of charge. All features required to extend the period of analysis to further variants are included in the model. Anyone who wants to collect the additional data and estimate the model is free to do so.
    \end{itemize}

    In the light of these points, we have refrained from re-estimating the model. We have not changed the manuscript.
}

}
~\\[4ex]


\begin{refcontext}[sorting=none]  % Sort BIBLIOGRAPHY by appearance.
    \printbibliography[heading=bibintoc]
\end{refcontext}


\end{document}
