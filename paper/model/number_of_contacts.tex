\subsection{Modeling Numbers of Contacts}
\label{sec:number_of_contacts}

Consider a hypothetical population of 1,000 individuals in which 50 were infected with a
novel infectious disease. From this alone, it is impossible to say whether only
those 50 people had contact with an infectious person and the disease has an infection
probability of 1 in each contact or whether everyone met an infectious person but the
disease has an infection probability of only 5 percent per contact. SEIR models do not
distinguish contact frequency from the infectiousness of each contact and
combine the two in one parameter that is not invariant to social distancing policies.

To model social distancing policies, we need to disentangle the effects of the number of
contacts of each individual and the effect of policy-invariant infection probabilities
specific to each contact type. Since not all contacts are equally infectious, we
distinguish different contact types.

The number and type of contacts in our model can be easily extended. Each type of
contacts is described by a function that maps individual characteristics, health states
and the date into a number of planned contacts for each individual. This allows to model
a wide range of contact types.

In our empirical application we distinguish the following types of contacts:

\begin{itemize}
    \item Households: Each household member meets all other household members every day.
    The household sizes and structures are calibrated to be representative for Germany.

    \item Random non-work contacts: Each person has contacts with randomly drawn other
    people which are assortative with respect to region and age group. This contact type
    reflects contacts during pure leisure activities as well as non leisure activities
    such as grocery shopping or medical appointments.

    \item Recurrent daily non-work contacts: Each person has daily recurring contacts
    which allows to model close relationships other than families between individuals.

    \item Recurrent weekly non-work contacts: Each person has weekly recurring contacts
    like sports groups or other weekly activities.

    \item Random work contacts: Each working adult has contact to randomly drawn other
    people at work.

    \item Recurrent daily work contacts: Each working adult meets other workers every
    day. This is meant to capture work colleagues.

    \item Recurrent weekly work contacts: Each working adult meets other workers once
    per week. We randomize over the days on which the meetings take place. This is meant
    to capture meetings with clients, superiors or other colleagues which happen
    infrequently.

    \item Schools: Each student meets all of his classmates every day. Class sizes are
    calibrated to be representative for Germany and students have the same age. Schools
    are closed on weekends and during vacations, which vary by states. School classes
    also meet three pairs of teachers every school day. The pairs are meant to represent
    interactions between teachers.

    \item Preschools: Children who are at least three years old and younger than six may
    attend preschool. Each group of nine children interacts with the same two adults
    every day. The children in each group are of the same age. The remaining mechanics
    are similar to schools.

    \item Nurseries: Children younger than three years may attend a nursery and interact
    with one adult. The age of the children varies within groups. The remaining
    mechanics are similar to schools.
\end{itemize}

The number of random and recurrent contacts at the workplace, during leisure activities
and at home is calibrated with data provided by \citet{Mossong2008}. For details see
Section~\ref{sec:number_of_contacts}. In particular, we sample the number of contacts or
group sizes from empirical distributions that sometimes depend on age. It is also
possible to use economic or other behavioral models to predict the number of contacts.

Theoretically, each contact type can have its own infection probability. However, to
reduce the number of free parameters and thus avoid a potential over-fitting we only
estimate different infection probabilities for the areas work, school, preschool and
nurseries, households and other contacts.
