\subsection{Testing} % (fold)
\label{sub:testing}

Having a realistic model of PCR and rapid tests is crucial for two reasons: Firstly,
only via a testing model the simulated infections from the model can be made comparable
to official case numbers. Secondly, individuals with undetected or not yet detected
infections are an important driver of the pandemic.

In principle, our modeling approach is flexible enough to incorporate mechanistic
test demand, allocation and processing models. However, there is not enough data
available to calibrate such a mechanistic model.

Therefore we aim for a simpler model of PCR and rapid tests that can be calibrated with
available data on test demand and availability and and, nevertheless, can produce
a share of undetected cases that varies over time and across age groups and agrees with
other estimates over the time periods where they are available.

PCR tests are modeled since the beginning of the simulation and determine whether a
infections is officially recorded. Rapid tests are only added at the beginning of
2021. Positive rapid tests do not enter official case numbers directly, but most people
with a positive rapid tests demand a confirmatory PCR test. However, positive rapid test
can have a strong effect on the infection dynamics because they trigger contact
reductions.

During 2020 people can demand PCR tests either because they have symptoms or randomly.
The probability that a PCR test is performed in each of the two situations depends on
the number of new infections and the number of available tests. It thus varies strongly
over time and is unknown.

To distribute the correct number of PCR tests among symptomatic and asymptomatic
infections without knowing explicit test demand probabilities, we use the following
approach: First we calculate the total number of positive PCR tests by multiplying the
number of newly infected individuals with an estimate of the share of detected cases
from the Dunkelzifferradar project. \footnote{The Dunkelzifferradar project publishes daily estimates of the dark figure of infections under \url{https://covid19.dunkelzifferradar.de/}}. Next we determine how many of these tests
should go to symptomatic and asymptomatic individuals from data by the Robert Koch
Institut.\comment[id=J]{add source} Then we sample the individuals to which those test
are allocated from the pools of symptomatic and asyptomatic infected but not yet tested
individuals.

Sampling uniformly from the pool of symptomatic individuals ensures that age groups
who are more likely to develop symptoms are also more likely to receive tests. Thus the
share of detected cases is much higher for adults than for children in time
periods where many tests are done because of symptoms which is in line with the
estimates from the literature.\comment[id=J]{need source}

At the beginning of 2021, two challenges arise: Firstly, the externally estimated share
of detected cases from the Dunkelzifferradar can no longer be used because it is based
on case fatality rates which drastically change due to vaccinations. Secondly, rapid
tests become available at a large scale.

We solve the first challenge by assuming that the share of detected cases would have
remained on the level it reached before Christmas if rapid tests had not become
available. While this is only an approximation to reality, changes in the share of
detected cases that would have happened without rapid tests are very likely to be small
compared to the changes caused by rapid tests.

The second challenge is solved by mechanistic rapid test demand models for the
workplace, schools and by private individuals. The calibration of these models is
described in Section~\ref{rapid_test_demand}.
Figure~\ref{fig:antigen_tests_vaccinations} that the number of performed rapid tests in
the model fits the empirical data well (where empirical data is available).

In contrast to PCR tests, rapid tests are not perfect and can be falsely positive or
falsely negative. While the specificity of rapid tests is constant over time, the
their sensitivity strongly depends on whether the tested individual is already
infectious and if so for how long he has been infectious. Before the onset of
infectiousness the sensitivity is very low (35 \%). On the first day of infectiousness
it is much higher (88 \%) but still lower than during the remaining infectious
period (92 \%). After infectiousness stops, the sensitivity drops to 50 \%.

Modeling the diagnostic gap before and at the beginning of infectiousness is very
important to address concerns that rapid tests are too unreliable to serve as screening
devices.

We do not distinguish between self administered and rapid tests and those that are
administered by medical personal. While there were concerns that self administered tests
are less reliable, a recent study has found no basis for this
concern.\comment[id=J]{cite charite study}

While rapid tests do not directly enter official case numbers, many positively tested
individuals confirm their rapid test result with a PCR test.\comment[id=J]{Need to
link to data appendix once the number is there and maybe mention the number}.
Importantly, those PCR tests are made in addition to the tests that would have been
done otherwise. Section~\ref{subsec:appendix_share_known_cases} discusses the effect
of rapid tests on the share of detected cases.
