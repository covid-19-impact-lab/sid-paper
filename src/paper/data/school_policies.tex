% educ policies

% summary
For nurseries, preschools and schools we implement vacations as announced by the German
federal states as well as school closures, emergency care and rotating schedules where
only one half of students attends every other week or day. An approximation of the share
of contacts still taking place with the different school regulations can be found in
Figure~\ref{fig:school_multiplier}. Note that schooling policies differ between states
and usually involve rules that depend on local incidences. We simplify these rules to one
federal policy from the federal incidence and the policies of the three most populous
federal states (North Rhine-Westphalia, Bavaria and Baden-Württemberg). The testing
policies for schools are described in Section~\ref{subsec:rapid_test_demand}.

% school
Until November schools were open normally. Starting in November, we assume that increased
hygiene measures were taken. Schools stayed open until mid December.
% mid Dec until mid Jan
From mid December until January 10 schools closed and only offered so called ``emergency
care'' for young children whose parents could credibly demonstrate that both had to work
and had no other child care arrangement. Approximately 25\% of primary school children and 5\%
of secondary students attended school as a result.
% Jan and Feb
After January 10 when parents had returned to work the rules for emergency care were
relaxed and approximately a third of primary school children and 10\% of secondary students
attended school as a result. In addition, graduating classes (most adolescents between 16
and 18) were allowed to return to school in a rotating scheme where each class was split
in two groups. Relying on anecdotal evidence we assume that the groups rotate on a daily
basis.% mid Feb to mid March
Starting on February 22 primary school children were also allowed to return to school on
a rotating basis until mid March.
% mid March to Easter
We summarize the school policy from mid March until Easter as all students being on a
rotating school schedule. In addition, children that qualify for emergency care also
attend on days where their group is scheduled to not attend school physically.
% after Easter
After the Easter break schools were mostly closed again. Part of this was a federal law,
the so called ``Bundesnotbremse'' \citep{Notbremse2021} that set rules for schools based
on local incidences that were binding at the time. As a result, most states adjusted
their schooling policies and during April most schools were closed with emergency care
arrangements as in the time from January 10 to February 21. As cases fell schools were
allowed to gradually open. We summarize this as students being on the same rotating
schedule as from mid March to Easter starting on May 1 \citep{schoolBW, schoolBWa,
schoolBWb, schoolBY, schoolBYa, schoolNRW, schoolNRWa, schoolNRWb, schoolNRWc}.

% young educ
The policies for preschools and nurseries are similar to the school policies but simpler.
Until November children attended completely normally, starting in November with increased
hygiene measures. Nurseries and preschools stayed open until mid December. From mid
December until January 10, nurseries and preschools were nearly completely closed. If
parents could credibly demonstrate that both parents work in systemically relevant
professions and no other child care arrangement was possible, nurseries and preschools
offered so called ``emergency care''. We assume 10\% of children qualified and used
emergency care during this time. After January 10 when parents had returned to work the
rules for emergency care were relaxed and we assume a third of children attended nursery
and preschool. This policy stayed in place until February 20. Afterwards, preschools and
nurseries were open normally (maintaining increased hygiene measures) until mid March.
Then during the third wave the restrictions February were put back into place until end
of April when nurseries and preschools opened again and stayed open for the rest of our
simulation period - maintaining increased hygiene measures. \citep{KiTa_BY, KiTa_NRW,
KiTa_BW, KiTa_BWa, KiTa_BYa, schoolBWb}.
