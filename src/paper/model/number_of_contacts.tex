\subsection{Modeling Numbers of Contacts}
\label{sec:number_of_contacts}

Consider a hypothetical population of 1,000 individuals in which 50 were infected with a
novel infectious disease. From this alone, it is impossible to say whether only those 50
people had contact with an infectious person and the disease has an infection probability
per contact ($\beta$) of one or whether everyone met one infectious person but the
disease has an infection probability of only 5 percent per contact. SEIR models do not
distinguish between the number of contacts ($\eta$) and the infectiousness of each
contact ($\beta$). Instead, they combine the two into one parameter that is not invariant
to social distancing policies.

To model social distancing policies, we need to disentangle the effects of the number of
contacts of each individual and the effect of mostly policy-invariant infection
probabilities specific to each contact type.

The number and type of contacts in our model can be easily extended. Each type of
contacts is described by a function that maps individual characteristics, health states
and the date into a number of planned contacts for each individual. This allows to model
a wide range of contact types.

In our empirical application we distinguish the following  contact types that
are depicted in Figure~\ref{fig:model_contacts_infections} and can be further grouped
in the categories household, work, education and others:

\begin{itemize}
    \item Households: Each household member meets all other household members every day.
    % Household sizes and structures are calibrated to be representative for Germany.


    \item Recurrent work contacts: These capture contacts with coworkers, repeating
          clients and superiors. Some of these recurrent contacts take place on every
          workday, others just once per week. The contacts are assortative in
          geographical location and age.

    \item Non recurrent work contacts: Working adults have contacts with randomly drawn
          other people, which are assortative in geographical location and age.

    \item Schools: Each student meets all of his classmates every day. Class sizes are
    calibrated to be representative for Germany and students have the same age and mostly
    live in the same county. Schools are closed on weekends and during vacations, which
    vary by states. School classes also meet six teachers every day and some of the
    teachers meet each other.

    \item Preschools: Children who are between three and six years old attend preschool.
    Each group consists of nine children of mixed ages and two adults who live mostly in
    the same county. They all meet each other every work day when there are no vacations.

    \item Nurseries: Children younger than three years may attend a nursery and interact
    with one adult. The age of the children varies within groups but all live in the same
    county. They all meet each other every work day when there are no vacations.

    \item Non recurrent other contacts: Contacts with randomly drawn other
    people, which are assortative with respect to geographic location and and age group.
    This contact type reflects contacts during leisure activities, grocery shopping,
    medical appointments, etc..

    \item Recurrent other contacts representing contacts with friends neighbours or
    family members who do not live in the same household. Some of these contacts happen
    daily, others only once per week. They are assortative in geographic location and
    age.


\end{itemize}

The number of random and recurrent contacts at the workplace, during leisure activities
and at home is calibrated with data provided by \citet{Mossong2008}. For details see
Section~\ref{subsec:data_number_of_contacts}. In particular, we sample the number of
contacts or group sizes from empirical distributions. It would also be possible to use
economic or other behavioral models to predict the number of contacts.

% Theoretically, each contact type can have its own infection probability. However, to
% reduce the number of free parameters and thus avoid a potential over-fitting we only
% estimate different infection probabilities for the areas work, school, preschool and
% nurseries, households and other contacts.
