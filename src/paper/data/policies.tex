\subsection{Policies}
\label{subsec:policies}

Our policies (denoted by $\rho$) usually affect one of three contact types: education,
work and other contacts.
% households
Germany had no policies limiting contacts within households so there are no policies on
them in our model.\footnote{Household contacts can, however, be reduced when individuals
quarantine themselves after developing symptoms, for example. This happens to a lesser
degree than other contacts to capture difficulties in isolation within the home.}

% educ models
For nurseries, preschools and schools we implement vacations as announced by the German
federal states as well as school closures, emergency care and rotating schooling
schedules where only one half of students attends every other week or day.
% For the moment we ignore that lack of childcare leads working parents to stay home.
An approximation of the share of contacts still taking place with the different school
regulations can be found in Figure~\ref{fig:school_multiplier}.


% work
For work contacts we use the reductions in work mobility reported by the Google Mobility
Data \citep{Google2021} to calibrate the reduction in physical work contacts
($\rho_{w,\:attend,\:t}$). Reductions in work contacts are not random but governed
through a work contact priority where the policy changes the threshold below which
workers stay home. Figure \ref{fig:work_multiplier} shows the share of workers that go to
work over time at the federal German level. We use the data on the state level to account
for local holidays and differences in state regulations.
% hygiene umltiplier
In addition, for both work and school contacts we assume that hygiene measures (such as
masks, ventilation and hand washing) became more strict and more conscientiously observed
in November 2020, leading to a reduction of 33\% in the number of contacts with the potential
to transmit Covid-19 ($\rho_{hygiene}$).


\begin{figure}[ht]
    \centering
    \begin{subfigure}[b]{0.425\textwidth}
        \centering
        \includegraphics[width=\textwidth]{figures/results/figures/data/school_multiplier_comparison}
        \caption{{School Attendance}}
        \label{fig:school_multiplier}
    \end{subfigure}
    \hfill
    \begin{subfigure}[b]{0.425\textwidth}
        \centering
        \includegraphics[width=\textwidth]{figures/results/figures/data/work_multiplier_since_sep}
        \caption{Work Attendance}
        \label{fig:work_multiplier}
    \end{subfigure}

    \vskip3ex

    \caption{The Contact Reduction Effects of School and Work Attendance Policies}

    \floatfoot{\noindent \textit{Note:} The left figure shows the approximate share of
    school contacts taking place with and without vacations factored in. In contrast
    to other policies, school policies are not implemented via multipliers but as
    mechanistic models (e.g. split class approaches in with emergency care). For the
    above plot we assigned approximate multipliers to those policies. The figure is,
    thus, only an
    illustration that shows the approximate share of contacts taking place compared to
    the pre-pandemic level with and without vacations.
    The right figure shows the work mobility as
    reported by \cite{Google2021}. We take this as a proxy of the share of workers who
    still have physical work contacts ($\rho_{w,\:attend,\:t}$). The figure interpolates
    over weekends as we handle weekend effects through information on work on weekends in
    the German census data we use. The figure shows the share for Germany as a whole. To
    capture the effect that local policies, school vacations, etc. have on work contacts
    we use the data on the state level to determine which workers go to work depending on
    the state they live in.}
    \label{fig:multipliers}
\end{figure}

Lastly, for the other contacts category ($\rho_{other,\:t}$) we could not calibrate the
policies from data but estimated the policy effects. The estimation and values are
detailed in Section~\ref{subsec:estimated_params} and Figure~\ref{fig:other_multiplier}.

\FloatBarrier

