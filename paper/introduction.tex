\section{Introduction}
\label{sec:introduction}

% setting
The first wave of the Covid-19 pandemic prompted strict lockdowns and restrictions
across the world. The reaction to second waves consisted of more targeted policy measures such as splitting school classes, closing restaurants and encouragement of home office. To combat the third and fourth waves, that were triggered by the spread of more infectious variants, governments relied on the same targeted measures and previously non-available tools such as large scale rapid testing and vaccinations.

While this multitude of policy measures has led to declining case numbers in most countries, it becomes harder and harder to evaluate the contribution each policy had on the overall outcome. Moreover, the longer the pandemic lasts, the more important it becomes to accurately model heterogeneities in contact behavior and a realistic network of recurrent contacts because some sub-populations might develop herd immunity even though the full population does not.

The workhorse model of epidemiology, the S(E)IR model as well as many recent extensions to it are not up to this task. We develop an agent-based simulation model that has been designed from the ground up to predict and quantify the effects of contact reducing policies, vaccinations and testing strategies in a constantly changing policy environment. It has the following features:

\begin{enumerate}
    \item At the core of the model, people meet people based on a matching algorithm. We
          distinguish various types of contacts. The contact types are households,
          leisure activities, schools, preschools, and nurseries and several types of
          contacts at the workplace. Contact types can be random or recurrent and vary
          in frequency and infectiousness.

    \item Policies can be implemented as shutting down contact types entirely or
          partially. The reduction of contacts can be random or systematic. For example,
          it is possible to implement split class schooling where only one half of each
          class attends and the attending half switches on a weekly bases. The extent to which contacts are reduced can be calibrated from observed data or estimated inside the model.

    \item Infection probabilities vary across contact types and reflect properties of
          the contact like the location (indoor/outdoor) and the kind of interaction
          (duration, physical contact). The probabilities are independent from the
          number of contacts and, thus, policy-invariant.

    \item We distinguish detected and undetected cases. The share of detected cases varies over time and across age groups. Moreover, it can be influenced by rapid testing policies.
    \item High quality Python code for the model is freely available on Github, well
          documented and very flexible\footnotemark. We are actively looking for
          researchers who want to use our model for their projects and apply it to other
          contexts.
\end{enumerate}

\footnotetext{
    The code can be found under \url{https://github.com/covid-19-impact-lab/sid} and the
    documentation with tutorials and background information under
    \url{https://sid-dev.readthedocs.io/}.
}

The model achieves a good fit on German data of infection and fatality rates, even though most parameters are calibrated from the literature and observable datasets and only a few parameters are estimated inside the model. At accurately predicts the rise of the B117 mutation in spring 2020. It can also explain the surprising decline of case numbers at the end of April, without making ad hoc assumptions on behavioral changes at the time.

The model has previously been applied to predict the effect of schooling policies, contact tracing policies over the Christmas holidays and the effect of work from home (\citet{Dorn2020a, Gabler2020, Gabler2021}).


\textcolor{blue}{Summary of the main results}


The remainder is structured as follows: In Section~\ref{sec:literature_review}, we give a short overview of epidemiological models. Section ~\ref{sec:background} summarizes the policy environments and dynamic of the Covid-19 pandemic in Europe, with a special focus on Germany. We continue with a general description of our modelling framework  as well as a specialization to the German context in section~\ref{sec:model}. Section~\ref{sec:calibration_and_estimation} describes our empirical datasets, sources of calibrated parameters and estimation procedure. \textcolor{blue}{desribe result section}. We conclude in section~\ref{sec:conclusion}.
