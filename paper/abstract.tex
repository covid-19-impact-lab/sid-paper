\noindent

% Abstracts of Research Articles and Reports should explain to the general reader why the research was done, what was found and why the results are important. 

% They should start with some brief BACKGROUND information: a sentence giving a broad introduction to the field comprehensible to the general reader, and then a sentence of more detailed background specific to your study. 

Background: 

Spread disease, initially NPIs, now testing and vaccines; seasonality unclear.

% This should be followed by an explanation of the OBJECTIVES/METHODS and then the RESULTS. 

Objective: Provide a model that allows studying these things in conjunction, allowing for different virus strains.

Results: Along the transition to vaccination-induced herd immunity, testing is most effective, also thanks to family structures. Seasonality contributed its share.

% The final sentence should outline the main CONCLUSIONS of the study, in terms that will be comprehensible to all our readers. The Abstract is distinct from the main body of the text, and thus should not be the only source of background information critical to understanding the manuscript. Please do not include citations or abbreviations in the Abstract. 

Conclusions: Frequent rapid testing should remain part of strategies to contain CoViD-19.

% The abstract should be 125 words or less.

\vspace{1cm}
\noindent JEL Classification: C63, I18

\noindent Keywords: Covid-19, agent based simulation model, public health measures
