\subsection{Medical Parameters}
\label{sec:medical_params}

This section discusses the medical parameters used in the model, their sources and how we
arrived at the distributions used in the model.\footnotemark See
Figure~\ref{fig:model_disease_progression} for a summary of our disease progression model.

\footnotetext{Additional information can be found in the
\href{https://sid-dev.readthedocs.io/en/latest/reference_guides/epi_params.html}{online
documentation}.}

% Duration until Infectiousness and Symptoms ----------------------------------

% Latency Period
The first medical parameter we need is the length of the period between infection and the
start of infectiousness, the so called latent period. We infer it from two other measures
that are more common in the medical literature: Firstly, the time between infection and
the onset of symptoms, the incubation period. Secondly, the time between the start of
infectiousness and the onset of symptoms. We assume that the latency period is the same
for symptomatic and asymptomatic individuals.

% Asymptomatic Share
Once individuals become infectious a share of them goes on to develop symptoms while
others remain asymptomatic. We rely on data by \cite{Davies2020} for the age-dependent
probability to develop symptoms. It varies from 25\% for children and young adults to
nearly 70\% for the elderly.

% Incubation Period
The incubation period is usually estimated to be two to twelve days. A meta analysis by
\citet{McAloon2020} comes to the conclusion that ``The incubation period distribution may
be modeled with a lognormal distribution with pooled $\mu$ and $\sigma$ parameters (95\%
CIs) of 1.63 (95\% CI 1.51 to 1.75) and 0.50 (95\% CI 0.46 to 0.55), respectively.'' For
simplicity we discretize this distribution into four bins.

% Length of the presymptomatic stage
The European Centre for Disease Prevention and Control \citep{ECDC2021} reports that
people become infectious between one and two days before symptoms start.\footnotemark

\footnotetext{This is similar to \citet{He2020} and in line with \citet{Peak2020}.}

Taking these estimates together, we arrive at a latent period of one to five days.

% Duration of Infectiousness ----------------------------------------------------------

We assume that the duration of infectiousness is the same for both symptomatic and
asymptomatic individuals as evidence suggests little differences in the transmission
rates between symptomatic and asymptomatic patients (\citet{Yin2020}) and that the viral
load between symptomatic and asymptomatic individuals are similar (\citet{Zou2020},
\citet{Byrne2020}, \citet{Singanayagam2020}). Our distribution of the duration of
infectiousness is based on \citet{Byrne2020}. For symptomatic cases they arrive at zero
to five days before symptom onset (see their figure 2) and three to eight days of
infectiousness afterwards.\footnote{Viral loads may be detected much later but eight days
seems to be the time after which most people are culture negative, as also reported by
\citet{Singanayagam2020}.} Thus, we arrive at 0 to 13 days as the range for
infectiousness among individuals who become symptomatic (see also figure 5).

% Duration of Symptoms ----------------------------------------------------------

We use the duration to recovery of mild and moderate cases reported by \cite[Figure~S3,
Panel~2]{Bi2020} for the duration of symptoms for non-ICU requiring symptomatic cases. We
only disaggregate by age how likely individuals are to require intensive
care.\footnote{The length of symptoms is not very important in our model given that
individuals mostly stop being infectious before their symptoms cease.}

% Time from Symptom Onset to Admission to ICU ------------------------------------------

For the time from symptom onset until need for intensive care we rely on data by the US
CDC \citep{Stokes2020} and the OpenABM Project \citep{Hinch2021}.

%Table~\ref{tab:symptomatic-to-ICU} shows our derivations for the probabilities of
%requiring intensive care per age group.
%
%\begin{table}[tb]
%    \caption{Shares of symptomatic patients who will require ICU care by age groups.}
%    \label{tab:symptomatic-to-ICU}
%    \centering
%
%    \begin{tabular}{ll}
%        \toprule
%        Age Group & Share \\
%        \midrule
%        0-9 & 0.00005 \\
%        10-19 & 0.00030 \\
%        20-29 & 0.00075 \\
%        30-39 & 0.00345 \\
%        40-49 & 0.01380 \\
%        50-59 & 0.03404 \\
%        60-69 & 0.10138 \\
%        70-79 & 0.16891 \\
%        80-100 & 0.26871 \\
%        \bottomrule
%    \end{tabular}
%
%    \tablenotes{The data is taken from \citet{Stokes2020} and citet{Hinch2021}.}
%
%\end{table}

For those who will require intensive care we follow \citet{Chen2020} who estimate the
time from symptom onset to ICU admission as 8.5 $\pm$ 4 days. This aligns well with
numbers reported for the time from first symptoms to hospitalization:
\citet{Gaythorpe2020} report a mean of 5.76 with a standard deviation of 4. This is also
in line with the duration estimates collected by the Robert Koch Institute
\citep{RKI2021b}. We assume that the time between symptom onset and ICU takes 4, 6, 8 or
10 days with equal probabilities. As we do not model nursing homes, do not focus on
matching deaths and do not use the number of individuals in intensive care to estimate
our parameters, these numbers are not important for our empirical results.


% Death and Recovery from ICU ----------------------------------------------------------

We take the survival probabilities and time to death and time until recovery from
intensive care from \citet{Hinch2021}. They report time until death to have a mean of
11.74 days and a standard deviation of 8.79 days. To match this approximately we
discretize that 41\% of individuals who will die from Covid-19 do so after one day in
intensive care, 22\% day after 12 days, 29\% after 20 days and 7\% after 32 days. Again,
we rescale this for every age group among those that will not survive. For survivors the
OpenABM Project reports a mean duration of 18.8 days until recovery and a standard
deviation of 12.21 days. We discretize this such that of those who recover in intensive
care, 22\% do so after one day, 30\% after 15 days, 28\% after 25 days and 18\% after 45
days.

\FloatBarrier
