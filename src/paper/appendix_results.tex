\section{Additional Results}\comment[id=K]{Philipp suggests to do a sensitivity analysis
for the seasonality}
\label{sec:additional_results}

\subsection{Simulated vs. Empirical Data}
\label{subsec:fit_results}

This compares simulated data from our model with empirical data from Germany. We look at
observed infections, fatality rates, the spread of the B117 mutation, vaccinations and
rapid test demands. Where available we do not only look at aggregated statistics but also
analyze the model fit for age groups and federal states.\comment[id=J]{summarize the fit}


\begin{figure}[ht]
  \centering
  \includegraphics[width=\textwidth]{figures/results/figures/scenario_comparisons/combined_fit/full_new_known_case_with_single_runs}
  \caption{Fit Over the Full Simulation Time Frame with Single Simulation Runs}
  \floatfoot{\noindent \textit{Note:} The figure shows the weekly incidence rates per
  100,000 people for the reported simulated infections rates. The mean infection rate is
  the thick blue line. Single simulation runs are plotted in lighter and thinner lines.
  The official case numbers as reported by the Robert-Koch-Institut are plotted in black.
  The fit is overall very good. The higher the mean incidence and the stronger the growth
  the more variance there is between simulation runs. We averaged over 30 simulation
  runs.}
  \label{fig:aggregated_fit2}
\end{figure}

\begin{figure}[ht]
  \centering
  \includegraphics[width=\textwidth]{figures/results/figures/scenario_comparisons/combined_fit/full_newly_infected_with_single_runs}
  \caption{Development of the Total Infections Over the Full Simulation Time Frame with
  Single Simulation Runs}
  \floatfoot{\noindent \textit{Note:} The figure shows the true weekly incidence rates
  per 100,000 people, including undetected cases. The mean infection rate is the thick
  blue line. Single simulation runs are plotted in lighter and thinner lines. The higher
  the mean incidence and the stronger the growth the more variance there is between
  simulation runs. We averaged over 30 simulation runs.}
  \label{fig:newly_infected_in_baseline}
\end{figure}




\begin{figure}[ht]
  \centering
  \includegraphics[width=\textwidth]{figures/results/figures/incidences_by_group/age_group_rki/full_combined_baseline_new_known_case}
  \caption{Simulated and Empirical Infections by Age Group}
  \floatfoot{\noindent \textit{Note:} The figure shows the weekly incidence rates per
  100,000 people for the reported versus the simulated infections rates for different age
  groups. The age group of individuals above 80 needs to be interpreted with caution
  because our synthetic population only includes private households, i.e. nursing homes
  are not represented in our model. They accounted for many cases and deaths in the
  winter of 2020 and many 80 to 100 year olds live in these facilities. However, the
  official data does not contain information on whether cases were nursing home
  inhabitants or not. We averaged over 30 simulation runs.}
  \label{fig:age_group_fit}
\end{figure}


\begin{figure}[ht]
  \centering
  \includegraphics[width=\textwidth]{figures/results/figures/incidences_by_group/state/full_combined_baseline_new_known_case}
  \caption{Simulated and Empirical Infections by Federal State}
  \floatfoot{\noindent \textit{Note:} The figure shows the weekly incidence rates per
  100,000 people for the reported versus the simulated infections rates for different
  federal states. We averaged over 30 simulation runs.}
  \label{fig:state_fit}
\end{figure}


\begin{figure}[ht]
  \centering
  \includegraphics[width=\textwidth]{figures/results/figures/scenario_comparisons/combined_fit/full_ever_vaccinated}
  \caption{Share of Vaccinated Individuals}
  \floatfoot{\noindent \textit{Note:} The figure shows the rate of individuals that are
  vaccinated in our synthetic population versus in the general German population. Note
  that we excluded the vaccinations that were given to nursing homes, approximately the
  first percent of the German population that were vaccinated. Overall, our model covers
  a time frame that goes from zero vaccinated individuals to a state where over 40\% of
  the population are vaccinated. Our vaccinations work imperfectly but we do not model
  different vaccines nor do we distinguish between first and second shot.}
  \label{fig:fit_vaccinations}
\end{figure}


\begin{figure}[ht]
  \centering
  \includegraphics[width=\textwidth]{figures/results/figures/scenario_comparisons/combined_fit/full_r_effective_with_single_runs}
  \caption{Effective Replication Number $R_t$ in the Model and as Reported by the
  Robert-Koch-Institute}
  \floatfoot{\noindent \textit{Note:} The figure shows the effective replication number
  ($R_t$) as reported by the RKI and as calculated in our model. The $R_t$ gives the
  average number of new infections caused by one infected individual. The $R_t$ in our
  model broadly follows the $R_t$ reported by the RKI. Two trends stand out. Firstly, the
  RKI's $R_t$ drops faster in November. This could be due to a change in the testing
  policy that focused tests on the elderly when the second wave hit Germany and led to a
  decline in the overall share of detected cases. The second difference is from mid
  February to mid March where the RKI's reported $R_t$ increased more rapidly than that
  in our model. Here the opposite effect can be expected. During this time rapid tests
  increased strongly leading to more cases being detected. In the short term this leads
  an $R_t$ estimation that is based on detected cases to overestimate the replication
  number.}
  \label{fig:fit_r_effective}
\end{figure}


\begin{figure}[ht]
  \centering
  \includegraphics[width=\textwidth]{figures/results/figures/scenario_comparisons/combined_fit/full_share_b117_with_single_runs}
  \caption{Share of B.1.1.7 in the Model and as Reported by the Robert-Koch-Institute}
  \floatfoot{\noindent \textit{Note:} The figure shows the share of B.1.1.7 as
  reported by the RKI and as calculated in our model. We only introduce a few cases over
  the cause of January. From then B.1.1.7 takes over endogenously through its increased
  infectiousness. We model no other features of B.1.1.7. At most we introduce 0.75 cases per 100,000 inhabitants.}
  \label{fig:fit_share_b117}
\end{figure}




\FloatBarrier


\subsection{Share of Cases that are Detected}
\label{subsec:appendix_share_known_cases}

\begin{figure}[ht]
  \centering
  \includegraphics[width=\textwidth]{figures/results/figures/share_known_cases/full_combined_baseline_by_age_group_rki}
  \caption{Share of Detected Cases by Age Group}
  \label{fig:share_known_cases_by_age_group}
  \floatfoot{\noindent \textit{Note:} The figure shows the share of cases that is
  reported as an official case via PCR confirmation. We use the overall share of known
  cases that was estimated through the case fatality ratio by the
  \href{https://covid19.dunkelzifferradar.de/}{Dunkelzifferradar} for all of 2020 and
  then assume it to be constant as vaccinations of the elderly strongly affect the case
  fatality rate which the project does not account for. To get from an overall share of
  detected cases to the share of cases that is detected in each age group we use that
  asymptomatic cases are much less likely to be detected. As our model covers age
  specific asymptomatic rates this endogenously leads to group specific share known cases
  that verify that infections in younger age groups are under-detected. Starting in 2021
  in addition to the overall numbers of detected cases through symptoms and the share
  known cases, cases are also detected through confirmation of positive rapid tests. This
  leads to an increase in the share of known cases for all age groups but in particular
  for the younger age groups that are covered extensively with rapid tests through the
  rapid test requirement for participating in school.}
\end{figure}

It's noteworthy that the share of detected cases increases rapidly in May for the five to
fourteen year olds. This is a direct result of the mandatory tests in
school.

\FloatBarrier

\subsection{Rapid Tests}
\label{subsec:appendix_rapid_tests}


\begin{figure}[ht]
  \centering
  \includegraphics[width=\textwidth]{figures/results/figures/scenario_comparisons/combined_fit/full_share_ever_rapid_test}
  \caption{Share of Individuals Who Have Ever Done a Rapid Test}
  \label{fig:fit_share_ever_rapid_test}
  \floatfoot{\noindent \textit{Note:} This figure clearly shows that overall our
  assumptions on rapid tests are conservative. In our model, the share of individuals who
  have ever done a rapid test lies  consistently 10 percentage points below the share
  reported in the COSMO study \citep{Betsch2021}. The main reason for this is that we
  introduce rapid tests only at the start of 2021. However, using the available data to
  calibrate our rapid test parameters and estimating the remaining parameters to fit the
  official case numbers we arrive at a shape that is very similar to that implied by the
  survey results. This together with fitting the share of Germans with weekly rapid tests
  well (see Figure~\ref{fig:fit_share_rapid_test_within_last_week} makes us confident
  that our rapid test model is a good -- and especially not over-confident --
  representation of rapid testing in Germany.}
\end{figure}


\begin{figure}[ht]
  \centering
  \includegraphics[width=\textwidth]{figures/results/figures/scenario_comparisons/combined_fit/full_share_rapid_test_in_last_week}
  \caption{Share of Individuals Who Have Done a Rapid Test in the Last Week}
  \label{fig:fit_share_rapid_test_within_last_week}
  \floatfoot{\noindent \textit{Note:} Our modelling of rapid tests leads to very similar rates of individuals that are tested at least weekly as in the COSMO study \citep{Betsch2021}.
  In that study, individuals were asked if they performed rapaid tests at least weekly in
  the last four weeks. In our model we cannot verify the rhythm over the last four
  weeks but just observe the last time of a rapid test. However, both work and school
  rapid tests which make up the largest share of rapid tests (see
  Figure~\ref{fig:rapid_test_demand_by_channel}) are both required on a weekly or twice
  weekly basis while the private rapid test demand which are partly just triggered by
  events make up only a small fraction of tests.}
\end{figure}


\begin{figure}[ht]
  \centering
  \includegraphics[width=\textwidth]{figures/results/figures/rapid_test_statistics/popshare_tested}
  \caption{Share of the Population Demanding a Rapid Test Because of Different Channels}
  \label{fig:rapid_test_demand_by_channel}
  \floatfoot{\noindent \textit{Note:}
    Rapid tests in the education setting are demanded by teachers (nursery, preschool and
    school) as well as school pupils. After Easter the required frequency of tests is
    increased from once per week to twice per week. Work rapid tests are demanded by
    individuals that still have work contacts, i.e. do not work from home. The share of
    employers offering rapid tests increases over the time frame and the frequency of
    testing is also increased. Tests are demanded by individuals for one of three private
    reasons: having developed symptoms without access to a PCR test, having a household
    member that has tested positive or developed symptoms or having planned weekly
    meeting with friends. }
\end{figure}

\begin{figure}
    \centering
    \begin{subfigure}[b]{0.425\textwidth}
        \centering
        \includegraphics[width=\textwidth]{figures/results/figures/rapid_test_statistics/number_true_positive}
        \caption{Number of Discovered Cases Due to Rapid Tests by Channel}
        \label{fig:rapid_tests_number_true_positive}
    \end{subfigure}
    \hfill
    \begin{subfigure}[b]{0.425\textwidth}
        \centering
        \includegraphics[width=\textwidth]{figures/results/figures/rapid_test_statistics/number_false_positive}
        \caption{Number of False Positive Rapid Tests by Channel}
        \label{fig:rapid_tests_number_false_positive}
    \end{subfigure}
    \vskip3ex
    \begin{subfigure}[b]{0.425\textwidth}
        \centering
        \includegraphics[width=\textwidth]{figures/results/figures/rapid_test_statistics/number_true_negative}
        \caption{Number of True Negative Rapid Tests by Channel}
        \label{fig:rapid_tests_number_true_negative}
    \end{subfigure}
    \hfill
    \begin{subfigure}[b]{0.425\textwidth}
        \centering
        \includegraphics[width=\textwidth]{figures/results/figures/rapid_test_statistics/number_false_negative}
        \caption{Number of False Negative Rapid Tests by Channel}
        \label{fig:rapid_tests_number_false_negative}
    \end{subfigure}
    \vskip3ex
    \caption{Rapid Test Results}
    \label{fig:rapid_test_results}

    \floatfoot{\noindent \textit{Note:}
      The number of rapid tests of each category are upscaled to the full German
      population. \textcolor{red}{To be written.}
    }
\end{figure}


\begin{figure}
    \centering
    \begin{subfigure}[b]{0.425\textwidth}
        \centering
        \includegraphics[width=\textwidth]{figures/results/figures/rapid_test_statistics/true_positive_rate}
        \caption{Rate of True Positive Rapid Tests by Channel}
        \label{fig:rapid_tests_true_positive_rate}
    \end{subfigure}
    \hfill
    \begin{subfigure}[b]{0.425\textwidth}
        \centering
        \includegraphics[width=\textwidth]{figures/results/figures/rapid_test_statistics/false_positive_rate}
        \caption{Rate of False Positive Rapid Tests by Channel}
        \label{fig:rapid_tests_false_positive_rate}
    \end{subfigure}
    \vskip3ex
    \begin{subfigure}[b]{0.425\textwidth}
        \centering
        \includegraphics[width=\textwidth]{figures/results/figures/rapid_test_statistics/true_negative_rate}
        \caption{Rate of True Negative Rapid Tests by Channel}
        \label{fig:rapid_tests_true_negative_rate}
    \end{subfigure}
    \hfill
    \begin{subfigure}[b]{0.425\textwidth}
        \centering
        \includegraphics[width=\textwidth]{figures/results/figures/rapid_test_statistics/false_negative_rate}
        \caption{Rate of False Negative Rapid Tests by Channel}
        \label{fig:rapid_tests_false_negative_rate}
    \end{subfigure}
    \vskip3ex
    \caption{Rapid Test Results}
    \label{fig:rapid_test_results}

    \floatfoot{\noindent \textit{Note:}
      . \textcolor{red}{To be written.}
    }
\end{figure}



\FloatBarrier


\subsection{Scenarios}
\label{subsec:appendix_scenarios}


\begin{figure}[ht]
  \centering
  \begin{subfigure}[b]{.49\textwidth}
    \centering
    \includegraphics[width=0.9 \textwidth]{figures/results/figures/scenario_comparisons/new_work_scenarios/full_new_known_case}
    \caption{Reported Cases}
    \label{fig:work_scenarios_new_known_case}
  \end{subfigure}
  \hfill
  \begin{subfigure}[b]{.49\textwidth}
    \centering
    \includegraphics[width=0.9 \textwidth]{figures/results/figures/scenario_comparisons/new_work_scenarios/full_newly_infected}
    \caption{Total Cases}
    \label{fig:work_scenarios_newly_infected}
  \end{subfigure}
  \caption{The Effect of Different Work Scenarios on Reported and Total Cases}
  \label{fig:work_scenarios_detailed}
  \floatfoot{\noindent \textit{Note:} The figure shows the development of cases after the
  policy changes took place at Easter until the end of our simulation period (end of
  May). We vary the share of workers that work from home and how many tests are performed
  at work relative to our baseline scenario. Making it mandatory to test all employees
  that do not work from home markedly reduces cases -- even when only assuming 95\%
  compliance on both the employer and the employee side. As before, the observed cases
  can be misleading because more testing leads to more detected cases. It takes two to
  three weeks for the reduction in new infections effect to dominate the increased
  detection effect. Furthermore, the two opposing effects lead to a smaller effect size
  than is actually the case.}
\end{figure}

 \comment[id=K]{Regarding figure \ref{fig:work_scenarios_detailed}: Look up numbers and
 add them to the description.}


\begin{figure}[ht]
  \centering
  \begin{subfigure}[b]{.49\textwidth}
    \centering
    \includegraphics[width=0.9 \textwidth]{figures/results/figures/scenario_comparisons/school_scenarios/full_new_known_case}
    \caption{Reported Cases}
    \label{fig:school_scenarios_new_known_case}
  \end{subfigure}%
  \hfill
  \begin{subfigure}[b]{.49\textwidth}
    \centering
    \includegraphics[width=0.9 \textwidth]{figures/results/figures/scenario_comparisons/school_scenarios/full_newly_infected}
    \caption{Total Cases}
    \label{fig:school_scenarios_newly_infected}
  \end{subfigure}
  \caption{The Effect of Different School Scenarios on Reported and Total Cases}
  \label{fig:school_scenarios_detailed}
  \floatfoot{\noindent \textit{Note:} The figure shows the development of cases after the
   policy changes took place at Easter until the end of our simulation period (end of
   May). Apart from the enacted school policies as our baseline we simulate how cases
   would have developed if schools had been closed completely as the strictest possible
   counterfactual scenario and two opening models: One where schools open normally (with
   hygiene measures) without any testing in the education sector and one where schools
   open normally but testing shares develop as in the baseline scenario. Our simulations
   suggest that the enacted policies were as effective as keeping schools closed. Opening
   schools with the testing schemes that were in place after Easter would have had a
   small effect on the overall incidence. However, this is mainly due to the stringent
   testing that was in place in schools by that time. Had schools opened without testing
   requirements the total incidence would have been up to 50 points higher, though this
   would have been less visible in the reported cases.}
\end{figure}

 \comment[id=K]{Regarding figure \ref{fig:school_scenarios_detailed}: Look up numbers and
 add them to the description.}


\FloatBarrier
