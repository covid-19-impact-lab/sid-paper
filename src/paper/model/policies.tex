\subsection{Reducing Numbers of Contacts via NPIs}
\label{sec:policies}

 \comment[id=K]{I think we need to make sure we distinguish very clearly what is possible
 and what's implemented.}


Our model makes it very easy to model a wide range of NPIs, either in isolation or
simultaneously. This is important for two reasons: Firstly, it allows to predict and
quantify the effect of novel NPIs. Secondly, it allows to model the actually implemented
policy environment in great detail, which is necessary to use the full time series
of infections and fatality rates to estimate the model parameters.\footnote{
See \citet{Avery2020} for an explanation why it can be harmful to use too long time
series to estimate simple SEIR type models.}


Instead of thinking of policies as completely replacing how many contacts people have,
it is often more helpful to think of them as adjusting the pre-pandemic number of
contacts. Therefore, we implement policies as a step that happens after the number of contacts is
calculated but before individuals are matched.

On an abstract level, a policy is a functions that modifies the number of contacts of
one contact type. This function can be random or deterministic. For example, school
closures simply set all school contacts to zero. A work from home mandate leads to a
share of workers staying home every day whereas those who cannot work from home are
unaffected. Hygiene measures at work randomly reduce the number of infectious contacts
for all workers who still go to work.

Policies can also interact. For example, school vacations are temporally reducing school
contacts to zero while at the same time increasing other contacts to account for
increased leisure activities and family visits during this time. This is important to
reproduce the finding that school vacations do not reduce infection numbers even though
schools lead to infections when open \citep{Isphording2021}.

The most complex policies are typically found in the education sector. Since the
beginning of \replaced[id=K]{2020}{2021} schools have switched back and fourth between
full closures, split class approaches with alternating schedules for some or all age
groups and reopening while maintaining hygiene measures. On top of that there are
different policies for allowing young students whose parents work full time to attend
school even on days where they normally would not. For details on how we calibrate these
policies see Section~\ref{subsec:policies}.\comment[id=K]{Did I write enough on the school policies? (No sources or details so far...)}

Importantly, policies can depend on the health states of participating individuals.
\replaced[id=K]{Thus, it would be possible}{This allows} to quarantine entire school
classes if one student tested positive \replaced[id=K]{and many other forms of contact
tracing}{or to implement official or private contact tracing}.

For some policies the exact effect on each contact type is not easy to determine. If this
refers to a policy that has been active during the estimation period, it is possible to
estimate such parameters by fitting the model to time series data of infection rates.
This is only possible if the policy was not active during the whole estimation period and
thus the infection probabilities can be identified separately. We do this to account for
hygiene measures at school and in the workplace that have been in effect since November
2020.\comment[id=K]{I don't understand this paragraph.}

Not all things that reduce contacts compared to the pre-pandemic level are driven by
NPIs. Therefore, we also model endogenous contact reductions that can depend on the
health state of individuals \replaced[id=K]{}{, known risk contacts} or the local
incidence. Examples in our model are contact reductions for symptomatic individuals or
those who have a positive PCR or rapid tests \replaced[id=K]{or contact reductions when a
household member tested positive}. The extent to which contacts are reduced can be
calibrated from surveys. For an application of our model showcasing private contact
tracing in the context of the Christmas holidays see \cite{Gabler2020}.





