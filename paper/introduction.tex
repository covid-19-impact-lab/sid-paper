\section{Introduction}
\label{sec:introduction}

% setting
The first wave of the Covid-19 pandemic prompted strict lockdowns and restrictions
across the world.
At large social and economic costs, these strict measures were able to suppress the
spread of the disease in many countries and allowed governments to relax restrictions
over the summer.
When infections started to soar again in the fall, governments were more hesitant
and imposed less restrictive policies.
For example, Germany imposed a ``lockdown-light'' in November, closing fewer businesses
than in spring and leaving schools open.

% challenge
The less restrictive lockdowns proved insufficient to stop the second wave, leading
many experts to call for a strict lockdown to bring cases back to very low levels,
such as single digit incidences (for example \cite{Priesemann2021}).
These calls have received additional attention since new and more contagiuos variants
have emerged in across the globe \citep{Duong2021} and increased the urgency to
reduce transmission dynamics and increased awareness for the mutagenic threat posed by
high incidences.
Given the importance to bring down and maintain low infection numbers until
vaccinations reach herd immunity levels, it is paramount for policy makers and the
public to understand the effectiveness and trade-offs involved in different policies.

However, epidemiological models are not suited for this task.
As they have not been designed to predict the effects of fine-grained policies,
they need to be extended for each new policy proposal.

% contribution
This report describes a model that has been designed from the ground up to predict the
effects of contact reducing policies in real time. It has the following features:

\begin{enumerate}
    \item At the core of the model, people meet people based on a matching algorithm. We
          distinguish various types of contacts. The contact types are households,
          leisure activities, schools, preschools, and nurseries and several types of
          contacts at the workplace. Contact types can be random or recurrent and vary
          in frequency.

    \item Policies can be implemented as shutting down contact types entirely or
          partially. The reduction of contacts can be random or systematic. For example,
          it is possible to implement split class schooling where only one half of each
          class attends and the attending half switches on a weekly bases.

    \item Infection probabilities vary across contact types and reflect properties of
          the contact like the location (indoor/outdoor) and the kind of interaction
          (duration, physical contact). The probabilities are independent from the
          number of contacts and, thus, policy-invariant.

    \item The model achieves a good fit on German data of infection rates even if only
          the infection probabilities are fit to the data and the remaining parameters
          are calibrated from the medical literature and datasets on contact
          frequencies and mobility reductions.

    \item High quality Python code for the model is freely available on Github, well
          documented and very flexible\footnotemark. We are actively looking for
          researchers who want to use our model for their projects and apply it to other
          contexts.
\end{enumerate}

\footnotetext{
    The code can be found under \url{https://github.com/covid-19-impact-lab/sid} and the
    documentation with tutorials and background information under
    \url{https://sid-dev.readthedocs.io/}.
}

This report describes the model in an abstract way, but uses many realistic examples
from a version that is specialized to Germany. It is important to note that it would
be easy to adjust the model to other countries if data on the number of contacts and
a dataset with background characteristics are available.

More details about the German model as well as applications to currently discussed
German policies can be found in \citet{Dorn2020a, Gabler2020, Gabler2021}.

The report proceeds as follows: In Section~\ref{sec:literature_review}, we give a
short overview of epidemiological models.
We continue with a detailed description of our model in Section~\ref{sec:model}
and proceed with a description of model parameters and the estimation in
Section~\ref{sec:calibration_and_estimation}. The model is validated in
Section~\ref{sec:model_validation} by assessing the in-sample fit for reported
infections from August to October and by comparing the out-of-sample fit for the period
from November until the beginning of Christmas for different lockdown scenarios. We
conclude in Section~\ref{sec:conclusion}.
