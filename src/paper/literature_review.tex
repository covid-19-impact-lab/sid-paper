\subsection{Literature Review}
\label{sec:literature_review}

The workhorse model in epidemiology is the SEIR model. However, The traditional SEIR
model is not fine-grained enough to model nuanced policies.\comment[id=HM]{Still old
motivation. Hard to get the timing of tests and results right?} This has motivated a
large number of researchers to extend the standard model to allow for more heterogeneity
and flexibility. Examples are \citet{Grimm2020}, \citet{Donsimoni2020} and
\citet{Acemoglu2020}\comment[id=HM]{We need to cite more strategically. Cut Grimm,
Donsinomi. More state of the art SEIR in other fields. Or drop SEIR altogether? If I
read that as a Michael Meyer-Hermann type, I would think, ``ah, these economists
again...''} who develop multi group SEIR models to analyze the effects of targeted
lockdowns and \citet{Berger2020} who extend the SEIR model to analyze testing and
conditional quarantines. For a more comprehensive review see \citet{Avery2020}. Others
have used the results of a standard SEIR model as input for economic models that
estimate the cost of policies (e.g. \citet{Dorn2020}).

While the popularity of the SEIR model is mainly due to its simplicity, the extensions
are quite complex. It is unlikely that there will be a SEIR model that combines all
proposed extensions. Moreover, the extensions do not address other key issues: The main
parameter of the SEIR model, the basic reproduction number ($R_0$), is not
policy-invariant. It is a composite of the number of contacts each person has and the
infection probability of the contacts. In fact, policy simulations are done by setting
$R_0$ to a different value but it is hard to translate a real policy into the value of
$R_0$ it will induce. In other words, SEIR models are not suited for evaluating the
effect of policies which have never been experienced before.

Another commonly used model class in epidemiology are agent-based simulation
models. In a prototypical agent-based simulation model, individuals are
simulated as moving particles. Infections take place when two particles come closer than
a certain contact radius (e.g. \citet{Silva2020} and \citet{Cuevas2020}). While the
simulation approach makes it easy to incorporate heterogeneity in disease progression, it
is hard to incorporate heterogeneity in meeting patterns. Moreover, policies are modeled
as changes in the contact radius or momentum equation of the particles. The translation
from real policies to corresponding model parameters is a hard task.

These shortcomings have motivated variations of agent-based simulation models where
moving particles have been replaced by contact networks for households, work and random
contacts.

The OpenABM-Covid-19 model by \citet{Hinch2020} is the closest in spirit to
ours\comment[id=HM]{I'd rather be close to Aleta et al. ;-)}. They model detailed
contact networks for workplaces, schools and households and can evaluate the effect of
several NPIs. The main focus of their application are contact tracing policies.

\citet{Aleta2020} develop an agent-based simulation model with a very high geographical
resolution by estimating contact networks from fine grained mobility data for the
Boston metropolitan area. They use this model to show how NPIs, contact tracing and PCR
testing can influence the infection dynamics. However, they do not calibrate their model
match actual infection numbers which makes it more suitable to explore the general
mechanics of different disease mitigation measures than for their quantitative
evaluation.

\citet{Bicher2021} simulate the Austrian population during the first wave (February 21
to April 9, 2020). It uses the same data provided by \citet{Mossong2008} as we use to
calibrate contact networks for households, workplaces and schools. The model focuses on
analyzing the effect of different contact tracing strategies.

Moreover, there are several forthcoming applications\comment[id=HM]{What is that
supposed to mean?} of agent-based simulation models with contact networks. Examples are
\citet{Basurto2020}, \citet{DelliGatti2020} and \citet{Mellacher2020}.

Our model combines elements from the above models and adds several new features. To
the best of our knowledge, ours is the only model that can explain observed case
numbers with an excellent fit over more than 9 Months. We also have the most fine
grained model of schools and NPIs that affect schools. Finally, we have an extremely
detailed model of PCR and rapid tests with a share of detected cases that varies over
time and across age groups.


