\subsection{Summary}
\label{sub:model_summary}

To predict and quantify the effects of a wide variety of fine-grained social distancing
policies, vaccinations and rapid testing, we propose a different model structure. Our
model inherits many features from prototypical agent-based simulation models but
replaces the contacts between moving particles by contacts between individuals who work,
go to school, live in a household and enjoy leisure activities.

The structure of the model is depicted in Figure~\ref{fig:model_contacts_infections}.

We distinguish between eight types of contact
models which are all listed in Figure~\ref{fig:model_contacts_infections}: households,
recurrent and random work contacts, recurrent and random leisure contacts, and nursery,
preschool, and school contacts.

The number of contacts is translated into infections by a matching algorithm.
There are different matching algorithms for recurrent contacts (e.g. classmates, family
members) and non-recurrent contacts (e.g. clients, contacts in supermarkets).
All types of contacts can be assortative with respect to geographic and demographic
characteristics.

The infection probabilities of contacts vary with contact type, age of the susceptible
person, and the virus strain of the infected person. Moreover, they follow a seasonal
pattern. The strength of the seasonality effect is higher for contacts that are easy to
be moved to an outside location in summer (such as leisure contacts) and smaller for
contacts that take place inside even in summer (e.g. work contacts).

Once a person is infected, the disease progresses in a fairly standard way which is
depicted in Figure~\ref{fig:model_disease_progression}. Asymptomatic cases and cases with mild
symptoms are infectious for some time and recover eventually. Cases with severe symptoms
additionally require hospitalization and lead to either recovery or death.

After rapid tests become available, people who work or go to school can receive rapid
tests there. Moreover, people can decide to make a rapid test if they develop symptoms,
have many planned contacts\comment[id=K]{at the moment it's implemented as planing to
participate in a weekly leisure meeting. Should we be more explicit here?} or observe
cases in their contact network. People who have a positive rapid test demand a
confirmatory PCR test with a certain probability. Moreover, PCR tests can be demanded
because of symptoms or randomly.

This rich model of PCR and rapid tests leads to a share of detected cases that varies
over time and across age groups. It also allows to quantify the effect of changes in
testing policies on the dynamic of infections.

People who have symptoms, received a positive test, or had a risk contact can reduce
their number of contacts across all contact types endogenously. The extent to which this
is done is calibrated from survey data.

The model makes it very simple to translate policies into model quantities. For example,
school closures imply the complete suspension of school contacts. A strict lockdown
implies shutting down work contacts of all people who are not employed in a systemically
relevant sector. It is also possible to have more sophisticated policies that condition
the number of contacts on observable characteristics, risk contacts or health states.

An important feature of the model is that the number of contacts an individual has of
each contact type can be calibrated from publicly available data \citep{Mossong2008}.
This in turn allows us to estimate policy-invariant infection probabilities from time
series of infection and death rates using the method of simulated moments
\citep{McFadden1989}. Since the infection probabilities are time-invariant, data
collected since the beginning of the pandemic can be used for estimation. Moreover,
since we model the testing strategies that were in place at each point in time, we can
correct the estimates for the fact that not all infections are observed.

The model has a very modular structure and can easily be extended to distinguish more
contact types, add more stages to the disease progression, implement new policies
or test demand models. The main bottleneck is not complexity or computational cost
but the availability of data to calibrate the additional model features.
