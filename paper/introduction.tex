\section{Introduction}
\label{sec:introduction}

The first wave of the Covid-19 pandemic prompted strict lockdowns and restrictions
across the world. As a result, many countries were able to reduce the number of positive
cases and could relax restrictions imposed on their societies. At the same time, the
social and economic costs were enormous. In the second wave, countries are trying to
implement more nuanced policies. For example, Germany imposed a ``lockdown-light'' in
November. Some businesses are closed and certain leisure activities are prohibited but
schools and nurseries stay open.

Current epidemiological models have not been designed to predict the effect of such
fine-grained policies. These models need to be extended for each new policy proposal and
cannot be easily adapted to fast changing environments. This report describes a model
that has been designed from the ground up to predict the effects of contact reducing
policies in real time. At the time of this writing, it has the following features:

\begin{enumerate}
    \item At the core of the model, people meet people based on a matching algorithm. We
          distinguish various types of contacts. Currently, these are households,
          leisure activities, schools, nurseries and several types of contacts at the
          workplace. Contact types can be random or recurrent and vary in frequency.

    \item Policies can be implemented as shutting down contact types entirely or
          partially. The reduction of contacts can be random or systematic, e.g., to
          allow for essential workers.

    \item Infection probabilities vary across contact types, but are invariant to
          policies which reduce contacts.

    \item The model achieves a good fit on German infection and fatality rate data even
          if only the infection probabilities are fit to the data and the remaining
          parameters are calibrated from the medical literature and datasets on contact
          frequencies.

    \item High quality Python code for the model is freely available on Github, well
          documented and very flexible\footnotemark. We are actively looking for
          researchers who want to use our model for their projects.
\end{enumerate}

\footnotetext{
    The code can be found under \url{https://github.com/covid-19-impact-lab/sid} and the
    documentation with tutorials and background information under
    \url{https://sid-dev.readthedocs.io/}.
}

In Section~\ref{sec:literature_review}, we give a short overview of epidemiological
models and how they relate to our model. We continue with a detailed model description
in Section~\ref{sec:model} and proceed with a description of model parameters and the
estimation in Section~\ref{sec:calibration_and_estimation}. The model is validated in
Section~\ref{sec:model_validation} by assessing the in-sample fit for reported
infections from August to October and by comparing the out-of-sample fit for the period
from November until the beginning of Christmas for different lockdown scenarios. We
conclude in Section~\ref{sec:conclusion}.

This report describes the model in an abstract way, but uses many realistic examples
from a version that is specialized to Germany. It is important to note that this
specialization is not baked into the model or the Python code. It is easy to adjust the
model to other countries if data on the number of contacts and a dataset with background
characteristics are available.

More details about the German model as well as applications to currently discussed
German policies can be found in a companion paper
(\citet{Dorn2020a}).\comment[id=T]{Needs to be updated with a reference to the new
report.}
