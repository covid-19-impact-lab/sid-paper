\subsection{Testing} % (fold)
\label{sub:testing}

We support to model testing as consisting of three stages.
Firstly, we model who demands a test. Demand functions map from individual
characteristics to a probability which is the probability for this individual to demand
a test. There can be multiple demand functions where each function may describe a
different channel. For example, individuals who experience symptoms or have a risk
contact may ask for a test. Or, the ministry of education requires a negative test
result from every teacher every second week. After the probabilities for each individual
and every demand model are computed, individuals who demand a test as well as the
channel is sampled.

The second stage is the allocation phase in which demand and supply for tests are
matched. The number of available tests can be inferred from official data and used to
model shortages in supply. When demand exceeds supply, some individuals might be given
preferred access to tests because of their own vulnerability or their potential to
become a super-spreader.

In the last and third phase, administered tests are processed.
This step can become a bottleneck in the testing process if there are not enough
laboratories or necessary resources available to evaluate all tests.

In our empirical estimation we use a very simplified testing model where the number
of tests to be distributed is calculated from estimates for the ratio of known to all
infections.\footnotemark
\footnotetext{The Dunkelzifferradar project publishes daily estimates of the dark figure
of infections under \url{https://covid19.dunkelzifferradar.de/}}
Using these estimates as well as data on the test distribution
over age groups by the RKI\footnotemark
\footnotetext{https://ars.rki.de/Content/COVID19/Main.aspx}
we allocate tests firstly among the symptomatic
and then randomly allocate tests to newly infected to fit the German test distribution.


% subsection testing (end)
