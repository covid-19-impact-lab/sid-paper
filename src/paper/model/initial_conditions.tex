\subsection{Initial Conditions} % (fold)
\label{sub:initial_conditions}

Consider a situation where you want to start a simulation with the beginning set amidst
the pandemic. It means that several thousands of individuals should already have
recovered from the disease, be infectious, symptomatic or in intensive care at the start
of your simulation. Additionally, the sample of infectious people who will determine the
course of the pandemic in the following periods is likely not representative of the whole
population because of differences in behavior (number of contacts, assortativity), past
policies (school closures), etc.. The distribution of \replaced[id=K]{health
states}{courses of diseases} in the population at the beginning of the simulation is
called initial conditions.

To come up with realistic initial conditions, we match reported infections from official
data to simulated individuals by available characteristics like age and geographic
information. The matching must be done for each day of a longer time frame like a month
to have individuals with all possible health states. Then, health statuses evolve until
the beginning of the simulation period without simulating infections by contacts. We also
correct reported infections for a reporting lag and scale them up \replaced[id=K]{by the
share of detected cases}{} to arrive at the true number of infections.

% subsection initial_conditions (end)
