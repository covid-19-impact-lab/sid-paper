% educ policies

% summary
For nurseries, preschools and schools we implement vacations as announced by the German
federal states as well as school closures, emergency care and rotating schooling
schedules where only one half of students attends every other week or day. An
approximation of the share of contacts still taking place with the different school
regulations can be found in Figure~\ref{fig:school_multiplier}. Note that schooling
policies are decided on the state level and usually involved rules that depend on local
incidences. We simplify these rules to one federal policy from the federal incidence and
the policies of the three most populous federal states (North Rhine-Westphalia, Bavaria
and Baden-Württemberg).

% young educ
The policies for preschools and nurseries are similar to the school policies but simpler.
Until November children attend completely normally, starting in November with increased
hygiene measures. Nurseries and preschools stay open until mid December. From mid
December until January 10, nurseries and preschools were nearly completely closed. If
parents could credibly demonstrate that both parents work in systemically relevant
professions and no other child care arrangement was possible, nurseries and preschools
offered so called ``emergency care''. We assume 10\% of children qualified and used
emergency care during this time. After January 10 when more and more parents returned to
work the rules for emergency care were made less strict and we assume a third children
attended to nursery and preschool. This policy stayed in place until February 20.
Afterwards, preschools and nurseries opened normally until mid March. Then during the
third wave the restrictions February were put back into place until end of April when
nurseries and preschools opened again and stayed open \citep{KiTa_BY, KiTa_NRW, KiTa_BW,
KiTa_BWa, KiTa_BYa}.

% school