\subsection{Initial Conditions} % (fold)
\label{sub:initial_conditions}

Consider a situation where you want to start a simulation with the beginning set amidst
the pandemic. It means that several thousands of individuals have already recovered from
the disease, are infectious, symptomatic or in ICUs. Additionally, the sample of
infectious people who will determine the course of the pandemic in the following periods
is likely not representative of the whole population because of differences in behavior
(number of contacts, assortativeness), past policies (school closures), etc.. The
distribution of courses of diseases in the population at the begin of the simulation is
called initial conditions.

To come up with realistic initial conditions, we match reported infections from official
data to simulated individuals by available characteristics like age and geographic
information. The matching must be done for each day of a longer time frame like a month
to have individuals with possible health states. Then, health statuses evolve until the
begin of the simulation period without simulating infections by contacts. It is also
possible to correct reported infections for a reporting lag and scale them up to arrive
at the true number of infections.

% subsection initial_conditions (end)
