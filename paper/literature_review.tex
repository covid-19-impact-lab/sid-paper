\section{Literature Review}
\label{sec:literature_review}

We build on two strands of literature: Recent extensions of the epidemiological SEIR
model and agent-based simulation models.

The traditional SEIR model is not fine-grained enough to model nuanced policies. This
has motivated a large number of researchers to extend the standard model to allow for
more heterogeneity and flexibility. Examples are \citet{Grimm2020},
\citet{Donsimoni2020} and \citet{Acemoglu2020} who develop multi group SEIR models to
analyze the effects of targeted lockdowns and \citet{Berger2020} who extend the SEIR
model to analyze testing and conditional quarantines. For a more comprehensive review
see \citet{Avery2020}. Others have used the results of a standard SEIR model as input
for economic models that estimate the cost of policies (e.g. \citet{Dorn2020}).

While the popularity of the SEIR model is mainly due to its simplicity, the extensions
are quite complex. It is unlikely that there will be a SEIR model that combines all
proposed extensions. Moreover, the extensions do not address other key issues: The main
parameter of the SEIR model, the basic reproduction number ($R_0$), is not
policy-invariant. It is a composite of the number of contacts each person has and the
infection probability of the contacts. In fact, policy simulations are done by setting
$R_0$ to a different value but it is hard to translate a real policy into the value of
$R_0$ it will induce. In other words, SEIR models are not suited for evaluating the
effect of policies which have never been experienced before.

Another commonly used model class in epidemiology are agent-based simulation models. In
these models individuals are simulated as moving particles. Infections take place when
two particles come closer than a certain contact radius (e.g. \citet{Silva2020} and
\citet{Cuevas2020}). While the simulation approach makes it easy to incorporate
heterogeneity in disease progression, it is hard to incorporate heterogeneity in meeting
patterns. Moreover, policies are modeled as changes in the contact radius or momentum
equation of the particles. The translation from real policies to corresponding model
parameters is a hard task.

\citet{Hinch2020} is a recent extension of the prototypical agent-based simulation model
that replaces moving particles by contact networks for households, work and random
contacts. This model is similar in spirit to ours but focuses on contact tracing rather
than social distancing policies.

The above assessment of epidemiological models is not meant as a critique. We are aware
that these models were not designed to predict the effect of fine-grained social
distancing policies in real time and are very well suited to their purpose. We invite
epidemiologists to provide feedback and collaborate to improve our model.
