\subsection{Model Fit}
\label{subsec:fit_results}

%%% WRITE MORE POSITIVELY

This section compares simulated data from our model with empirical data from Germany. We
look at observed infections, the spread of the B.1.1.7 mutation,
vaccinations\comment[id=K]{We fit vaccinations in the population by construction. Maybe
this should go to a different section?} and rapid test demand. Where available we do not
only look at aggregated statistics but also analyze the model fit for age groups and
federal states.

% summary of the fit
Overall, our model achieves an excellent fit of the two waves of infections with few free
parameters (Figure~\ref{fig:aggregated_fit2}. As a result the effective replication
number is similar to that reported by the RKI (see Figure~\ref{fig:fit_r_effective}).
This excellent fit is also achieved for most age groups in Germany. The fit is also good
for many German federal states. Despite, the fact that the number of performed rapid
tests and their distribution in the population results endogenously inside our model we
fit the share of the population with at least a weekly rapid test very well and err on
the side of too few individuals that have ever done a rapid test.

% observed infections
Our fit of the infection rates in Germany between October 2020 and June 2021 is very
good. The incidence in our model follows the shape of the reported incidence almost
perfectly. % The increase in December is somewhat less steep than it was. This can be
% attributed to the large number of cases in nursing homes during that time that our
% model cannot capture because it does not include nursing homes. \comment[id=K]{Write
% about uncertainty and that we take enough simulations to have a stable mean}

\begin{figure}[ht]   % observed infections with single runs
  \centering
  \includegraphics[width=\textwidth]{figures/results/figures/scenario_comparisons/combined_fit/full_new_known_case_with_single_runs}
  \caption{Fit Over the Full Simulation Time Frame with Single Simulation Runs}
  \floatfoot{\noindent \textit{Note:} The figure shows the weekly incidence rates per
  100,000 people for the reported simulated infections rates. The mean infection rate is
  the thick blue line. Single simulation runs are plotted in lighter and thinner lines.
  The official case numbers as reported by the Robert-Koch-Institut are plotted in black.
  The fit is overall very good. The higher the mean incidence and the stronger the growth
  the more variance there is between simulation runs. We averaged over 30 simulation
  runs.}
  \label{fig:aggregated_fit2}
\end{figure}

% observed infections by age group
Zooming into the different age groups in Figure~\ref{fig:age_group_fit}, we can see that
our model is also able to reproduce the infection rates on this level. Differences
between the age groups allow us to identify the infection probabilities because different
age groups have contacts of different types (e.g. Mostly children go to school, the share
of workers is highest in the 35 to 59 age bracket etc.). There are only three
\textcolor{red}{\ldots}. Firstly, the age group of 80 to 100 year olds due to the absence
of nursing homes in our synthetic population. Secondly, the increase in adolescents in
April. It is surprising that we do not match this increase despite explicitly modeling
the introduction of rapid tests for students. However, it can be expected that every
rapid test in a school setting is followed up by a PCR test while only 82\% of our
population \citep{Betsch2021} demand a confirming PCR test when receiving a positive
rapid test. The last age group where our fit does not fit the empirical curve very well
is the group of 15 to 34 year olds. This group has been touted to have a very active
social life and it has often been suspected that this age group also stayed more socially
active during the pandemic. Since our model does not include either effect it's
unsurprising that the actual case numbers in this group are consistently higher than our
simulated case numbers.\comment[id=K]{Does anyone have an idea why we do not match the
shape from Oct to Dec?}

\begin{figure}[ht]  % observed infections by age group
  \centering
  \includegraphics[width=\textwidth]{figures/results/figures/incidences_by_group/age_group_rki/full_combined_baseline_new_known_case}
  \caption{Simulated and Empirical Infections by Age Group}
  \floatfoot{\noindent \textit{Note:} The figure shows the weekly incidence rates per
  100,000 people for the reported versus the simulated infections rates for different age
  groups. The age group of individuals above 80 needs to be interpreted with caution
  because our synthetic population only includes private households, i.e. nursing homes
  are not represented in our model. They accounted for many cases and deaths in the
  winter of 2020 and many 80 to 100 year olds live in these facilities. However, the
  official data does not contain information on whether cases were nursing home
  inhabitants or not. We averaged over 30 simulation runs.}
  \label{fig:age_group_fit}
\end{figure}


% observed infections by federal state
Looking at another disaggr
Our states are uniform with respect to their age distribution.

\begin{figure}[ht]   % observed infections by federal state
  \centering
  \includegraphics[height=0.95\textheight]{figures/results/figures/incidences_by_group/state/full_combined_baseline_new_known_case}
  \caption{Simulated and Empirical Infections by Federal State}
  \floatfoot{\noindent \textit{Note:} The figure shows the weekly incidence rates per
  100,000 people for the reported versus the simulated infections rates for different
  federal states. We averaged over 30 simulation runs.}
  \label{fig:state_fit}
\end{figure}


Our fit of the effective replication number $R_t$ closely follows the values reported by
the RKI. \textcolor{red}{ldots}

\begin{figure}[ht]   % R effective
  \centering
  \includegraphics[width=\textwidth]{figures/results/figures/scenario_comparisons/combined_fit/full_r_effective_with_single_runs}
  \caption{Effective Replication Number $R_t$ in the Model and as Reported by the
  Robert-Koch-Institute}
  \floatfoot{\noindent \textit{Note:} The figure shows the effective replication number
  ($R_t$) as reported by the RKI and as calculated in our model. The $R_t$ gives the
  average number of new infections caused by one infected individual. The $R_t$ in our
  model broadly follows the $R_t$ reported by the RKI. Two trends stand out. Firstly, the
  RKI's $R_t$ drops faster in November. This could be due to a change in the testing
  policy that focused tests on the elderly when the second wave hit Germany and led to a
  decline in the overall share of detected cases. The second difference is from mid
  February to mid March where the RKI's reported $R_t$ increased more rapidly than that
  in our model. Here the opposite effect can be expected. During this time rapid tests
  increased strongly leading to more cases being detected. In the short term this leads
  an $R_t$ estimation that is based on detected cases to overestimate the replication
  number.}
  \label{fig:fit_r_effective}
\end{figure}


% fatality rates: MISSING


\begin{figure}[ht]   % Share B.1.1.7
  \centering
  \includegraphics[width=\textwidth]{figures/results/figures/scenario_comparisons/combined_fit/full_share_b117_with_single_runs}
  \caption{Share of B.1.1.7 in the Model and as Reported by the Robert-Koch-Institute}
  \floatfoot{\noindent \textit{Note:} The figure shows the share of B.1.1.7 as
  reported by the RKI and as calculated in our model. We only introduce a few cases over
  the cause of January. From then B.1.1.7 takes over endogenously through its increased
  infectiousness. We model no other features of B.1.1.7. At most we introduce 0.75 cases per 100,000 inhabitants.}
  \label{fig:fit_share_b117}
\end{figure}


\begin{figure}[ht]   % fit of vaccinations
  \centering
  \includegraphics[width=\textwidth]{figures/results/figures/scenario_comparisons/combined_fit/full_ever_vaccinated}
  \caption{Share of Vaccinated Individuals}
  \floatfoot{\noindent \textit{Note:} The figure shows the rate of individuals that are
  vaccinated in our synthetic population versus in the general German population. Note
  that we excluded the vaccinations that were given to nursing homes, approximately the
  first percent of the German population that were vaccinated. Overall, our model covers
  a time frame that goes from zero vaccinated individuals to a state where over 40\% of
  the population are vaccinated. Our vaccinations work imperfectly but we do not model
  different vaccines nor do we distinguish between first and second shot.}
  \label{fig:fit_vaccinations}
\end{figure}


\begin{figure}[ht]     % rapid test demand
    \centering
    \caption{Share of Individuals With Rapid Tests}
    \label{fig:share_ever_rapid_test2}
    \begin{subfigure}{.55\textwidth}
        \includegraphics[width=0.9 \textwidth]{figures/results/figures/scenario_comparisons/combined_fit/full_share_ever_rapid_test}
        \caption{Share of Individuals That Have Ever Done a Rapid Test}
    \end{subfigure}%
    \begin{subfigure}{.55\textwidth}
        \includegraphics[width=0.9 \textwidth]{figures/results/figures/scenario_comparisons/combined_fit/full_share_rapid_test_in_last_week}
        \caption{Share of Individuals Having Done a Rapid Test in the Last Week}
    \end{subfigure}
    \label{fig:share_rapid_test_last_week2}
    \floatfoot{\noindent \textit{Note:} The figure compares the share of individuals who
        have ever done a rapid test or done a rapid test within the last week in our
        simulations to the shares reported in the
        \href{https://projekte.uni-erfurt.de/cosmo2020/web/topic/wissen-verhalten/80-schnelltests/}{COVID-19
        Snapshot Monitoring Survey}. The left panel compares the share of individuals who
        have ever done a rapid test. The right panel compares the share of individuals
        who have done a rapid test within the last seven days in our simulation compared
        to the share reporting to have done at least weekly rapid tests in the last four
        weeks in the COSMO survey. Overall our calibration of rapid tests are slightly
        conservative. The overall share is below that in the study. We fit the share of
        weekly tests quite exactly. However, the study only covers adults while our share
        also includes children who are tested very regularly when attending school.}
\end{figure}



\FloatBarrier

