
In our model, there are five reasons why rapid tests are done:\comment[id=J]{Add a
section on how we calibrate rapid test demand; Mainly describe the datapoints we have and
say that we usually interpolate linearly in between data points. (Only exception to that
is private rapid test demand, which we fit to data)}

\begin{enumerate}
    \item someone plans to have work contacts
    \item someone is an employee of an educational facility or a school pupil
    \item a household member has tested positive or developed symptoms
    \item someone has developed symptoms but has not received a PCR test
    \item someone plans to participate in a weekly non-work meeting
\end{enumerate}

\subsubsection{work rapid tests}

For work contacts, we know from the COSMO study (\cite{Betsch2021}, 20th/21st of April)
that 60\% of workers who receive a test offer by their employer regularly use it. We
assume this share to be time constant.

In addition, there are some surveys that allow us to trace the expansion of employers who
offer tests to their employees. Mid march, 20\% of employers offered tests to their
employees \citep{DIHK2021}. In the second half of March, 23\% of employees reported being
offered weekly rapid tests by their employer \citep{Ahlers2021}. This share increased to
60\% until the first days of April \cite{ZDF2021}.

\comment[id=K]{ToDo: Find the survey that the ZDF is citing here}

Until mid April 70\% of workers were expected to receive a
weekly test offer \citep{AerzteZeitung2021}. However, according to surveys conducted in
mid April \citep{Betsch2021}, less than two thirds of individuals with work contacts
receive a test offer. Starting on April 19th employers were required by law to provide
two weekly tests to their employees \citep{Bundesanzeiger2021}. We assume that compliance
is incomplete and only 80\% of employers actually offer tests.

\subsubsection{educ rapid tests}

We assume that employees in educational facilities start getting tested in 2021 and that
by March 1st 30\% of them are tested weekly. The share increases to 90\% for the week
before Easter. At that time both Bavaria \citep{BayrischerRundfunk2021} and
Baden-Württemberg \citep{MinisteriumKultus2021} were offering tests to teachers and
North-Rhine
Westphalia\footnote{\url{https://www.land.nrw/de/pressemitteilung/umfassende-informationen-fuer-die-schulen-zu-corona-selbsttests-fuer-schuelerinnen}}
\cite{DPA2021} and Lower Saxony \citep{SueddeutscheZeitung2021} were already testing
students and tests for students and teachers were already mandatory in Saxony
\citep{SueddeutscheZeitung2021a}. After Easter we assume that 95\% of teachers get tested
twice per week.

Tests for students started
later\footnote{\url{https://www.land.nrw/de/pressemitteilung/umfassende-informationen-fuer-die-schulen-zu-corona-selbsttests-fuer-schuelerinnen}}
\citep{MinisteriumKultus2021} so we assume that they only start in February and only 10\%
of students get tested by March 1st. Relying on the same sources as above we approximate
that by the week before Easter this share had increased to 40\%.\footnote{\url{https://www.land.nrw/de/pressemitteilung/umfassende-informationen-fuer-die-schulen-zu-corona-selbsttests-fuer-schuelerinnen}, }

After Easter the share of students receiving twice weekly tests is set to 75\%. This as
based on tests becoming mandatory becoming mandatory in Bavaria after Easter
break\footnote{Bavaria\footnote{\url{https://bit.ly/3nz5fXS}}, in North-Rhine Westphalia
on April
12th\footnote{https://www.schulministerium.nrw/ministerium/schulverwaltung/schulmail-archiv/14042021-schulbetrieb-im-wechselunterricht-ab-montag},
\url{https://bit.ly/2QHilX3}} and on the 19th in
Baden-Württemberg\footnote{https://bit.ly/3vuetaD, https://bit.ly/3vuetaD}.


To limit our degrees of freedom, we only have one parameter that governs how many
individuals do a rapid test because of any of the private demand reasons (own symptoms
but no PCR test, planned weekly leisure meeting or a symptomatic or positively tested
household member).

We assume that there is no private rapid test demand until March when both the citizens'
tests and rapid tests for lay people started to become available
\footnote{\url{https://bit.ly/3ehmGcj}, \url{https://bit.ly/3xJCIn8}} and other access to
rapid tests was very limited.

According to the COSMO study\footnote{\url{https://bit.ly/2QSFAgR}} 63\% would have been
willing to take a test in the round of 23rd of February 2021 when an acquaintance would
have tested positive. Since this is only asking for willingness not actual behavior and
the demand when meeting with friends is very likely lower, we take this as the upper
bound of private rapid test demand which is reached on May 4th. To cover that many people
are likely to have sought and done their first rapid test before the Easter holidays to
meet friends or family, we let the share of individuals doing rapid tests in that time
increase more rapidly than before and after. By end of March 25\% of individuals would do
a rapid test due to a private reason.

