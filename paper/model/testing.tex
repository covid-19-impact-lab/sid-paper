\subsection{Testing} % (fold)
\label{sub:testing}

The model offers two approaches to implement testing for Covid-19.

The first way can be described as an top-down approach. Using data on estimates for the ratio of known to all infections like from the DunkelzifferRadar project\footnotemark, a random sample of individuals who is newly infected is sampled and assigned a test. The result of the test will be revealed after some duration which must take into account the time it takes for a person to develop symptoms, the availability of tests and time to process the tests.

\footnotetext[dunkelzifferradar]{The Dunkelzifferradar can be found here: \url{https://covid19.dunkelzifferradar.de/}}

The second way is an bottom-up approach which consists of three phases. In the first phase, the demand for tests is modeled. Demand functions map from individual characteristics to a probability which is the probability for this individual to demand a test. There can be multiple demand functions where each function may describe a different channel. For example, individuals who experience symptoms or have a risk contact may ask for a test. Or, the ministry of education requires a negative test result from every teacher every second week. After the probabilities for each individual and every demand model are computed, individuals who demand a test as well as the channel is sampled.

The second stage is the allocation phase in which demand and supply for tests are matched. The number of available tests can be inferred from official data and used to model shortages in supply. When demand exceeds supply, some individuals might be given preferred access to tests because of their own vulnerability or their potential to become a super-spreader.

In the last and third phase, administered tests are determined to be processed. This step can become a bottleneck in the testing process if there are not enough laboratories or necessary resources available to evaluate all tests.

The strength of the first approach is its simplicity, but it relies on estimates for the ratio of known to unknown cases for which not much data exist in the absence of systematic and randomized testing. The second approach is much more detailed and in principle able to reflect individual responses and governmental restrictions on the access to tests. At the same time, individual behavior and restrictions are complex and the determinants of the demand, allocation and processing functions require a lot of calibration.

% subsection testing (end)
