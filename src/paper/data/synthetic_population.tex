\subsection{The Synthetic Population}
\label{subsec:synthetic_population}

We build a synthetic population based on the German microcensus \citep{FDSAeDBUDL2018}.
We only use private households, i.e. exclude living arrangements such as nursing homes as
non-private households vary widely in size and it is very difficult to know which
contacts take place in such living arrangements.

We sample households to build our synthetic population of over one million households
keeping for each individual their age, gender, occupation and whether they work on
Saturdays and Sundays. For each household we draw its county and set the corresponding
federal state.% we draw the county because county is not part of the campus file

We randomly assign 35\% of children below three to attend a nursery \citep{Destatis2020}.
For children between three and six years old, we assume all go to preschool (officially
92.5\% according to \cite{Destatis2020}). Nursery groups consist

Children that attend a nursery meet in groups of four plus one adult care taker every
weekday when there are no school vacations. Preschool children meet in groups of nine
with two adult care takers. These groups are mixed with respect to age but all belong to
the same state and mostly to the same county. For nurseries the group size is four, for
preschools it's nine \citep{BertelsmannStiftung2019}.

Every child that goes to school is part of three different classes that meet every day on
weekdays when there are no vacations and no school policies are in place.\footnote{We
implement vacations on the federal state level.} Each class consists of approximately 23
students \citep{OECD2013} and two teachers. All students in a class are of the same age
and live in the same state and if mostly also in the same county. In addition, each
child gets assigned a value that captures his or her need to attend nursery, preschool or
school. This allows us to capture various degrees of emergency care that can be granted
while educational facilities are closed or function in some kind of rotating schedule.

Workers are assigned to a daily meeting work group. The group sizes vary to match the
number of daily repeating work contacts reported by working individuals in
\cite{Mossong2008}. These groups only consist of workers that work in the same county.
For a distribution of the number of daily recurring work contacts see
Figure~\ref{n_contacts_work_daily_recurrent}. To match the number of weekly work groups
we match each worker with up to 14 other workers to match the number of reported weekly
work contacts shown in Figure~\ref{n_contacts_work_weekly_recurrent}. Each pair is
assigned a weekend on which they always meet in the absence of work policies. 80\% of
these contacts are individuals from the same county.
% work contact priority
In the same way children have a educational priority determining if they are entitled to
emergency care workers are assigned a work contact priority that captures how necessary
their work is and to which degree they can work from home. This means that it's always
the same individuals that continue to have work contacts when work from home mandates of
a certain strictness are in place. Since our work contact priority is uniformly normal
from zero to one, we can use the reduction in work mobility as reported by
\cite{Google2021} as a proxy for the share of individuals that do not have work contacts.

In addition to creating groups for educational facilities and work we also have other
recurring contacts to represent things like groups of friends or sports teams that
practice regularly together. Both daily and weekly groups are created analogously to the
work groups but matching the numbers in Figure~\ref{n_contacts_other_daily_recurrent} and
Figure~\ref{n_contacts_other_weekly_recurrent}. In addition, since leisure contacts are
highly assortative by age all individuals that have a daily leisure contact are matched
with a person not only from the same county but also from the same age group.

The individuals in our population can react to events such as developing symptoms that
are typical of CoViD-19, a positive PCR test or a positive rapid test by reducing their
contacts. To determine who would reduce their contacts in such a situation or demand a
rapid test to avoid quarantine we introduce a quarantine compliance parameter. Similarly
we introduce a rapid test compliance parameter that determines in which order individuals
start demanding rapid tests when rapid tests become increasingly available. This makes
sure that when for example only 10\% of workers get tested, it's the same workers that
have access to tests every week.

Lastly, for the distribution of vaccinations every individual is assigned a vaccination
group and a vaccination rank from that group that creates a complete vaccination queue
over the population including 15\% that refuse to be vaccinated \citep{RKI2021}. The
vaccination groups are created to match the recommendations by the Ständige
Impfkommission \citep{VygenBonnet2020}.\footnote{We cover that teachers were prioritized
more than recommended by the commission.}\comment[id=K]{Write how that plays out in our simulations for the different age groups.}

% To cover that the Pfizer-BioNTech vaccine was later approved for younger age groups we
% put adolescents and children into two groups that follow after the general population.
% These groups do not become eligible within our simulation frame until June.


% \begin{figure}[ht]   % vaccinations by age group
%   \centering
%   \includegraphics[width=\textwidth]{figures/results/figures/vaccinations/spring_baseline}
%   \caption{Simulated and Empirical Infections by Federal State}
%   \floatfoot{\noindent \textit{Note:} \textcolor{red}{To be written.}}
%   \label{fig:vaccinations_by_age_group}
% \end{figure}


\FloatBarrier
os