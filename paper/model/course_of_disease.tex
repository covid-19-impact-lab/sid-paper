\subsection{Course of the Disease}
\label{sub:disease_progression}

The following medical parameters describing the progression of the disease are taken from
systematic reviews (e.g. \citet{He2020}). After an infection occurs, the disease
progresses in the way depicted in Figure~\ref{fig:model_disease_progression}.


First, infected individuals will become infectious after one to five days. About one
third of people remain asymptomatic. The rest develop symptoms about one to two days
after they become infectious. Modeling asymptomatic and pre-symptomatic cases is
important because those people do not reduce their contacts or demand a test and can
potentially infect many other people \citep{Donsimoni2020}. The probability to develop
symptoms with Covid-19 is highly age dependent with 75\% of children not developing
symptoms \citep{Davies2020}.

A small share of symptomatic people will develop strong symptoms that require intensive
care. The exact share and time span is age-dependent. An age-dependent share of intensive
care unit (ICU) patients will die after spending up to 32 days in intensive care.
Moreover, if the ICU capacity was reached, all patients who require intensive care but do
not receive it die.

It would be easy to make the course of disease even more fine-grained. For example, we
could model people who require hospitalization but not intensive care. So far we opted
against that because only the intensive care capacities are feared to become a bottleneck
in Germany.

We allow the progression of the disease to be stochastic in two ways: Firstly, state
changes only occur with a certain probability (e.g. only a fraction of infected
individuals develops symptoms). Secondly, the number of periods for which an individual
remains in a state is drawn randomly. The parameters that govern these processes are
taken from the literature\footnotemark and age-dependent.

\footnotetext{ Detailed information on the calibration of the disease parameters is
    available as part of our
    \href{https://sid-dev.readthedocs.io/en/latest/reference_guides/epi_params.html}{online
    documentation}. }
