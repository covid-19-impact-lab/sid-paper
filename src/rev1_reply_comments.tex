\documentclass[a4paper,11pt]{article}
\usepackage{a4wide}
\usepackage[utf8]{inputenc}
\usepackage[T1]{fontenc}
\usepackage{afterpage,rotating,graphicx,setspace,xcolor}
\usepackage{longtable,booktabs,tabularx}
\usepackage{chngcntr}
\usepackage{eurosym,calc}
\usepackage{amsmath,amssymb,amsfonts,amsthm,delarray}
\usepackage{bm}
\usepackage{bbm}
\usepackage{caption}
\usepackage{subcaption}
\usepackage{xfrac}
\usepackage{listings}
\usepackage{adjustbox}
\usepackage{xr}
\usepackage{xr-hyper}
\usepackage[unicode=true]{hyperref}
\usepackage{afterpage}
\usepackage{clipboard}
% !TeX program = pdflatex
% !TeX TXS-program:compile = txs:///pdflatex/
% !TeX TS-program = pdflatex
% !BIB program = biber
% !TeX TXS-program:bibliography = txs:///biber




%%%%%%%%%%%%%%%%%%%%%%%%%%%%%%%%%%%%%%%%%%%%%%%%
%%  CITATION COMMANDS AND BIBLIOGRAPHY STYLE  %%
%%%%%%%%%%%%%%%%%%%%%%%%%%%%%%%%%%%%%%%%%%%%%%%%


% Science style
\usepackage[style=science, backend=biber, natbib=true, bibencoding=inputenc]{biblatex}

\openclipboard{clipboard-rev1}

\definecolor{darkblue}{rgb}{0, 0, 0.8}
\hypersetup{ colorlinks=true, linkcolor=darkblue, anchorcolor=darkblue,
    citecolor=darkblue, filecolor=darkblue, menucolor=darkblue, runcolor=darkblue,
    urlcolor=darkblue }
\usepackage{enumerate}
\usepackage{caption}
\usepackage{booktabs,caption,rotating}
\usepackage[capposition=top]{floatrow}
% \usepackage{bbm}
\usepackage{epstopdf}
\usepackage{longtable, tabularx, array}
\usepackage{colortbl}

% \usepackage[british]{babel} \usepackage[useregional]{datetime2}
% \DTMlangsetup[en-GB]{showdayofmonth=false}

\usepackage{subcaption}
\usepackage{eurosym}

% \setlength{\marginparwidth}{2cm}
\usepackage[colorinlistoftodos, backgroundcolor=white, prependcaption, textsize=tiny,
    disable]{todonotes}


%%%%PROPOSITION AND PROOF ENVIRONMENTS%%%%
\theoremstyle{plain}
\newtheorem{result}{Result}
\newtheorem{assumption}{Assumption}

\DeclareMathOperator*{\argmin}{arg\,min} \DeclareMathOperator*{\argmax}{arg\,max}

% don't let footnotes split across pages
\interfootnotelinepenalty=10000 \widowpenalty=10000 \clubpenalty=10000

\hyphenation{bet-ween}
\hyphenation{with-out}

\frenchspacing

\newlength{\parindentaux}
\setlength{\parindentaux}{\parindent}

\usepackage{enumitem}
\setlist[itemize]{%
    leftmargin=\parindentaux, listparindent=\parindentaux }

% \usepackage{calc}  % Needed to be able to calculate, e.g., sum of dimensions
%\usepackage{ifthen}

\makeatletter
\newcommand*\@myenumerate{enumerate} \newcommand*\@myitemize{itemize}
\newcommand{\comment}[1]{{\color{black!60}#1}} \newcommand{\response}[1]{{%
		\color{black}%
		\par\frenchspacing%
		\ifx\@currenvir\@myenumerate%
		\addtolength{\leftskip}{-\parindentaux}%
		\rm\noindent #1\par%
		\addtolength{\leftskip}{\parindentaux}%
		\else%
		\ifx\@currenvir\@myitemize%
		\addtolength{\leftskip}{-\parindentaux}%
		\rm\noindent #1\par%
		\addtolength{\leftskip}{\parindentaux}%
		\else%
		\rm\noindent #1\par%
		\fi%
		\fi%
}}
\makeatother

\externaldocument{paper}

\renewcommand{\theequation}{\roman{equation}} \renewcommand{\thetable}{\Roman{table}}
\renewcommand{\thefigure}{\Roman{figure}} \renewcommand{\thesection}{\Roman{section}}


\setlength{\parindent}{0ex}

\bibliography{references.bib}

\begin{document}

\title{\large Reply to the Editor's and Reviewers' comments on: \\[2ex]
    \LARGE The Effectiveness of Testing, Vaccinations, and Contact Restrictions for
    Containing the Covid-19 Pandemic\\[2ex]
    \large Submitted to Scientific Reports \\(Submission ID
    db917d3c-b56a-4129-8eb1-a62a716aaede)\\[-6ex]
}

\author{}
\date{}

\maketitle

\large{\textbf{\textcolor{red}{SUMMARY}}}

\section*{Editors}

\comment{ The title should have a clear, precise scientific meaning and should not
contain a colon. Where possible, the title should be read as one concise sentence. }
\vskip1ex \response{%
    We changed the title from ``The Effectiveness of Strategies to Contain SARS-CoV-2:
    Testing, Vaccinations, and NPIs'' to ``The Effectiveness of Testing, Vaccinations, and Contact Restrictions for Containing the Covid-19 Pandemic''
}


\section*{Reviewer 1}
\comment{The authors of the manuscript presented here have used mathematical models to
examine the impact of different measures to contain the SARS-CoV-2 pandemic. Using
epidemiological data from Germany, the contribution of rapid antigen testing is
considered in particular.

The topic studied here is of great importance for the control of the coronavirus
pandemic. The presentation of the scientific methodology is sufficiently detailed,
especially with the supplementary materials. It allows the reader to follow the research
process in detail. However, I noticed a significant deficiency when reviewing the
submitted work. Therefore, the manuscript should definitely be revised again before
publication: A crucial point of the authors is to point out the special benefit of a
rapid test strategy. However, for the calculation of the corresponding influencing
factors in their mathematical model, the authors simplify, specifically, the sensitivity
of rapid antigen tests in an unacceptable way. In the development of the mathematical
model, apparently only the knowledge available a few months ago on the performance of
rapid antigen tests was taken into account. This is evident from literature references
mentioned above. In part, incorrect references are also used, especially for the
assumption of the high sensitivity of the rapid tests. For example, reference No. 74
explicitly does not refer to rapid antigen tests, but to an immunological laboratory
test method (Quidel SARS Sofia antigen FIA). Especially in light of the benefit of rapid
antigen tests described by the authors, it should be considered that many of the
commercially marketed tests show only extremely poor performance. This has recently been
demonstrated by several publications:
\begin{itemize}
    \item Comparative sensitivity evaluation for 122 CE-marked rapid diagnostic tests
          for SARS-CoV-2 antigen, Germany, September 2020 to April 2021, Scheiblauer,
          Heinrich and Filomena, Angela and Nitsche, Andreas and Puyskens, Andreas and
          Corman, Victor M and Drosten, Christian and Zwirglmaier, Karin and Lange,
          Constanze and Emmerich, Petra and Müller, Michael and Knauer, Olivia and
          Nübling, C Micha, Eurosurveillance, 26, 2100441 (2021),
          \url{https://doi.org/10.2807/1560-7917.ES.2021.26.44.2100441}
    \item Assessment of SARS-CoV-2 rapid antigen tests, Özcürümez, Mustafa, Katsounas,
          Antonios, Holdenrieder, Stefan, von Meyer, Alexander, Renz, Harald and Wölfel,
          Roman, Journal of Laboratory Medicine, vol. 45, no. 3, 2021, pp. 143-148.
          \url{https://doi.org/10.1515/labmed-2021-0036}
\end{itemize}

Especially in the case of a mass application of antigen tests, it is therefore to be
expected that products with lowest sensitivity will also be used in the general
population. The authors should at least model this with an alternative calculation and
discuss it accordingly. Such an improvement would increase the importance of this
manuscript enormously.

\vskip1ex

\response{%
    We would like to thank the reviewer for these insightful comments. First of all, we
    agree that the references we used have become outdated in the meantime. Note that we
    submitted the original manuscript in early September 2021; the analysis had been
    finalised in June of last year. Indeed, in the line of the additional evidence, the
    numbers we use for test sensitivity appear to be in the upper part of plausible
    values. We have followed your suggestion and report detailed robustness analyses in
    the paper and its Appendix.

    Before describing our changes, we would like to thank you for pointing out that the
    test analysed in \citet{Smith2021} is not a rapid test. While this is true, the
    crucial feature in \citet{Smith2021} is that the paper reports the sensitivity over
    the course of an infection. We have not found a comparable study. Getting this
    profile right is important particularly in the beginning of an infection, as we
    detail shortly. At that point, the differences between different methods to assess
    test sensitivity by timing within an infection are relatively small.

    On page~\pageref{r.1.a} of the paper, we cite the references you pointed out
    (numbering of references relates to this document) and note that we conduct a
    robustness analysis:\\[1ex]

    \textit{\Paste{r.1.a}}

    \vskip1ex

    We followed your suggestion and ran additional analyses with more conservative
    assumptions on the sensitivity of rapid tests. The results are very robust. We write
    in the main text on page~\pageref{r.1.b}:\\[1ex]

    \textit{\Paste{r.1.b}}

    \vskip1ex

    The robustness is due to the fact that over the course of an infection, the largest
    differences between the parameters we use in our main analysis and those based on
    imputing values from other studies are found during the later stages of an
    infection. Uncovering an infection that was previously undetected at that stage does
    not have a large effect on subsequent infection dynamics. Our whole analysis in
    Appendix~\ref{subsec:robustness_rapid_test_sensitivity} reads as follows:\\[1ex]

    \textit{\Paste{r.1.b-appendix}}

}

~\\[4ex]

\FloatBarrier

Furthermore, I recommend including a declaration of potential conflicts of interest in
the manuscript. This may be particularly important in light of the significance of the
findings from the study for the continued operation of industrial companies.\\[2ex]

\response{We have added such a declaration at the end of the paper:\\[2ex]

    \textit{\Paste{r.1.c}}
}

}

\clearpage
\section*{Reviewer 2}

\comment{ This is a comprehensive and important study that highlights the relevance of
testing to stop the spread of the pandemic.

I cannot comment on the validity of the underlaying mathematical analysis, but I would
like to comment on some assumptions made, which I think require further clarification or
discussion.

\begin{enumerate}
    \item They use contact-frequency from pre-pandemic diary data. Is  this truly useful
          if behavior of people is already changed by the fact there is a pandemic?

          \response{%

              Many thanks for this comment. Ideally, we would of course use real-time
              data. For example, several studies have used smartphone tracking data for
              this purpose. However, any such data are not detailed enough for our
              requirements because they do not provide any information about
              \textit{who} is meeting with each other. These dimensions make up our
              contact networks, which are absolutely crucial.

              Furthermore, note that we only use the pre-pandemic diary data as the
              starting point. Contact restrictions and individual behavior adjustments
              reduce contacts. These are active throughout the period of analysis and
              they vary with the strictness of measures. In order to do so, it is crucial
              to take the type of contact into account; see for example our analysis of
              potential school closures on Page~\pageref{ref.2.a}. Closed schools mean
              that children have far fewer and parents have somewhat fewer contacts than
              without school closures.

              Finally, the very good fit of our model to official infection rates across
              age groups increases our confidence that we manage to match actual meeting
              patterns. If we were far off for, for example, the number of work
              contacts, we would not be able to match either the infections of
              individuals during working age (35-59) or of those that mostly do not work
              anymore (60-79).
          }
    \item They state that susceptibility is dependent on age. This requires references
          and needs to be further explained. Also, younger populations are highly susceptible,
          they may just not contribute transmission as much. On that same note, infectiousness
          may not be independent of age

          Edward Goldstein, Marc Lipsitch, Muge Cevik, On the Effect of Age on the
          Transmission of SARS-CoV-2 in Households, Schools, and the Community, The
          Journal of Infectious Diseases, Volume 223, Issue 3, 1 February 2021, Pages
          362–369, \url{https://doi.org/10.1093/infdis/jiaa691}

          \response{%
              Many thanks for pointing this out, indeed we had not cited our key
              reference for susceptibilities by age group \citep{Davies2020} in the main
              text. In fact, the paper you suggested \citep{Goldstein2021} comes to
              similar conclusions: \textit{``We found evidence that compared to
                  younger/middle-aged adults, children younger than 10 years have
                  significantly lower estimated susceptibility to SARS-CoV-2 infection,
                  while adults older than 60 years have elevated susceptibility to
                  infection.''} The current version cites both papers on Page~\ref{r.2.b}.

              We base our assumption on infectiousness being independent of age on
              \citet{Jones2021}. That study was the only one we could find that was able
              to differentiate between age differences in susceptibility (given a risk
              contact) vs. age differences in contact patterns. Moreover, they had fine
              grained age bins. That paper reports only slight differences in viral
              loads between German adults and children: ``The least infectious youngest
              children have 78\% (61, 94) of the peak culture probability of adults aged
              45 to 55''. Furthermore, they caution that these lower numbers may be due
              to different sample taking---such as the usage of smaller pediatric swab
              devices---rather than actual lower viral loads and conclude that the
              differences, if existent, are likely not clinically relevant. In fact,
              \citet{Goldstein2021}  reach similar conclusions, writing: \textit{``There
                  is limited evidence in the literature regarding age-related differences in
                  infectivity, although point estimates in several studies suggest that
                  infectivity may increase somewhat with age.''} We are thus confident that
              different assumptions in the range of plausible values would not
              change our results in a quantitatively meaningful way.

          }
    \item It is not accurate that the vaccines stop transmission. This has to be adapted
          accordingly. Also, the authors should state which vaccines they based this
          analysis on, as in which ones were in use in Germany at that time and how does
          that affect the relative transmissibility as it cannot be assumed that all
          vaccines will have the same effect.

          \response{%
              Many thanks, it is of course true that vaccines do not stop transmission
              completely. We do not assume that, either. However, we have not been able
              to find data on the way in which this affects different population groups
              for the wild type and the $\alpha$-variant.

              In our model, we thus simplify the way vaccines reduce transmission risk.
              In particular, we assume that that 75\% of individuals achieve sterile
              immunity through a vaccination while the remaining 25\% remain fully
              susceptible. Firstly, the existing evidence supports the
              assumption that all four vaccines approved in the EU (Biontech/Pfizer,
              Moderna, AstraZeneca and Johnson \& Johnson) provide similar levels of
              protection in the first few months after vaccination against the wild type
              and the $\alpha$ variant (for details see Appendix~\ref{r.2.c}, starting
              on Page~\pageref{r.2.c}), which is the only time period and variant our
              simulations cover. Secondly, given the surprisingly good performance of
              rapid tests we tried to err on the side of being too optimistic with our
              modelling of vaccine performance. This is why we chose the perfect
              protection from vaccination for a large part of the population. Thirdly,
              reinfections are not a big issue for the time period that we look at. The
              incidences were small and the chance of more than one exposure to Covid
              are negligibly small. Thus, the value added by a probabilistic model of
              vaccination protection is very small for the period that we cover. }
\end{enumerate}
}

\begin{refcontext}[sorting=none]  % Sort BIBLIOGRAPHY by appearance.
    \printbibliography[heading=bibintoc]
\end{refcontext}


\end{document}
