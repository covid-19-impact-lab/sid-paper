\subsection{Testing} % (fold)
\label{sub:testing}

Our model includes both PCR tests which are scarce and take time until the result becomes
available and rapid tests which are done after an individual's contacts are determined
but before the contacts take place.

PCR testing consists of three stages. Firstly, we model who demands a PCR test. Demand
functions map from individual characteristics to a probability which is the probability
for this individual to demand a test. There can be multiple demand functions where each
function may describe a different channel. For example, individuals who experience
symptoms or have a risk contact may ask for a test. Or, the ministry of education
requires a negative test result from every teacher every second week. After the
probabilities for each individual and every demand model are computed, individuals who
demand a test as well as the channel is sampled.

The second stage is the allocation phase in which demand and supply for tests are
matched. The number of available tests can be inferred from official data and used to
model shortages in supply. When demand exceeds supply, some individuals might be given
preferred access to tests because of their own vulnerability or their potential to
become a super-spreader.

In the last and third phase, administered tests are processed. This step can become a
bottleneck in the testing process if there are not enough laboratories or necessary
resources available to evaluate all tests.

In our empirical estimation we use a very simplified testing model where the number of
tests to be distributed is calculated from estimates for the ratio of known to all
infections.\footnote{The Dunkelzifferradar project publishes daily estimates of the dark
    figure of infections under \url{https://covid19.dunkelzifferradar.de/}} Using these
estimates as well as data on the test distribution over age groups by the
RKI\footnote{https://ars.rki.de/Content/COVID19/Main.aspx} we allocate tests firstly
among the symptomatic and then randomly allocate tests to newly infected to fit the
German test distribution.

\comment[id=J]{To Do für Appendix an der Stelle: Institutional details:
\begin{itemize}
    \item PCR tests only at doctor's office / test centers.
    \item Fraction PCR tests performed because of contact with infected people is small
          (quarantine mandate more common) -- need citation
    \item Rapid tests available in many pop up test centers, self-administered tests in
          every  drug store / supermarket, prices in early June < 1 Euro
\end{itemize}
}


Rapid tests are modeled much simpler. Every day before individuals have contacts they can
decide to be tested.\comment[id=HM]{If we get econ readers, we should not use language
like this because we do not have a choice model.} For example students that plan to
attend school that day and have not done a rapid test in the last three days get a rapid
test. Then they immediately receive the test result. After they have received their test
result individuals can react to it by reducing their contacts. For example positively
tested individuals may not go to work and reduce their household contacts to some degree.
Who reduces their contacts to what degree depends on a quarantine compliance attribute.

Our rapid tests include false positives and false negatives. The sensitivity of rapid
tests in our model depends on when the individual has or will become infectious. This way
we can account for the fact that rapid tests are likely to be false positive before
infectiousness starts.


% subsection testing (end)
