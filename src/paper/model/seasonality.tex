\subsection{Seasonality}\comment[id=K]{Philipp suggests to write that we choose upper
bounds}
\label{subsec:seasonality}

It is widely acknowledged that the transmission of SARS-CoV-2 is subject to seasonal
influences. Infectiousness is increased in winter when most contacts take place inside
and the immune system is weakened by low levels of vitamin D, dry air and large
temperature swings. For a detailed overview of possible drivers see
\cite{KronfeldSchor2021}.

We follow \cite{Kuehn2020} and \cite{Gavenciak2021} in modeling seasonality in the
transmission of SARS-CoV-2 as a multiplicative factor on infection probabilities. The
factor follows a sine curve that reaches its maximum at January first and its minimum
on June 30.

For simplicity we normalize the factor to reach one at its maximum. Thus the formula of
the seasonality factor is given by:

\begin{equation}
    s_k(t) = 1 + 0.5 \kappa_k  sin \left ( \pi  \left (\frac{1}{2} + \frac{t}{182.5}\right ) \right ) - 0.5 \kappa_k
\end{equation}

Where $\kappa_k$ is difference in the seasonality factor between peak infectiousness
and lowest infectiousness.

The subscript $k$ is needed because the strength of the seasonality effect differs
across contact types: Work, household and school contacts are likely to take place
inside even in summer. Thus they are only subject to seasonality due to factors that
influence the immune system. Other contacts (for example meeting friends and while doing
leisure activities) are mostly happening outside in the summer. Therefore, transmission
via those contacts should have a stronger seasonal pattern.

We calibrate $kappa_{strong}$ to 0.42 and $kappa_{weak}$ to 0.21. This is in line with
\cite{Gavenciak2021} and \cite{Kuehn2020}.
