\section{Introduction}
\label{sec:introduction}

The first wave of the Covid-19 pandemic prompted strict lockdowns and restrictions
across the world. Many countries were able to contain the spread of the virus and to
relax restrictions during midyear. At the same time, the social and economic costs were
enormous. In the second wave, countries are trying to implement more nuanced policies.
For example, Germany imposed a ``lockdown-light'' in November. Some businesses are
closed and certain leisure activities are prohibited but educational facilities remain
open.

Current epidemiological models have not been designed to predict the effect of
fine-grained policies. The models need to be extended for each new policy proposal and
cannot be easily adapted to fast changing environments. This report describes a model
that has been designed from the ground up to predict the effects of contact reducing
policies in real time. It has the following features:

\begin{enumerate}
    \item At the core of the model, people meet people based on a matching algorithm. We
          distinguish various types of contacts. The contact types are households,
          leisure activities, schools, preschools, and nurseries and several types of
          contacts at the workplace. Contact types can be random or recurrent and vary
          in frequency.

    \item Policies can be implemented as shutting down contact types entirely or
          partially. The reduction of contacts can be random or systematic, e.g., to
          allow for essential workers.

    \item Infection probabilities vary across contact types and reflect properties of
          the contact like the location (indoor/outdoor) and the kind of interaction
          (duration, physical contact). The probabilities are independent from the
          number of contacts and thus policy-invariant.

    \item The model achieves a good fit on German data of infection rates even if only
          the infection probabilities are fit to the data and the remaining parameters
          are calibrated from the medical literature and datasets on contact
          frequencies.

    \item High quality Python code for the model is freely available on Github, well
          documented and very flexible\footnotemark. We are actively looking for
          researchers who want to use our model for their projects and apply it to other
          contexts.
\end{enumerate}

\footnotetext{
    The code can be found under \url{https://github.com/covid-19-impact-lab/sid} and the
    documentation with tutorials and background information under
    \url{https://sid-dev.readthedocs.io/}.
}

This report describes the model in an abstract way, but uses many realistic examples
from a version that is specialized to Germany. It is important to note that this
specialization is not baked into the model or the Python code. It is easy to adjust the
model to other countries if data on the number of contacts and a dataset with background
characteristics are available.

More details about the German model as well as applications to currently discussed
German policies can be found in \citet{Dorn2020a}.

In Section~\ref{sec:literature_review}, we give a short overview of epidemiological
models. We continue with a detailed description of our model in Section~\ref{sec:model}
and proceed with a description of model parameters and the estimation in
Section~\ref{sec:calibration_and_estimation}. The model is validated in
Section~\ref{sec:model_validation} by assessing the in-sample fit for reported
infections from August to October and by comparing the out-of-sample fit for the period
from November until the beginning of Christmas for different lockdown scenarios. We
conclude in Section~\ref{sec:conclusion}.
