\subsection{Simulated Rapid Tests}
\label{subsec:results_rapid_test_statistics}

In order to make the most use out of limited data sources on rapid test usage, we model
the number of performed rapid tests as a result of time invariant willingness to do
rapid tests and time varying supply side factors and events that trigger rapid tests.
Thus, the $\pi$ parameters governing when individuals do rapid tests described in
Section~\ref{subsec:rapid_test_demand} are only indirectly related to the number of
rapid tests that are actually performed in the model. When it comes to positive and
negative rapid tests, there is even an additional layer because rapid tests are
imperfectly sensitive and specific.

In this section we look at how rapid tests expanded in our simulations over time and to
what degree they are useful as a screening device despite their imperfections.

We start with the share of the population doing a rapid test and receiving a positive
rapid test over time by the channel through which the test was demanded in
Figures~\ref{fig:rapid_test_demand_by_channel}, \ref{fig:pos_rapid_tests_by_channel},
respectively. Overall, the share of the population getting a rapid test on a given day
increases from 2\% in mid March to over 10\% by May. The work rapid tests are a little
ragged because of public holidays. For education rapid tests both vacations (first half
of April) as well as the opening of schools in May are very visible in the rapid test
demand. Overall, work tests make up the largest fraction of rapid tests. The image is
very similar for the share of positive tests, except that the overall number of positive
tests starts decreasing in May as rapid test expansion comes to a halt and cases fall,
especially the positive share of private rapid tests falls as less and less individuals
are triggered to seek a rapid test because of a risk contact in their household.

\begin{figure}[ht] % Share tested per day
   \centering
   \begin{subfigure}[b]{0.49\textwidth}
      \includegraphics[width=\textwidth]{figures/results/figures/rapid_test_statistics/popshare_tested}
      \caption{Share of the Population Doing a Rapid Test on a Given Day, by Channels}
      \label{fig:rapid_test_demand_by_channel}
   \end{subfigure}
   \hfill
   \begin{subfigure}[b]{0.49\textwidth}
      \includegraphics[width=\textwidth]{figures/results/figures/rapid_test_statistics/popshare_tested_positive}
      \caption{Share of the Population Testing Positive on a Given Day, by Channels}
      \label{fig:pos_rapid_tests_by_channel}
   \end{subfigure}
   \caption{Rapid Test Shares in the Population by Channel}
   \floatfoot{\noindent \textit{Note:}
      Rapid tests in the education setting are demanded by teachers (nursery,
      preschool and school) as well as pupils. After Easter the required frequency of
      tests is increased from once per week to twice per week. Work rapid tests are
      demanded by individuals that still have work contacts, i.e. do not work from
      home. The share of employers offering rapid tests increases over the time frame
      and the frequency of testing is also increased. Private tests are demanded by
      individuals for one of three reasons: having developed symptoms without access
      to a PCR test, having a household member that has tested positive or developed
      symptoms or having planned a weekly meeting with friends.
      Panel~\subref{fig:rapid_test_demand_by_channel} shows the share of the
      population doing a rapid test on a given day.
      Panel~\subref{fig:pos_rapid_tests_by_channel} shows the share of the population
      testing positive on a given day (true and false positives).}
\end{figure}

\FloatBarrier

Next, we show the tests split by whether they are true positive, false positive, true
negative or false negative (see Figure~\ref{fig:rapid_test_results_numbers}) in numbers
per million individuals to make the metric comparable to incidences.

The number of true positives (Figure~\ref{fig:rapid_tests_number_true_positive}) rapidly
increases and peaks at the end of April with over 200 cases per million detected through
rapid tests per day. This means that our model suggests that Germany was able to detect
up to 16,600 cases per day that would have likely gone undetected otherwise. The most
powerful tool for detecting cases are the private rapid tests. This is because a large
share of them are targeted, i.e. triggered by events in the household. However, this
does not mean that rapid tests in the workplace or at school are less important. It is
rather the combination of large scale screening at work and in schools and very
efficient follow up tests whenever those screening tests detected a case. Shapley values
(Figure~\ref{fig:2021_scenarios_decomposition_tests}) take this into account and assign
about 50\% of the overall reduction of case numbers via rapid tests to private rapid
tests with work and school rapid tests accounting for 40\% and 7\%, respectively.

Such a large effect of rapid tests seems to be at odds with the general perception that
they are not very reliable. However, one has to differentiate between the reliability of
one test in isolation and the effect imperfect tests can have when employed at a large
scale. On average our tests have a sensitivity of slightly more than 70\%. This means
they miss almost 30\% of infections among the tested. Of course perfect tests would have
an even larger effect but the relevant number to compare is that up to 200 cases per
million are detected by rapid tests every day which would have otherwise gone
undetected.

This clearly shows that the large effect of rapid tests on the infection dynamic is not
driven by unrealistic assumptions about their sensitivity but rather by the fact that
there was a very large number of infected individuals who did not know they are
infected. Detecting and isolating some of them is enough to slow down the overall
infection dynamic.

\begin{figure}   % Number of True Positive / False Positive / True Negative / False
   Negative
   \centering
   \begin{subfigure}[b]{0.425\textwidth}
      \centering
      \includegraphics[width=\textwidth]{figures/results/figures/rapid_test_statistics/number_true_positive}
      \caption{Number of Discovered Cases Due to Rapid Tests by Channel}
      \label{fig:rapid_tests_number_true_positive}
   \end{subfigure}
   \hfill
   \begin{subfigure}[b]{0.425\textwidth}
      \centering
      \includegraphics[width=\textwidth]{figures/results/figures/rapid_test_statistics/number_false_positive}
      \caption{Number of False Positive Rapid Tests by Channel}
      \label{fig:rapid_tests_number_false_positive}
   \end{subfigure}
   \vskip3ex
   \begin{subfigure}[b]{0.425\textwidth}
      \centering
      \includegraphics[width=\textwidth]{figures/results/figures/rapid_test_statistics/number_true_negative}
      \caption{Number of True Negative Rapid Tests by Channel}
      \label{fig:rapid_tests_number_true_negative}
   \end{subfigure}
   \hfill
   \begin{subfigure}[b]{0.425\textwidth}
      \centering
      \includegraphics[width=\textwidth]{figures/results/figures/rapid_test_statistics/number_false_negative}
      \caption{Number of False Negative Rapid Tests by Channel}
      \label{fig:rapid_tests_number_false_negative}
   \end{subfigure}
   \vskip3ex
   \caption{Simulated Rapid Test Statistics}
   \label{fig:rapid_test_results_numbers}
   \floatfoot{\noindent \textit{Note:}
      Each panel shows the number of rapid tests per million inhabitants that fall
      into the respective category. Private rapid tests are especially good at
      detecting cases but since they are often triggered by rapid tests from other
      channels, the other groups of tests, especially rapid tests at the workplace,
      also play an important role for containing the pandemic. All results are
      averaged over 30 simulation runs. For legibility reasons, all lines are rolling
      7-day averages. }
\end{figure}

\FloatBarrier

A similar picture arises, when looking at the false positive rate, i.e. the share of
positive tests that go to people who are not infected.
Figure~\ref{fig:rapid_tests_false_positive_rate} shows that the false positive rate is
very high. On average 60\% to 93\% of positive tests are received by individuals that
are not infected. The false positive rate increases over time. This is due to the low
prevalence of infections in the population, which falls over time. Again, private rapid
tests are an exception with a much lower false positive rate because those tests are
primarily demanded when there is a high likelihood of being infected. The false negative
rate of 0.2\% looks very low. As discussed above this is deceiving and just a mechanical
consequence of a very low prevalence of the disease and the many rapid tests done by
non-infected people.

\begin{figure} % True Positive / False Positive / True Negative / False Negative Rate
   \centering
   % \begin{subfigure}[b]{0.425\textwidth} \centering
   %     \includegraphics[width=\textwidth]{figures/results/figures/rapid_test_statistics/true_positive_rate}
   %     \caption{Rate of True Positive Rapid Tests by Channel}
   %     \label{fig:rapid_tests_true_positive_rate} \end{subfigure} \hfill
   \begin{subfigure}[b]{0.425\textwidth}
      \centering
      \includegraphics[width=\textwidth]{figures/results/figures/rapid_test_statistics/false_positive_rate}
      \caption{Rate of False Positive Rapid Tests by Channel}
      \label{fig:rapid_tests_false_positive_rate}
   \end{subfigure}
   % \vskip3ex \begin{subfigure}[b]{0.425\textwidth} \centering
   % \includegraphics[width=\textwidth]{figures/results/figures/rapid_test_statistics/true_negative_rate}
   % \caption{Rate of True Negative Rapid Tests by Channel}
   % \label{fig:rapid_tests_true_negative_rate} \end{subfigure}
   \hfill
   \begin{subfigure}[b]{0.425\textwidth}
      \centering
      \includegraphics[width=\textwidth]{figures/results/figures/rapid_test_statistics/false_negative_rate}
      \caption{Rate of False Negative Rapid Tests by Channel}
      \label{fig:rapid_tests_false_negative_rate}
   \end{subfigure}
   \vskip3ex
   \caption{False Positive and False Negative Rates by Channel}
   \label{fig:rapid_test_results_rates}

   \floatfoot{\noindent \textit{Note:}
      The left panel shows the share of positive tests that are given to people who
      are not infected. This share is large as can be expected with a very low
      baseline rate of positive individuals. As the incidence in the population drops,
      the false positive rate increases. An exception are the private rapid tests
      because they are -- especially when the incidence is high -- often triggered by
      events that make it likely that the test taker is infected and therefore their
      false positive rate is much lower. The right panel shows the false negative rate
      in the population, i.e. the share of negative tests done by infected
      individuals. This is very low because there are many truly negative tests in
      times of low incidences and large scale screening tests.}
\end{figure}

\FloatBarrier

